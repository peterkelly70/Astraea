% %%%%%%%%%%%%%%%%%%%%%%%%%%%%%%%%%%%%%%%%%%%%%%%%%%%%%%%%%%%%%%%%%%%%%%% %
%                                                                         %
% Project Gutenberg's An Introduction to Mathematics, by Alfred North Whitehead
%                                                                         %
% This eBook is for the use of anyone anywhere at no cost and with        %
% almost no restrictions whatsoever.  You may copy it, give it away or    %
% re-use it under the terms of the Project Gutenberg License included     %
% with this eBook or online at www.gutenberg.org                          %
%                                                                         %
%                                                                         %
% Title: An Introduction to Mathematics                                   %
%                                                                         %
% Author: Alfred North Whitehead                                          %
%                                                                         %
% Release Date: December 6, 2012 [EBook #41568]                           %
% Most recently updated: June 11, 2021                         %
%                                                                         %
% Language: English                                                       %
%                                                                         %
% Character set encoding: UTF-8                                           %
%                                                                         %
% *** START OF THIS PROJECT GUTENBERG EBOOK AN INTRODUCTION TO MATHEMATICS ***
%                                                                         %
% %%%%%%%%%%%%%%%%%%%%%%%%%%%%%%%%%%%%%%%%%%%%%%%%%%%%%%%%%%%%%%%%%%%%%%% %

\def\ebook{41568}
%%%%%%%%%%%%%%%%%%%%%%%%%%%%%%%%%%%%%%%%%%%%%%%%%%%%%%%%%%%%%%%%%%%%%%
%%                                                                  %%
%% Packages and substitutions:                                      %%
%%                                                                  %%
%% book:     Required.                                              %%
%% inputenc: Latin-1 text encoding. Required.                       %%
%%                                                                  %%
%% ifthen:   Logical conditionals. Required.                        %%
%%                                                                  %%
%% amsmath:  AMS mathematics enhancements. Required.                %%
%% amssymb:  Additional mathematical symbols. Required.             %%
%%                                                                  %%
%% alltt:    Fixed-width font environment. Required.                %%
%%                                                                  %%
%% indentfirst: Indent first paragraph of each section. Optional.   %%
%%                                                                  %%
%% footmisc: Start footnote numbering on each page. Required.       %%
%%                                                                  %%
%% multicol: Multicolumn environment for index. Required.           %%
%% makeidx:  Index. Required.                                       %%
%%                                                                  %%
%% graphicx: Standard interface for graphics inclusion. Required.   %%
%% caption:  Caption customization. Required.                       %%
%%                                                                  %%
%% calc:     Length calculations. Required.                         %%
%%                                                                  %%
%% fancyhdr: Enhanced running headers and footers. Required.        %%
%%                                                                  %%
%% geometry: Enhanced page layout package. Required.                %%
%% hyperref: Hypertext embellishments for pdf output. Required.     %%
%%                                                                  %%
%%                                                                  %%
%% Producer's Comments:                                             %%
%%                                                                  %%
%%   OCR text for this ebook was obtained on Nov. 24, 2012, from    %%
%%   http://archive.org/details/introductiontoma00whitiala.         %%
%%                                                                  %%
%%   Minor changes to the original are noted in this file in three  %%
%%   ways:                                                          %%
%%     1. \Typo{}{} for typographical corrections, showing original %%
%%        and replacement text side-by-side.                        %%
%%     2. \Chg{}{} and \Add{}, for inconsistent/missing punctuation,%%
%%        italicization, and capitalization.                        %%
%%     3. [** TN: Note]s for lengthier or stylistic comments.       %%
%%                                                                  %%
%%                                                                  %%
%% Compilation Flags:                                               %%
%%                                                                  %%
%%   The following behavior may be controlled by boolean flags.     %%
%%                                                                  %%
%%   ForPrinting (false by default):                                %%
%%   If false, compile a screen optimized file (one-sided layout,   %%
%%   blue hyperlinks). If true, print-optimized PDF file: Larger    %%
%%   text block, two-sided layout, black hyperlinks.                %%
%%                                                                  %%
%%                                                                  %%
%% PDF pages: 228 (if ForPrinting set to false)                     %%
%% PDF page size: 4.5 x 6.5" (non-standard)                         %%
%%                                                                  %%
%% Summary of log file:                                             %%
%% * One overfull hbox (0.86pt), one slightly underfull hbox.       %%
%%                                                                  %%
%% Compile History:                                                 %%
%%                                                                  %%
%% December, 2012: (Andrew D. Hwang)                                %%
%%             texlive2011, GNU/Linux                               %%
%%                                                                  %%
%% Command block:                                                   %%
%%                                                                  %%
%%     pdflatex x2                                                  %%
%%     makeindex                                                    %%
%%     pdflatex x2                                                  %%
%%                                                                  %%
%%                                                                  %%
%% December 2012: pglatex.                                          %%
%%   Compile this project with:                                     %%
%%   pdflatex 41568-t.tex ..... TWO times                           %%
%%   makeindex 41568-t.idx                                          %%
%%   pdflatex 41568-t.tex ..... TWO times                           %%
%%                                                                  %%
%%   pdfTeX, Version 3.1415926-1.40.10 (TeX Live 2009/Debian)       %%
%%                                                                  %%
%%%%%%%%%%%%%%%%%%%%%%%%%%%%%%%%%%%%%%%%%%%%%%%%%%%%%%%%%%%%%%%%%%%%%%
\listfiles
\documentclass[12pt,leqno]{book}[2005/09/16]

%%%%%%%%%%%%%%%%%%%%%%%%%%%%% PACKAGES %%%%%%%%%%%%%%%%%%%%%%%%%%%%%%%
\usepackage[utf8]{inputenc}[2006/05/05]

\usepackage{ifthen}[2001/05/26]  %% Logical conditionals

\usepackage{amsmath}[2000/07/18] %% Displayed equations
\usepackage{amssymb}[2002/01/22] %% and additional symbols

\usepackage{alltt}[1997/06/16]   %% boilerplate, credits, license

\IfFileExists{indentfirst.sty}{%
  \usepackage{indentfirst}[1995/11/23]
}{}

\usepackage[perpage,symbol]{footmisc}[2005/03/17]

\usepackage{multicol}[2006/05/18]
\usepackage{makeidx}[2000/03/29]

\usepackage{graphicx}[1999/02/16]%% For diagrams
\usepackage[labelformat=empty,textfont=small]{caption}[2007/01/07]

\usepackage{calc}[2005/08/06]

\usepackage{fancyhdr} %% For running heads

%%%%%%%%%%%%%%%%%%%%%%%%%%%%%%%%%%%%%%%%%%%%%%%%%%%%%%%%%%%%%%%%%
%%%% Interlude:  Set up PRINTING (default) or SCREEN VIEWING %%%%
%%%%%%%%%%%%%%%%%%%%%%%%%%%%%%%%%%%%%%%%%%%%%%%%%%%%%%%%%%%%%%%%%

% ForPrinting=true                     false (default)
% Asymmetric margins                   Symmetric margins
% 1 : 1.6 text block aspect ratio      3 : 4 text block aspect ratio
% Black hyperlinks                     Blue hyperlinks
% Start major marker pages recto       No blank verso pages
%
\newboolean{ForPrinting}

%% UNCOMMENT the next line for a PRINT-OPTIMIZED VERSION of the text %%
%\setboolean{ForPrinting}{true}

%% Initialize values to ForPrinting=false
\newcommand{\Margins}{hmarginratio=1:1}     % Symmetric margins
\newcommand{\HLinkColor}{blue}              % Hyperlink color
\newcommand{\PDFPageLayout}{SinglePage}
\newcommand{\TransNote}{Transcriber's Note}
\newcommand{\TransNoteCommon}{%
  The camera-quality files for this public-domain ebook may be
  downloaded \textit{gratis} at
  \begin{center}
    \texttt{www.gutenberg.org/ebooks/\ebook}.
  \end{center}

  This ebook was produced using scanned images and OCR text generously
  provided by the University of California, Santa Barbara, through the
  Internet Archive.
  \bigskip

  Minor typographical corrections and presentational changes have been
  made without comment.
  \bigskip
}

\newcommand{\TransNoteText}{%
  \TransNoteCommon

  This PDF file is optimized for screen viewing, but may be recompiled
  for printing. Please consult the preamble of the \LaTeX\ source file
  for instructions and other particulars.
}
%% Re-set if ForPrinting=true
\ifthenelse{\boolean{ForPrinting}}{%
  \renewcommand{\Margins}{hmarginratio=2:3} % Asymmetric margins
  \renewcommand{\HLinkColor}{black}         % Hyperlink color
  \renewcommand{\PDFPageLayout}{TwoPageRight}
  \renewcommand{\TransNote}{Transcriber's Note}
  \renewcommand{\TransNoteText}{%
    \TransNoteCommon

    This PDF file is optimized for printing, but may be recompiled for
    screen viewing. Please consult the preamble of the \LaTeX\ source
    file for instructions and other particulars.
  }
}{% If ForPrinting=false, don't skip to recto
  \renewcommand{\cleardoublepage}{\clearpage}
}
%%%%%%%%%%%%%%%%%%%%%%%%%%%%%%%%%%%%%%%%%%%%%%%%%%%%%%%%%%%%%%%%%
%%%%  End of PRINTING/SCREEN VIEWING code; back to packages  %%%%
%%%%%%%%%%%%%%%%%%%%%%%%%%%%%%%%%%%%%%%%%%%%%%%%%%%%%%%%%%%%%%%%%

\ifthenelse{\boolean{ForPrinting}}{%
  \setlength{\paperwidth}{8.5in}%
  \setlength{\paperheight}{11in}%
% 1:1.6
  \usepackage[body={5in,8in},\Margins]{geometry}[2002/07/08]
}{%
  \setlength{\paperwidth}{4.5in}%
  \setlength{\paperheight}{6.5in}%
  \raggedbottom
% 3:4
  \usepackage[body={4.25in,5.6in},\Margins,includeheadfoot]{geometry}[2002/07/08]
}

\providecommand{\ebook}{00000}    % Overridden during white-washing
\usepackage[pdftex,
  hyperfootnotes=false,
  pdftitle={The Project Gutenberg eBook \#\ebook: An Introduction to Mathematics.},
  pdfauthor={Alfred North Whitehead},
  pdfkeywords={University of California, Santa Barbara, The Internet Archive, Andrew D. Hwang},
  pdfstartview=Fit,    % default value
  pdfstartpage=1,      % default value
  pdfpagemode=UseNone, % default value
  bookmarks=true,      % default value
  linktocpage=false,   % default value
  pdfpagelayout=\PDFPageLayout,
  pdfdisplaydoctitle,
  pdfpagelabels=true,
  bookmarksopen=true,
  bookmarksopenlevel=0,
  colorlinks=true,
  linkcolor=\HLinkColor]{hyperref}[2007/02/07]

%% Fixed-width environment to format PG boilerplate %%
\newenvironment{PGtext}{%
\begin{alltt}
\fontsize{8.1}{10}\ttfamily\selectfont}%
{\end{alltt}}

% Errors found during digitization
\newcommand{\Typo}[2]{#2}

% Changes made for consistency; use \newcommand{\Chg}[2]{#1} to match original
\newcommand{\Chg}[2]{#2}
\newcommand{\Add}[1]{\Chg{}{#1}}

%% Miscellaneous global parameters %%
% No hrule in page header
\renewcommand{\headrulewidth}{0pt}

% Loosen spacing
\setlength{\emergencystretch}{1em}
\newcommand{\Loosen}{\spaceskip 0.375em plus 0.75em minus 0.25em}

% Scratch pad for length calculations
\newlength{\TmpLen}

%% Running heads %%
\newcommand{\FlushRunningHeads}{\clearpage\fancyhf{}}
\newcommand{\InitRunningHeads}{%
  \setlength{\headheight}{15pt}
  \pagestyle{fancy}
  \thispagestyle{empty}
  \ifthenelse{\boolean{ForPrinting}}
             {\fancyhead[RO,LE]{\thepage}}
             {\fancyhead[R]{\thepage}}
}

% Uniform style for running heads
\newcommand{\RHeads}[1]{\small\textsc{\MakeUppercase{#1}}}

\newcommand{\SetRunningHeads}[1]{%
  \fancyhead[CO]{\RHeads{Introduction to Mathematics}}%
  \fancyhead[CE]{\RHeads{#1}}%
}

\newcommand{\BookMark}[2]{\phantomsection\pdfbookmark[#1]{#2}{#2}}

%% Major document divisions %%
\newcommand{\PGBoilerPlate}{%
  \pagenumbering{Alph}
  \pagestyle{empty}
  \BookMark{0}{PG Boilerplate.}
}
\newcommand{\FrontMatter}{%
  \cleardoublepage
  \frontmatter
  \BookMark{-1}{Front Matter.}
}
\newcommand{\MainMatter}{%
  \FlushRunningHeads
  \InitRunningHeads
  \mainmatter
  \BookMark{-1}{Main Matter.}
}
\newcommand{\BackMatter}{%
  \FlushRunningHeads
  \InitRunningHeads
  \backmatter
  \BookMark{-1}{Back Matter.}
}
\newcommand{\PGLicense}{%
  \FlushRunningHeads
  \pagenumbering{Roman}
  \InitRunningHeads
  \BookMark{-1}{PG License.}
  \fancyhead[C]{\RHeads{License}}
}

%% ToC formatting %%
\newcommand{\TableofContents}{%
  \FlushRunningHeads
  \InitRunningHeads
  \SetRunningHeads{Contents}
  \BookMark{0}{Contents.}
  \SectTitle{Contents}
}

% Set the section number in a fixed-width box
\newcommand{\ToCBox}[1]{\settowidth{\TmpLen}{XVII.}%
  \makebox[\TmpLen][r]{#1}\hspace*{1em}%
}
% For internal use, to determine if we need the Sect./Page line
\newcommand{\ToCAnchor}{}

% \ToCLine{Chapter}{Title}{page number}
\newcommand{\ToCLine}[3]{%
  \ifthenelse{\not\equal{#1}{}}{%
    \label{toc:#1}%
    \ifthenelse{\not\equal{\pageref{toc:#1}}{\ToCAnchor}}{%
      \renewcommand{\ToCAnchor}{\pageref{toc:#1}}%
      \noindent\makebox[\textwidth][r]{\scriptsize CHAP.\hfill PAGE}\\[8pt]%
    }{}%
    \settowidth{\TmpLen}{999}%
    \noindent\strut\parbox[b]{\textwidth-\TmpLen}{\small%
      \ToCBox{#1}\hangindent4em\MakeUppercase{#2}\dotfill}%
    \makebox[\TmpLen][r]{\pageref{chapter:#1}}%
   }{% else #1 = {}
    \label{toc:#2}%
    \ifthenelse{\not\equal{\pageref{toc:#2}}{\ToCAnchor}}{%
      \renewcommand{\ToCAnchor}{\pageref{toc:#2}}%
      \noindent\makebox[\textwidth][r]{\scriptsize CHAP.\hfill PAGE}\\[8pt]%
    }{}%
    \settowidth{\TmpLen}{999}%
    \noindent\strut\parbox[b]{\textwidth-\TmpLen}{\small%
      \ToCBox{}\hangindent4em\MakeUppercase{#2}\dotfill}%
     \makebox[\TmpLen][r]{\pageref{appendix:#2}}%
   }%
   \smallskip
}

%% Sectional units %%
% Typographical abstraction
\newcommand{\ChapHead}[2]{%
  \SectTitle{#1}
  \SubsectTitle{#2}
}

\newcommand{\SectTitle}[1]{%
  \section*{\centering\large\normalfont\MakeUppercase{#1}}
}

\newcommand{\SubsectTitle}[1]{%
  \subsection*{\centering\normalsize\normalfont\MakeUppercase{#1}}
}

% \Chapter[running head]{Number}{Title}
\newcommand{\Chapter}[3][]{%
  \FlushRunningHeads
  \InitRunningHeads
  \ifthenelse{\equal{#1}{}}{%
    \BookMark{0}{#2: #3.}%
    \SetRunningHeads{#3}%
  }{%
    \ifthenelse{\equal{#2}{VIII}}{
      \BookMark{0}{#2: #1 (Continued).}%
    }{%
      \BookMark{0}{#2: #1.}%
    }
    \SetRunningHeads{#1}%
  }
  \Pagelabel[chapter]{#2}
  \ifthenelse{\equal{#2}{I}}{%
    \section*{\normalfont\centering\Large AN INTRODUCTION TO MATHEMATICS}
  }{}
  \ChapHead{Chapter #2}{#3}
}

\newcommand{\Appendix}[2][]{%
  \FlushRunningHeads
  \InitRunningHeads
  \BookMark{0}{#2.}
  \fancyhead[C]{\RHeads{#2}}
  \Pagelabel[appendix]{#2}%
  \SectTitle{#2}%
  \ifthenelse{\not\equal{#1}{}}{%
    \SubsectTitle{#1}%
  }{}
}

%% Diagrams %%
\newcommand{\Graphic}[2]{%
  \phantomsection\label{fig:#2}%
  \includegraphics[width=#1]{./images/#2.pdf}%
}
% \Figure[width]{figure number}
\newcommand{\DefWidth}{4in}% Default figure width
\newcommand{\Figure}[2][\DefWidth]{%
  \begin{figure}[hbt!]
    \centering
    \phantomsection\label{fig:#2}
    \Graphic{#1}{fig#2}
    \caption{Fig.~#2.}
  \end{figure}\ignorespaces%
}

\newcommand{\Diagram}[1]{%
  \begin{figure}[hbt!]
    \centering
    \Graphic{\DefWidth}{#1}
  \end{figure}\ignorespaces%
}

% Figure labels
\newcommand{\FigNum}[1]{\hyperref[fig:#1]{#1}}
\newcommand{\Fig}[2][Fig.]{\hyperref[fig:#2]{#1~#2}}

\newcommand{\ChapNum}[1]{\hyperref[chapter:#1]{#1}}
\newcommand{\ChapRef}[2][Chapter]{\hyperref[chapter:#2]{\Chg{#1}{Chapter}~#2}}

\newcommand{\Note}[1]{#1\Pagelabel{note#1}}
\newcommand{\Pagelabel}[2][page]{\phantomsection\label{#1:#2}}
\newcommand{\Pageref}[2][p.]{\hyperref[page:#2]{#1~\pageref*{page:#2}}}

% Page separators
\newcommand{\PageSep}[1]{\ignorespaces}

%% Index formatting
\makeindex
\makeatletter
\renewcommand{\@idxitem}{\par\hangindent 30\p@\global\let\idxbrk\nobreak}
\renewcommand\subitem{\idxbrk\@idxitem \hspace*{12\p@}\let\idxbrk\relax}
\renewcommand{\indexspace}{\par\penalty-3000 \vskip 10pt plus5pt minus3pt\relax}

\renewenvironment{theindex}{%
  \setlength\columnseprule{0.5pt}\setlength\columnsep{18pt}%
  \begin{multicols}{2}[{\FlushRunningHeads%
      \InitRunningHeads%
      \BookMark{0}{Index.}%
      \fancyhead[C]{\RHeads{Index}}%
      \Pagelabel[appendix]{Index}%
      \SectTitle{Index}\small}]%
    \setlength\parindent{0pt}\setlength\parskip{0pt plus 0.3pt}%
    \let\item\@idxitem\raggedright%
  }{%
  \end{multicols}\normalsize\FlushRunningHeads
}
\makeatother

\newcommand{\EtSeq}[1]{\hyperpage{#1}\,\textit{et~seqq.}}

% Miscellaneous textual conveniences (N.B. \emph, not \textit)
\newcommand{\Cf}{\emph{Cf.}}
\newcommand{\cf}{\emph{cf.}}
\newcommand{\eg}{\emph{e.g.}}
\newcommand{\ie}{\emph{i.e.}}
\newcommand{\viz}{\emph{viz.}}

\newcommand{\First}[1]{\textsc{#1}}
\newcommand{\Title}[1]{\textit{#1}}
\newcommand{\Foreign}[1]{\textit{#1}}

% Small-caps A.D. and B.C.
\newcommand{\SCAbbrev}[3]{%
  \ifthenelse{\equal{#3}{.}}%
  {\textsc{\MakeLowercase{#1.#2}.}}%
  {\textsc{\MakeLowercase{#1.#2}.}\@#3}%
}

\newcommand{\AD}[1]{\SCAbbrev{A}{D}{#1}}
\newcommand{\BC}[1]{\SCAbbrev{B}{C}{#1}}


%% Miscellaneous mathematical formatting %%
\DeclareInputMath{176}{{}^{\circ}}
\DeclareInputMath{183}{\cdot}
\newcommand{\Strut}[1][12pt]{\rule{0pt}{#1}}

% Cross-ref-able equation tags
\newcommand{\Tag}[1]{\tag*{\quad\ensuremath{#1}}}
\newcommand{\Eq}[1]{\ensuremath{#1}}

%%%%%%%%%%%%%%%%%%%%%%%% START OF DOCUMENT %%%%%%%%%%%%%%%%%%%%%%%%%%
\begin{document}
%% PG BOILERPLATE %%
\PGBoilerPlate
\begin{center}
\begin{minipage}{\textwidth}
\small
\begin{PGtext}
Project Gutenberg's An Introduction to Mathematics, by Alfred North Whitehead

This eBook is for the use of anyone anywhere at no cost and with
almost no restrictions whatsoever.  You may copy it, give it away or
re-use it under the terms of the Project Gutenberg License included
with this eBook or online at www.gutenberg.org


Title: An Introduction to Mathematics

Author: Alfred North Whitehead

Release Date: December 6, 2012 [EBook #41568]
Most recently updated: June 11, 2021

Language: English

Character set encoding: UTF-8     

*** START OF THIS PROJECT GUTENBERG EBOOK AN INTRODUCTION TO MATHEMATICS ***
\end{PGtext}
\end{minipage}
\end{center}
\newpage
%% Credits and transcriber's note %%
\begin{center}
\begin{minipage}{\textwidth}
\begin{PGtext}
Produced by Andrew D. Hwang. (This ebook was produced using
OCR text generously provided by the University of
California, Santa Barbara, through the Internet Archive.)
\end{PGtext}
\end{minipage}
\vfill
\end{center}

\begin{minipage}{0.85\textwidth}
\small
\BookMark{0}{Transcriber's Note.}
\subsection*{\centering\normalfont\scshape%
\normalsize\MakeLowercase{\TransNote}}%

\raggedright
\TransNoteText
\end{minipage}
%%%%%%%%%%%%%%%%%%%%%%%%%%% FRONT MATTER %%%%%%%%%%%%%%%%%%%%%%%%%%
\PageSep{i}
\FrontMatter
%[** TN: Publisher's front matter]
\noindent\footnotesize HOME UNIVERSITY LIBRARY \\
OF MODERN KNOWLEDGE
\vfill

\begin{center}
\Large AN INTRODUCTION TO \\
MATHEMATICS
\medskip

\normalsize
\textsc{By A. N. WHITEHEAD, Sc.D., F.R.S.}
\vfill

\footnotesize
\scshape London \\
{\normalsize WILLIAMS \& NORGATE} \\[6pt]
\rule{0.5in}{0.5pt} \\[6pt]
HENRY HOLT \& Co., New York \\
Canada: WM. BRIGGS, Toronto \\
India: R. \& T. WASHBOURNE, Ltd.
\end{center}
\normalsize
\PageSep{ii}
\iffalse
HOME
UNIVERSITY
LIBRARY
OF
MODERN KNOWLEDGE

Editors:

HERBERT FISHER, M.A.. F.B.A.

PROF. GILBERT MURRAY, D.LlTT.,
LL.D., F.B.A.

PROF. J. ARTHUR THOMSON, M.A.

PROF. WILLIAM T. BREWSTER, M.A.

\Add{(}COLUMBIA UNIVERSITY, U.S.A.)

NEW YORK

HENRY HOLT AND COMPANY
\PageSep{iii}
AN
INTRODUCTION
TO
MATHEMATICS

BY
A. N. WHITEHEAD,
Sc.D., F.R.S.,

AUTHOR OF ``UNIVERSAL ALGEBRA,'' JOINT
AUTHOR OF ``PRINCIPIA MATHEMATICA''

NEW AND REVISED EDITION

LONDON
WILLIAMS AND NORGATE
\PageSep{iv}
PRINTED BY

HALELL, WATSON AND VINEY, LD.,
LONDON AND AYLESBURY.
\fi
\PageSep{v}
\TableofContents

%CHAP.                         PAGE

\ToCLine{I}{The Abstract Nature of Mathematics}{7}

\ToCLine{II}{Variables}{15}

\ToCLine{III}{Methods of Application}{25}

\ToCLine{IV}{Dynamics}{42}

\ToCLine{V}{The Symbolism of Mathematics}{58}

\ToCLine{VI}{Generalizations of Number}{71}

\ToCLine{VII}{Imaginary Numbers}{87}

\ToCLine{VIII}{Imaginary Numbers (Continued)}{101}

\ToCLine{IX}{Coordinate Geometry}{112}

\ToCLine{X}{Conic Sections}{128}

\ToCLine{XI}{Functions}{145}

\ToCLine{XII}{Periodicity in Nature}{164}
\PageSep{vi}

%CHAP.                         PAGE
\ToCLine{XIII}{Trigonometry}{173}

\ToCLine{XIV}{Series}{194}

\ToCLine{XV}{The Differential Calculus}{217}

\ToCLine{XVI}{Geometry}{236}

\ToCLine{XVII}{Quantity}{245}

\ToCLine{}{Notes}{250}

\ToCLine{}{Bibliography}{251}

\ToCLine{}{Index}{253}
\PageSep{7}
\MainMatter
% [** TN: Text printed by \Chapter macro]
% AN INTRODUCTION TO
% MATHEMATICS

\Chapter[Nature of Mathematics]{I}{The Abstract Nature of Mathematics}

\First{The} study of mathematics is apt to commence
in disappointment. The important
applications of the science, the theoretical
interest of its ideas, and the logical rigour of
its methods, all generate the expectation of
a speedy introduction to processes of interest.
We are told that by its aid the stars are
weighed and the billions of molecules in a
drop of water are counted. Yet, like the
ghost of Hamlet's father, this great science
eludes the efforts of our mental weapons
to grasp it---``\,'Tis here, 'tis there, 'tis
gone''---and what we do see does not suggest
the same excuse for illusiveness as sufficed
for the ghost, that it is too noble for
our gross methods. ``A show of violence,''
if ever excusable, may surely be ``offered''
to the trivial results which occupy the
\PageSep{8}
pages of some elementary mathematical
treatises.

The reason for this failure of the science to
live up to its reputation is that its fundamental
ideas are not explained to the student
disentangled from the technical procedure
which has been invented to facilitate their
exact presentation in particular instances.
Accordingly, the unfortunate learner finds
himself struggling to acquire a knowledge of
a mass of details which are not illuminated
by any general conception. Without a doubt,
technical facility is a first requisite for valuable
mental activity: we shall fail to appreciate
the rhythm of Milton, or the passion of
Shelley, so long as we find it necessary to
spell the words and are not quite certain of
the forms of the individual letters. In this
sense there is no royal road to learning. But
it is equally an error to confine attention to
technical processes, excluding consideration
of general ideas. Here lies the road to
pedantry.

The object of the following Chapters is not
to teach mathematics, but to enable students
from the very beginning of their course to
know what the science is about, and why it is
necessarily the foundation of exact thought
as applied to natural phenomena. All allusion
in what follows to detailed deductions
in any part of the science will be inserted
\PageSep{9}
merely for the purpose of example, and care
will be taken to make the general argument
comprehensible, even if here and there some
technical process or symbol which the reader
does not understand is cited for the purpose
of illustration.

The first acquaintance which most people
\index{Abstractness (\emph{defined})}%
have with mathematics is through arithmetic.
That two and two make four is usually taken
as the type of a simple mathematical proposition
which everyone will have heard of.
Arithmetic, therefore, will be a good subject
to consider in order to discover, if possible,
the most obvious characteristic of the science.
Now, the first noticeable fact about arithmetic
is that it applies to everything, to tastes and
to sounds, to apples and to angels, to the
ideas of the mind and to the bones of the
body. The nature of the things is perfectly
indifferent, of all things it is true that two
and two make four. Thus we write down as
the leading characteristic of mathematics
that it deals with properties and ideas
which are applicable to things just because
they are things, and apart from any particular
feelings, or emotions, or sensations, in any
way connected with them. This is what
is meant by calling mathematics an abstract
science.

The result which we have reached deserves
attention. It is natural to think that an
\PageSep{10}
abstract science cannot be of much importance
in the affairs of human life, because it
has omitted from its consideration everything
of real interest. It will be remembered
that Swift, in his description of Gulliver's
\index{Swift}%
voyage to Laputa, is of two minds on this
\index{Laputa}%
point. He describes the mathematicians of
that country as silly and useless dreamers,
whose attention has to be awakened by
flappers. Also, the mathematical tailor measures
his height by a quadrant, and deduces
his other dimensions by a rule and compasses,
producing a suit of very ill-fitting clothes.
On the other hand, the mathematicians of
Laputa, by their marvellous invention of the
magnetic island floating in the air, ruled the
country and maintained their ascendency
over their subjects. Swift, indeed, lived at
a time peculiarly unsuited for gibes at contemporary
mathematicians. Newton's \Title{Principia}
\index{Newton}%
had just been written, one of the great
forces which have transformed the modern
world. Swift might just as well have laughed
at an earthquake.

But a mere list of the achievements of
mathematics is an unsatisfactory way of
arriving at an idea of its importance. It is
worth while to spend a little thought in
getting at the root reason why mathematics,
because of its very abstractness, must always
remain one of the most important topics
\PageSep{11}
for thought. Let us try to make clear to
ourselves why explanations of the order of
events necessarily tend to become mathematical.

Consider how all events are interconnected.
When we see the lightning, we listen for the
thunder; when we hear the wind, we look
for the waves on the sea; in the chill autumn,
the leaves fall. Everywhere order reigns, so
that when some circumstances have been
noted we can foresee that others will also be
present. The progress of science consists in
observing these interconnections and in showing
with a patient ingenuity that the events
of this evershifting world are but examples of
a few general connections or relations called
laws. To see what is general in what is particular
and what is permanent in what is
transitory is the aim of scientific thought. In
the eye of science, the fall of an apple, the
motion of a planet round a sun, and the clinging
of the atmosphere to the earth are all
seen as examples of the law of gravity. This
possibility of disentangling the most complex
evanescent circumstances into various examples
of permanent laws is the controlling
idea of modern thought.

Now let us think of the sort of laws which
we want in order completely to realize this
scientific ideal. Our knowledge of the particular
facts of the world around us is gained
\PageSep{12}
from our sensations. We see, and hear, and
taste, and smell, and feel hot and cold, and
push, and rub, and ache, and tingle. These
are just our own personal sensations: my
toothache cannot be your toothache, and my
sight cannot be your sight. But we ascribe
the origin of these sensations to relations between
the things which form the external
world. Thus the dentist extracts not the
toothache but the tooth. And not only so,
we also endeavour to imagine the world as
one connected set of things which underlies
all the perceptions of all people. There is not
one world of things for my sensations and another
for yours, but one world in which we
both exist. It is the same tooth both for
dentist and patient. Also we hear and we
touch the same world as we see.

It is easy, therefore, to understand that we
want to describe the connections between
these external things in some way which does
not depend on any particular sensations, nor
even on all the sensations of any particular
person. The laws satisfied by the course of
events in the world of external things are to
be described, if possible, in a neutral universal
fashion, the same for blind men as for
deaf men, and the same for beings with
faculties beyond our ken as for normal human
beings.

But when we have put aside our immediate
\PageSep{13}
\index{Abstractness (\emph{defined})}%
\index{Dynamical Explanation}%
sensations, the most serviceable part---from
its clearness, definiteness, and universality---of
what is left is composed of our general ideas
of the abstract formal properties of things;
in fact, the abstract mathematical ideas mentioned
above. Thus it comes about that,
step by step, and not realizing the full meaning
of the process, mankind has been led to
search for a mathematical description of the
properties of the universe, because in this way
only can a general idea of the course of events
be formed, freed from reference to particular
persons or to particular types of sensation.
For example, it might be asked at dinner:
``What was it which underlay my sensation
of sight, yours of touch, and his of taste
and smell?''\ the answer being ``an apple.''
But in its final analysis, science seeks to
describe an apple in terms of the positions
and motions of molecules, a description which
ignores me and you and him, and also ignores
sight and touch and taste and smell.
Thus mathematical ideas, because they
are abstract, supply just what is wanted
for a scientific description of the course of
events.

This point has usually been misunderstood,
%[** TN: Entry listed on p. 18 in the original]
\index{Pythagoras}%
from being thought of in too narrow a way.
Pythagoras had a glimpse of it when he proclaimed
that number was the source of all
things. In modern times the belief that the
\PageSep{14}
ultimate explanation of all things was to be
found in Newtonian mechanics was an adumbration
of the truth that all science as it
grows towards perfection becomes mathematical
\index{Dynamical Explanation}%
in its ideas.
\PageSep{15}


\Chapter{II}{Variables}

\First{Mathematics} as a science commenced when
first someone, probably a Greek, proved propositions
about \emph{any} things or about \emph{some}
things, without specification of definite particular
things. These propositions were first
enunciated by the Greeks for geometry; and,
accordingly, geometry was the great Greek
mathematical science. After the rise of geometry
centuries passed away before algebra
made a really effective start, despite some
faint anticipations by the later Greek mathematicians.

The ideas of \emph{any} and of \emph{some} are introduced
into algebra by the use of letters, instead
of the definite numbers of arithmetic.
Thus, instead of saying that $2 + 3 = 3 + 2$, in
algebra we generalize and say that, if $x$ and~$y$
stand for \emph{any} two numbers, then $x + y = y + x$.
Again, in the place of saying that $3 > 2$, we
generalize and say that if $x$~be \emph{any} number
there exists \emph{some} number (or numbers)~$y$ such
that $y > x$. We may remark in passing that
this latter assumption---for when put in its
strict ultimate form it is an assumption---is
\PageSep{16}
of vital importance, both to philosophy and
to mathematics; for by it the notion of infinity
is introduced. Perhaps it required the
introduction of the arabic numerals, by which
the use of letters as standing for definite
numbers has been completely discarded in
mathematics, in order to suggest to mathematicians
the technical convenience of the
use of letters for the ideas of \emph{any} number
and \emph{some} number. The Romans would have
stated the number of the year in which this
is written in the form MDCCCCX., whereas
we write it~1910, thus leaving the letters for
the other usage. But this is merely a speculation.
After the rise of algebra the differential
calculus was invented by Newton and
\index{Newton}%
Leibniz, and then a pause in the progress
\index{Leibniz}%
of the philosophy of mathematical thought
occurred so far as these notions are concerned;
and it was not till within the last few years
that it has been realized how fundamental
\emph{any} and \emph{some} are to the very nature of mathematics,
with the result of opening out still
further subjects for mathematical exploration.

Let us now make some simple algebraic
statements, with the object of understanding
exactly how these fundamental ideas occur.

\Eq{(1)} For \emph{any} number~$x$, $x + 2 = 2 + x$;

\Eq{(2)} For \emph{some} number~$x$, $x + 2 = 3$;

\Eq{(3)} For \emph{some} number~$x$, $x + 2 > 3$.
\PageSep{17}

The first point to notice is the possibilities
contained in the meaning of \emph{some}, as here
used. Since $x + 2 = 2 + x$ for any number~$x$, it
is true for \emph{some} number~$x$. Thus, as here used,
\emph{any} implies \emph{some} and \emph{some} does not exclude
\emph{any}. Again, in the second example, there is,
in fact, only one number~$x$, such that $x + 2 = 3$,
namely only the number~$1$. Thus the \emph{some}
may be one number only. But in the third\Typo{,}{}
example, any number~$x$ which is greater than~$1$
gives $x + 2 > 3$. Hence there are an infinite
number of numbers which answer to the \emph{some}
number in this case. Thus \emph{some} may be anything
between \emph{any} and \emph{one only}, including
both these limiting cases.

It is natural to supersede the statements
\Eq{(2)} and \Eq{(3)} by the questions:

\Eq{(2')} For what number~$x$ is $x + 2 = 3$;

\Eq{(3')} For what numbers~$x$ is $x + 2 > 3$.

%[** TN: No indent in the original]
Considering~\Eq{(2')}, $x + 2 = 3$ is an equation, and
\index{Unknown, The}%
it is easy to see that its solution is $x = 3 - 2 = 1$.
When we have asked the question implied in
the statement of the equation $x + 2 = 3$, $x$~is
called the unknown. The object of the solution
of the equation is the determination of
the unknown. Equations are of great importance
in mathematics, and it seems as
%[** TN: thoroughgoing hyphenated in the original; only instance]
though \Eq{(2')}~exemplified a much more thoroughgoing
and fundamental idea than the original
statement~\Eq{(2)}. This, however, is a complete
mistake. The idea of the undetermined
\PageSep{18}
``variable'' as occurring in the use of ``some''
or ``any'' is the really important one in
mathematics; that of the ``unknown'' in an
equation, which is to be solved as quickly as
possible, is only of subordinate use, though
of course it is very important. One of the
causes of the apparent triviality of much of
elementary algebra is the preoccupation of
the text-books with the solution of equations.
The same remark applies to the solution of
the inequality~\Eq{(3')} as compared to the original
statement~\Eq{(3)}.

But the majority of interesting formulæ,
\index{Relations between Variables|EtSeq}%
\index{Variable, The}%
especially when the idea of \emph{some} is present,
involve more than one variable. For example,
the consideration of the pairs of numbers
$x$ and~$y$ (fractional or integral) which
satisfy $x + y = 1$ involves the idea of two correlated
variables, $x$~and~$y$. When two variables
are present the same two main types of
statement occur. For example, \Eq{(1)}~for
\emph{any} pair of numbers, $x$~and~$y$, $x + y = y + x$,
and \Eq{(2)}~for \emph{some} pairs of numbers, $x$~and~$y$,
$x + y = 1$.

The second type of statement invites consideration
of the aggregate of pairs of numbers
which are bound together by some fixed
relation---in the case given, by the relation
$x + y = 1$. One use of formulæ of the first
type, true for \emph{any} pair of numbers, is that by
them formulæ of the second type can be
\PageSep{19}
thrown into an indefinite number of equivalent
forms. For example, the relation $x + y = 1$
is equivalent to the relations
\[
y + x = 1,\quad
(x - y) + 2y = 1,\quad
6x + 6y = 6,
\]
and so on. Thus a skilful mathematician
uses that equivalent form of the relation
under consideration which is most convenient
for his immediate purpose.

It is not in general true that, when a pair
of terms satisfy some fixed relation, if one of
the terms is given the other is also definitely
determined. For example, when $x$ and~$y$
satisfy $y^{2} = x$, if $x = 4$, $y$~can be~$±2$, thus,
for any positive value of~$x$ there are alternative
values for~$y$. Also in the relation
$x + y > 1$, when either $x$ or~$y$ is given, an
indefinite number of values remain open for
the other.

Again there is another important point to
be noticed. If we restrict ourselves to positive
numbers, integral or fractional, in considering
the relation $x + y = 1$, then, if either
$x$ or~$y$ be greater than~$1$, there is no positive
number which the other can assume so as to
satisfy the relation. Thus the ``field'' of
the relation for~$x$ is restricted to numbers less
than~$1$, and similarly for the ``field'' open
to~$y$. Again, consider integral numbers only,
positive or negative, and take the relation
\PageSep{20}
$y^{2} = x$, satisfied by pairs of such numbers.
Then whatever integral value is given to~$y$,
$x$~can assume one corresponding integral
value. So the ``field'' for~$y$ is unrestricted
among these positive or negative integers.
But the ``field'' for~$x$ is restricted in two
ways. In the first place $x$~must be positive,
and in the second place, since $y$~is to be integral,
$x$~must be a perfect square. Accordingly,
the ``field'' of~$x$ is restricted to the set
of integers $1^{2}$, $2^{2}$, $3^{2}$, $4^{2}$, and so on, \ie, to $1$,
$4$, $9$, $16$, and so on.

The study of the general properties of a
relation between pairs of numbers is much
facilitated by the use of a diagram constructed
as follows:
\Figure[3.5in]{1}

Draw two lines $OX$ and $OY$ at right angles;
let any number~$x$ be represented by $x$~units
\PageSep{21}
(in any scale) of length along~$OX$, any number~$y$
by $y$~units (in any scale) of length along~$OY$.
Thus if $OM$, along~$OX$, be $x$~units in
length, and $ON$, along~$OY$, be $y$~units in length,
by completing the parallelogram $OMPN$ we
find a point~$P$ which corresponds to the pair
of numbers $x$~and~$y$. To each point there
corresponds one pair of numbers, and to each
pair of numbers there corresponds one point.
The pair of numbers are called the coordinates
of the point. Then the points
whose coordinates satisfy some fixed relation
can be indicated in a convenient way,
by drawing a line, if they all lie on a line,
or by shading an area if they are all points
in the area. If the relation can be represented
by an equation such as $x + y = 1$, or
$y^{2} = x$, then the points lie on a line, which is
straight in the former case and curved in
the latter. For example, considering only
positive numbers, the points whose coordinates
satisfy $x + y = 1$ lie on the straight
line~$AB$ in \Fig{1}, where $0A = 1$ and $OB = 1$.
Thus this segment of the straight line~$AB$
gives a pictorial representation of the properties
of the relation under the restriction to
positive numbers.

Another example of a relation between two
variables is afforded by considering the variations
in the pressure and volume of a given
mass of some gaseous substance---such as air
\PageSep{22}
or coal-gas or steam---at a constant temperature.
Let $v$~be the number of cubic feet in
its volume and $p$~its pressure in lb.\ weight
per square inch. Then the law, known as
Boyle's law, expressing the relation between
$p$ and~$v$ as both vary, is that the product~$pv$
is constant, always supposing that the
temperature does not alter. Let us suppose,
for example, that the quantity of the gas
and its other circumstances are such that
we can put $pv = 1$ (the exact number on
the right-hand side of the equation makes
no essential difference).
\Figure{2}

Then in \Fig{2} we take two lines, $OV$ and~$OP$,
at right angles and draw~$OM$ along~$OV$
to represent $v$~units of volume, and $ON$ along~$OP$
\PageSep{23}
to represent $p$~units of pressure. Then
the point~$Q$, which is found by completing the
parallelogram $OMQN$, represents the state of
the gas when its volume is $v$~cubic feet and its
pressure is $p$~lb.\ weight per square inch. If
the circumstances of the portion of gas considered
are such that $pv = 1$, then all these
points~$Q$ which correspond to any possible
state of this portion of gas must lie on the
curved line $ABC$, which includes all points
for which $p$~and $v$ are positive, and $pv = 1$.
Thus this curved line gives a pictorial representation
of the relation holding between the
volume and the pressure. When the pressure
is very big the corresponding point~$Q$ must
be near~$C$, or even beyond~$C$ on the undrawn
part of the curve; then the volume will be
very small. When the volume is big $Q$~will
be near to~$A$, or beyond~$A$; and then the
pressure will be small. Notice that an engineer
or a physicist may want to know the
particular pressure corresponding to some
definitely assigned volume. Then we have
the case of determining the \emph{unknown}~$p$ when
\index{Unknown, The}%
$v$~is a known number. But this is only in
particular cases. In considering generally
the properties of the gas and how it will behave,
he has to have in his mind the general
form of the whole curve $ABC$ and its general
properties. In other words the really fundamental
idea is that of the pair of \emph{variables}
\PageSep{24}
satisfying the relation $pv = 1$. This example
illustrates how the idea of \emph{variables} is fundamental,
\index{Variable, The}%
both in the applications as well as in
the theory of mathematics.
\PageSep{25}


\Chapter{III}{Methods of Application}

\First{The} way in which the idea of variables
satisfying a relation occurs in the applications
of mathematics is worth thought, and by
devoting some time to it we shall clear up
our thoughts on the whole subject.

Let us start with the simplest of examples:---Suppose
that building costs $1$\textit{s.}\ per cubic
foot and that $20$\textit{s.}\ make~£$1$. Then in all
the complex circumstances which attend the
building of a new house, amid all the various
sensations and emotions of the owner, the
architect, the builder, the workmen, and the
onlookers as the house has grown to completion,
this fixed correlation is by the law
assumed to hold between the cubic content
and the cost to the owner, namely that if $x$~be
the number of cubic feet, and £$y$~the cost,
then $20y = x$. This correlation of $x$~and $y$ is
assumed to be true for the building of any
house by any owner. Also, the volume of
the house and the cost are not supposed to
have been perceived or apprehended by any
particular sensation or faculty, or by any
\PageSep{26}
particular man. They are stated in an abstract
general way, with complete indifference
to the owner's state of mind when he has
to pay the bill.

Now think a bit further as to what all this
means. The building of a house is a complicated
set of circumstances. It is impossible
to begin to apply the law, or to test
it, unless amid the general course of events
it is possible to recognize a definite set of
occurrences as forming a particular instance
of the building of a house. In short, we must
know a house when we see it, and must recognize
the events which belong to its building.
Then amidst these events, thus isolated in
idea from the rest of nature, the two elements
of the cost and cubic content must be determinable;
and when they are both determined,
if the law be true, they satisfy the general
formula
\[
20y = x.
\]
But is the law true? Anyone who has had
much to do with building will know that we
have here put the cost rather high. It is
only for an expensive type of house that it
will work out at this price. This brings out
another point which must be made clear.
While we are making mathematical calculations
connected with the formula $20y = x$, it
is indifferent to us whether the law be true or
\PageSep{27}
false. In fact, the very meanings assigned
to $x$~and~$y$, as being a number of cubic feet
and a number of pounds sterling, are indifferent.
During the mathematical investigation
we are, in fact, merely considering the
properties of this correlation between a pair
of variable numbers $x$ and~$y$. Our results
will apply equally well, if we interpret $y$ to
mean a number of fishermen and $x$~the number
of fish caught, so that the assumed law
is that on the average each fisherman catches
twenty fish. The mathematical certainty of
the investigation only attaches to the results
considered as giving properties of the correlation
$20y = x$ between the variable pair of
numbers $x$ and~$y$. There is no mathematical
certainty whatever about the cost of the
actual building of any house. The law is not
quite true and the result it gives will not be
quite accurate. In fact, it may well be hopelessly
wrong.

Now all this no doubt seems very obvious.
But in truth with more complicated instances
there is no more common error than to assume
that, because prolonged and accurate mathematical
calculations have been made, the
application of the result to some fact of
nature is absolutely certain. The conclusion
of no argument can be more certain than the
assumptions from which it starts. All mathematical
calculations about the course of
\PageSep{28}
nature must start from some assumed law of
nature, such, for instance, as the assumed
law of the cost of building stated above.
Accordingly, however accurately we have
calculated that some event must occur, the
doubt always remains---Is the law true? If
the law states a precise result, almost certainly
it is not precisely accurate; and thus
even at the best the result, precisely as calculated,
is not likely to occur. But then we
have no faculty capable of observation with
ideal precision, so, after all, our inaccurate
laws may be good enough.

We will now turn to an actual case, that
of Newton and the Law of Gravity. This law
states that any two bodies attract one another
with a force proportional to the product
of their masses, and inversely proportional to
the square of the distance between them.
Thus if $m$~and~$M$ are the masses of the two
bodies, reckoned in lbs.\ say, and $d$~miles is
the distance between them, the force on either
body, due to the attraction of the other and
directed towards it, is proportional to~$\dfrac{mM}{d^{2}}$;
thus this force can be written as equal to
$\dfrac{kmM}{d^{2}}$, where $k$~is a definite number depending
on the absolute magnitude of this attraction
and also on the scale by which we choose to
measure forces. It is easy to see that, if we
\PageSep{29}
wish to reckon in terms of forces such as the
weight of a mass of $1$~lb., the number which
$k$~represents must be extremely small; for
when $m$~and $M$ and~$d$ are each put equal to~$1$,
$\dfrac{kmM}{d^{2}}$~becomes the gravitational attraction
of two equal masses of $1$~lb.\ at the distance of
one mile, and this is quite inappreciable.

However, we have now got our formula for
the force of attraction. If we call this force~$F$,
it is $F = k\dfrac{mM}{d^{2}}$, giving the correlation between
the variables $F$,~$m$,~$M$, and~$d$. We all
know the story of how it was found out.
Newton, it states, was sitting in an orchard
and watched the fall of an apple, and then
the law of universal gravitation burst upon
\index{Gravitation}%
his mind. It may be that the final formulation
of the law occurred to him in an
orchard, as well as elsewhere---and he must
have been somewhere. But for our purposes
it is more instructive to dwell upon the vast
amount of preparatory thought, the product
of many minds and many centuries, which
was necessary before this exact law could be
formulated. In the first place, the mathematical
habit of mind and the mathematical
procedure explained in the previous two
chapters had to be generated; otherwise
Newton could never have thought of a formula
representing the force between \emph{any} two masses
\PageSep{30}
at \emph{any} distance. Again, what are the meanings
\index{Distance}%
of the terms employed, Force, Mass, Distance?
\index{Force}%
\index{Mass}%
Take the easiest of these terms,
Distance. It seems very obvious to us to
conceive all material things as forming a definite
geometrical whole, such that the distances
of the various parts are measurable in
terms of some unit length, such as a mile or
a yard. This is almost the first aspect of a
material structure which occurs to us. It is
the gradual outcome of the study of geometry
and of the theory of measurement. Even
now, in certain cases, other modes of thought
are convenient. In a mountainous country
distances are often reckoned in hours. But
leaving distance, the other terms, Force and
Mass, are much more obscure. The exact
comprehension of the ideas which Newton
\index{Newton}%
meant to convey by these words was of slow
growth, and, indeed, Newton himself was the
first man who had thoroughly mastered the
true general principles of Dynamics.
\index{Dynamics}%

Throughout the middle ages, under the influence
of Aristotle, the science was entirely
\index{Aristotle}%
misconceived. Newton had the advantage of
coming after a series of great men, notably
Galileo, in Italy, who in the previous two
\index{Galileo}%
centuries had reconstructed the science and
had invented the right way of thinking about
it. He completed their work. Then, finally,
having the ideas of force, mass, and distance,
\PageSep{31}
clear and distinct in his mind, and realising
their importance and their relevance to the
fall of an apple and the motions of the planets,
he hit upon the law of gravitation and proved
it to be the formula always satisfied in these
various motions.

The vital point in the application of mathematical
formulæ is to have clear ideas and a
correct estimate of their relevance to the
phenomena under observation. No less than
ourselves, our remote ancestors were impressed
with the importance of natural
phenomena and with the desirability of taking
energetic measures to regulate the sequence
of events. Under the influence of irrelevant
ideas they executed elaborate religious ceremonies
to aid the birth of the new moon, and
performed sacrifices to save the sun during
the crisis of an eclipse. There is no reason to
believe that they were more stupid than we
are. But at that epoch there had not been
opportunity for the slow accumulation of
clear and relevant ideas.

The sort of way in which physical sciences
\index{Electromagnetism|EtSeq}%
grow into a form capable of treatment by
mathematical methods is illustrated by the
history of the gradual growth of the science
of electromagnetism. Thunderstorms are
events on a grand scale, arousing terror in
men and even animals. From the earliest
times they must have been objects of wild
\PageSep{32}
\index{Electricity|EtSeq}%
and fantastic hypotheses, though it may be
doubted whether our modern scientific discoveries
in connection with electricity are not
more astonishing than any of the magical
explanations of savages. The Greeks knew
that amber (Greek, electron) when rubbed
would attract light and dry bodies. In
1600~\AD, Dr.~Gilbert, of Colchester, published
\index{Gilbert, Dr.}%
the first work on the subject in which any
scientific method is followed. He made a
list of substances possessing properties similar
to those of amber; he must also have the
credit of connecting, however vaguely, electric
and magnetic phenomena. At the end of the
seventeenth and throughout the eighteenth
century knowledge advanced. Electrical
machines were made, sparks were obtained
from them; and the Leyden Jar was invented,
by which these effects could be intensified.
Some organised knowledge was
being obtained; but still no relevant mathematical
ideas had been found out. Franklin,
\index{Franklin}%
in the year 1752, sent a kite into the clouds
and proved that thunderstorms were electrical.

Meanwhile from the earliest epoch (2634~\BC)
the Chinese had utilized the characteristic
property of the compass needle, but do not
seem to have connected it with any theoretical
ideas. The really profound changes in human
life all have their ultimate origin in knowledge
\PageSep{33}
pursued for its own sake. The use of the compass
was not introduced into Europe till the end
of the twelfth century~\AD, more than $3000$~years
after its first use in China. The importance
which the science of electromagnetism
has since assumed in every department of
human life is not due to the superior practical
bias of Europeans, but to the fact that in the
West electrical and magnetic phenomena
were studied by men who were dominated by
abstract theoretic interests.

The discovery of the electric current is due
\index{Electric Current}%
to two Italians, Galvani in~1780, and Volta
\index{Galvani}%
\index{Volta}%
in~1792. This great invention opened a new
series of phenomena for investigation. The
scientific world had now three separate,
though allied, groups of occurrences on hand---the
effects of ``statical'' electricity arising
from frictional electrical machines, the magnetic
phenomena, and the effects due to
electric currents. From the end of the
eighteenth century onwards, these three lines
of investigation were quickly \Chg{inter-connected}{interconnected}
and the modern science of electromagnetism
was constructed, which now threatens to
transform human life.

Mathematical ideas now appear. During
the decade 1780 to~1789, Coulomb, a Frenchman,
\index{Coulomb}%
proved that magnetic poles attract or
repel each other, in proportion to the inverse
square of their distances, and also that the
\PageSep{34}
same law holds for electric charges---laws
curiously analogous to that of gravitation.
In~1820, Öersted, a Dane, discovered that
\index{Oersted@Öersted}%
electric currents exert a force on magnets,
and almost immediately afterwards the
mathematical law of the force was correctly
formulated by Ampère, a Frenchman, who
\index{Ampere@Ampère}%
also proved that two electric currents exerted
forces on each other. ``The experimental investigation
by which Ampère established the
law of the mechanical action between electric
currents is one of the most brilliant achievements
in science. The whole, theory and
experiment, seems as if it had leaped, full-grown
and full armed, from the brain of
the `Newton of Electricity.' It is perfect
\index{Newton}%
in form, and unassailable in accuracy, and it
is summed up in a formula from which all
the phenomena may be deduced, and which
must always remain the cardinal formula of
electro-dynamics.''\footnote
  {\Title{Electricity and Magnetism}, Clerk Maxwell, Vol.~II.,
\index{Clerk Maxwell}%
  ch.~iii.}

The momentous laws of induction between
currents and between currents and magnets
were discovered by Michael Faraday in 1831--82.
\index{Faraday}%
Faraday was asked: ``What is the use
of this discovery?'' He answered: ``What is
the use of a child---it grows to be a man.''
Faraday's child has grown to be a man and
is now the basis of all the modern applications
\PageSep{35}
of electricity. Faraday also reorganized the
whole theoretical conception of the science.
His ideas, which had not been fully understood
by the scientific world, were extended
and put into a directly mathematical form by
Clerk Maxwell in~1873. As a result of his
\index{Clerk Maxwell}%
mathematical investigations, Maxwell recognized
that, under certain conditions, electrical
vibrations ought to be propagated. He at
once suggested that the vibrations which
form light are electrical. This suggestion has
\index{Light}%
since been verified, so that now the whole
theory of light is nothing but a branch of the
% [** TN: Herz [sic]]
great science of electricity. Also Herz, a
\index{Herz}%
German, in~1888, following on Maxwell's
ideas, succeeded in producing electric vibrations
by direct electrical methods\Add{.} His
experiments are the basis of our wireless
telegraphy.

In more recent years even more fundamental
discoveries have been made, and the
science continues to grow in theoretic importance
and in practical interest. This rapid
sketch of its progress illustrates how, by the
gradual introduction of the relevant theoretic
ideas, suggested by experiment and themselves
suggesting fresh experiments, a whole
mass of isolated and even trivial phenomena
are welded together into one coherent science,
in which the results of abstract mathematical
deductions, starting from a few simple assumed
\PageSep{36}
laws, supply the explanation to the
complex tangle of the course of events.

Finally, passing beyond the particular
sciences of electromagnetism and light, we
can generalize our point of view still further,
and direct our attention to the growth of
mathematical physics considered as one great
chapter of scientific thought. In the first
place, what in the barest outlines is the story
of its growth?

It did not begin as one science, or as the
product of one band of men. The Chaldean
shepherds watched the skies, the agents of
Government in Mesopotamia and Egypt
measured the land, priests and philosophers
brooded on the general nature of all things.
The vast mass of the operations of nature
appeared due to mysterious unfathomable
forces. ``The wind bloweth where it listeth''
expresses accurately the blank ignorance then
existing of any stable rules followed in detail
by the succession of phenomena. In broad outline,
then as now, a regularity of events was
patent. But no minute tracing of their interconnection
was possible, and there was no
knowledge how even to set about to construct
such a science.

Detached speculations, a few happy or unhappy
shots at the nature of things, formed
the utmost which could be produced.

Meanwhile land-surveys had produced geometry,
\index{Geometry}%
\PageSep{37}
and the observations of the heavens
disclosed the exact regularity of the solar
system. Some of the later Greeks, such as
Archimedes, had just views on the elementary
\index{Archimedes|EtSeq}%
phenomena of hydrostatics and optics. Indeed,
Archimedes, who combined a genius for
mathematics with a physical insight, must
rank with Newton, who lived nearly two
\index{Newton}%
thousand years later, as one of the founders
of mathematical physics. He lived at Syracuse,
the great Greek city of Sicily. When
the Romans besieged the town (in 212~to
210~\BC), he is said to have burned their ships
by concentrating on them, by means of
mirrors, the sun's rays. The story is highly
improbable, but is good evidence of the reputation
which he had gained among his contemporaries
for his knowledge of optics. At
the end of this siege he was killed. According
to one account given by Plutarch, in his life of
\index{Plutarch}%
Marcellus, he was found by a Roman soldier
\index{Marcellus}%
absorbed in the study of a geometrical diagram
which he had traced on the sandy floor of his
room. He did not immediately obey the orders
of his captor, and so was killed. For the credit
of the Roman generals it must be said that
the soldiers had orders to spare him. The
internal evidence for the other famous story
of him is very strong; for the discovery
attributed to him is one eminently worthy of
his genius for mathematical and physical research.
\PageSep{38}
Luckily, it is simple enough to be
explained here in detail. It is one of the best
easy examples of the method of application
of mathematical ideas to physics.

Hiero, King of Syracuse, had sent a quantity
\index{Hiero}%
of gold to some goldsmith to form the
material of a crown. He suspected that the
craftsmen had abstracted some of the gold
and had supplied its place by alloying the
remainder with some baser metal. Hiero
sent the crown to Archimedes and asked him
to test it. In these days an indefinite number
of chemical tests would be available.
But then Archimedes had to think out the
matter afresh. The solution flashed upon
him as he lay in his bath. He jumped
up and ran through the streets to the
palace, shouting \Foreign{Eureka! Eureka!} (I have
found it, I have found it). This day, if we
knew which it was, ought to be celebrated as
the birthday of mathematical physics; the
science came of age when Newton sat in his
\index{Newton}%
orchard. Archimedes had in truth made a
great discovery. He saw that a body when
immersed in water is pressed upwards by the
surrounding water with a resultant force
equal to the weight of the water it displaces.
This law can be proved theoretically from the
mathematical principles of hydrostatics and
can also be verified experimentally. Hence,
if $W$~lb.\ be the weight of the crown, as weighed
\PageSep{39}
in air, and $w$~lb.\ be the weight of the water
which it displaces when completely immersed,
$W - w$ would be the extra upward force
necessary to sustain the crown as it hung in
water.

Now, this upward force can easily be ascertained
by weighing the body as it hangs in
water, as shown in the annexed figure. If
\Figure{3}
the weights in the right-hand scale come to
$F$~lb., then the apparent weight of the crown
in water is $F$~lb.; and we thus have
\[
F = W - w
\]
and thus
\[
w = W - F,
\]
and
\[
\frac{W}{w} = \frac{W}{W - F}
\Tag{(A)}
\]
where $W$ and $F$ are determined by the easy,
and fairly precise, operation of weighing.
\PageSep{40}
Hence, by equation~\Eq{(A)}, $\dfrac{W}{w}$~is known. But
$\dfrac{W}{w}$~is the ratio of the weight of the crown to
the weight of an equal volume of water.
This ratio is the same for any lump of metal of
the same material: it is now called the specific
gravity of the material, and depends only on
the intrinsic nature of the substance and not
on its shape or quantity. Thus to test if the
crown were of gold, Archimedes had only to
take a lump of indisputably pure gold and
find its specific gravity by the same process.
If the two specific gravities agreed, the crown
was pure; if they disagreed, it was debased.

This argument has been given at length,
because not only is it the first precise example
of the application of mathematical ideas to
physics, but also because it is a perfect and
simple example of what must be the method
and spirit of the science for all time.

The death of Archimedes by the hands of a
Roman soldier is symbolical of a world-change
of the first magnitude: the theoretical Greeks,
with their love of abstract science, were superseded
in the leadership of the European world
by the practical Romans. Lord Beaconsfield,
\index{Beaconsfield, Lord}%
in one of his novels, has defined a practical
man as a man who practises the errors of
his forefathers. The Romans were a great
race, but they were cursed with the sterility
\PageSep{41}
\index{Specific Gravity}%
which waits upon practicality. They did not
improve upon the knowledge of their forefathers,
and all their advances were confined
to the minor technical details of engineering.
They were not dreamers enough to arrive at
new points of view, which could give a more
fundamental control over the forces of nature.
No Roman lost his life because he was absorbed
in the contemplation of a mathematical
diagram.
\PageSep{42}


\Chapter{IV}{Dynamics}

\First{The} world had to wait for eighteen hundred
years till the Greek mathematical physicists
found successors. In the sixteenth and seventeenth
centuries of our era great Italians, in
particular Leonardo da~Vinci, the artist
\index{Aristotle}%
\index{Galileo|EtSeq}%
\index{Leonardo da Vinci}%
(born 1452, died 1519), and Galileo (born 1564,
died 1642), rediscovered the secret, known to
Archimedes, of relating abstract mathematical
ideas with the experimental investigation of
natural phenomena. Meanwhile the slow
advance of mathematics and the accumulation
of accurate astronomical knowledge had
placed natural philosophers in a much more
advantageous position for research. Also the
very egoistic self-assertion of that age, its
greediness for personal experience, led its
thinkers to want to see for themselves what
happened; and the secret of the relation of
mathematical theory and experiment in inductive
reasoning was practically discovered.
It was an act eminently characteristic of the
age that Galileo, a philosopher, should have
\PageSep{43}
dropped the weights from the leaning tower
of Pisa. There are always men of thought
and men of action; mathematical physics is
the product of an age which combined in the
same men impulses to thought with impulses
to action.

This matter of the dropping of weights from
\index{Dynamics|EtSeq}%
the tower marks picturesquely an essential
step in knowledge, no less a step than the
first attainment of correct ideas on the science
of dynamics, the basal science of the whole
subject. The particular point in dispute was
as to whether bodies of different weights
would fall from the same height in the same
time. According to a dictum of Aristotle,
universally followed up to that epoch, the
heavier weight would fall the quicker. Galileo
affirmed that they would fall in the same
time, and proved his point by dropping
weights from the top of the leaning tower.
The apparent exceptions to the rule all arise
when, for some reason, such as extreme lightness
or great speed, the air resistance is important.
But neglecting the air the law is
exact.

Galileo's successful experiment was not the
\index{Motion, First Law of}%
result of a mere lucky guess. It arose from
his correct ideas in connection with inertia
and mass. The first law of motion, as following
Newton we now enunciate it, is---Every
\index{Newton}%
body continues in its state of rest or of uniform
\PageSep{44}
motion in a straight line, except so far
as it is compelled by impressed force to
change that state. This law is more than a
dry formula: it is also a pæan of triumph
over defeated heretics. The point at issue
can be understood by deleting from the law
the phrase ``or of uniform motion in a straight
line.'' We there obtain what might be taken
as the Aristotelian opposition formula:
``Every body continues in its state of rest
except so far as it is compelled by impressed
force to change that state.''

In this last false formula it is asserted that,
apart from force, a body continues in a state
of rest; and accordingly that, if a body is
moving, a force is required to sustain the
motion; so that when the force ceases, the
motion ceases. The true Newtonian law
takes diametrically the opposite point of view.
The state of a body unacted on by force is
that of uniform motion in a straight line, and
no external force or influence is to be looked
for as the cause, or, if you like to put it so, as
the invariable accompaniment of this uniform
rectilinear motion. Rest is merely a particular
case of such motion, merely when the
velocity is and remains zero. Thus, when a
body is moving, we do not seek for any external
influence except to explain changes in
the rate of the velocity or changes in its direction.
So long as the body is moving at the
\PageSep{45}
same rate and in the same direction there is
no need to invoke the aid of any forces.

The difference between the two points of
view is well seen by reference to the theory of
the motion of the planets. Copernicus, a
\index{Copernicus}%
Pole, born at Thorn in West Prussia (born
1473, died 1543), showed how much simpler
it was to conceive the planets, including the
\Figure[2.25in]{4}
earth as revolving round the sun in orbits
which are nearly circular; and later, Kepler,
\index{Kepler}%
a German mathematician, in the year 1609
proved that, in fact, the orbits are practically
ellipses, that is, a special sort of oval curves
\index{Ellipse}%
which we will consider later in more detail.
Immediately the question arose as to what
are the forces which preserve the planets in
this motion. According to the old false view,
\PageSep{46}
held by Kepler, the actual velocity itself required
\index{Kepler}%
preservation by force. Thus he looked
for tangential forces as in the accompanying
figure~(\FigNum{4}). But according to the Newtonian
law, apart from some force the planet would
move for ever with its existing velocity in a
straight line, and thus depart entirely from
the sun. Newton, therefore, had to search
\index{Newton}%
for a force which would bend the motion
\Figure[2.25in]{5}
round into its elliptical orbit. This he showed
must be a force directed towards the sun as in
the next figure~(\FigNum{5}). In fact, the force is the
gravitational attraction of the sun acting
according to the law of the inverse square of
the distance, which has been stated above.

The science of mechanics rose among the
\index{Mechanics}%
Greeks from a consideration of the theory of
the mechanical advantage obtained by the use
\PageSep{47}
\index{Dynamical Explanation|EtSeq}%
of a lever, and also from a consideration of
various problems connected with the weights
of bodies. It was finally put on its true basis
at the end of the sixteenth and during the
seventeenth centuries, as the preceding account
shows, partly with the view of explaining
the theory of falling bodies, but chiefly
in order to give a scientific theory of planetary
motions. But since those days dynamics has
taken upon itself a more ambitious task, and
now claims to be the ultimate science of which
the others are but branches. The claim
amounts to this: namely, that the various
qualities of things perceptible to the senses
are merely our peculiar mode of appreciating
changes in position on the part of things
existing in space. For example, suppose we
look at Westminster Abbey. It has been
standing there, grey and immovable, for centuries
past. But, according to modern scientific
theory, that greyness, which so heightens
our sense of the immobility of the building, is
itself nothing but our way of appreciating the
rapid motions of the ultimate molecules, which
form the outer surface of the building and
communicate vibrations to a substance called
the ether. Again we lay our hands on its
stones and note their cool, even temperature,
so symbolic of the quiet repose of the building.
But this feeling of temperature simply marks
our sense of the transfer of heat from the
\PageSep{48}
hand to the stone, or from the stone to the
hand; and, according to modern science,
heat is nothing but the agitation of the molecules
of a body. Finally, the organ begins
playing, and again sound is nothing but the
result of motions of the air striking on the
drum of the ear.

Thus the endeavour to give a dynamical
explanation of phenomena is the attempt to
explain them by statements of the general
form, that such and such a substance or body
was in this place and is now in that place.
Thus we arrive at the great basal idea of
modern science, that all our sensations are
the result of comparisons of the changed
configurations of things in space at various
times. It follows therefore, that the laws
of motion, that is, the laws of the changes
of configurations of things, are the ultimate
laws of physical science.

In the application of mathematics to the
investigation of natural philosophy, science
does systematically what ordinary thought
does casually. When we talk of a chair, we
usually mean something which we have been
seeing or feeling in some way; though most
of our language will presuppose that there
is something which exists independently of
our sight or feeling. Now in mathematical
physics the opposite course is taken. The
chair is conceived without any reference to
\PageSep{49}
\index{Variable, The}%
anyone in particular, or to any special modes
of perception. The result is that the chair
becomes in thought a set of molecules in space,
or a group of electrons, a portion of the ether
in motion, or however the current scientific
ideas describe it. But the point is that
science reduces the chair to things moving in
space and influencing each other's motions.
Then the various elements or factors which
enter into a set of circumstances, as thus
conceived, are merely the things, like lengths
of lines, sizes of angles, areas, and volumes, by
which the positions of bodies in space can be
settled. Of course, in addition to these geometrical
elements the fact of motion and
change necessitates the introduction of the
rates of changes of such elements, that is to
say, velocities, angular velocities, accelerations,
and suchlike things. Accordingly, mathematical
physics deals with correlations between
variable numbers which are supposed
to represent the correlations which exist in
nature between the measures of these geometrical
elements and of their rates of change.
But always the mathematical laws deal with
variables, and it is only in the occasional
testing of the laws by reference to experiments,
or in the use of the laws for special
predictions that definite numbers are substituted.

The interesting point about the world as
\PageSep{50}
thus conceived in this abstract way throughout
the study of mathematical physics, where
only the positions and shapes of things are
considered together with their changes, is that
the events of such an abstract world are sufficient
to ``explain'' our sensations. When we
hear a sound, the molecules of the air have
been agitated in a certain way: given the
agitation, or air-waves as they are called, all
normal people hear sound; and if there are
no air-waves, there is no sound. And, similarly,
a physical cause or origin, or parallel
event (according as different people might like
to phrase it) underlies our other sensations.
Our very thoughts appear to correspond to
conformations and motions of the brain; injure
the brain and you injure the thoughts.
Meanwhile the events of this physical universe
succeed each other according to the mathematical
laws which ignore all special sensations
and thoughts and emotions.

Now, undoubtedly, this is the general aspect
of the relation of the world of mathematical
physics to our emotions, sensations, and
thoughts; and a great deal of controversy
has been occasioned by it and much ink
spilled. We need only make one remark. The
whole situation has arisen, as we have seen,
from the endeavour to describe an external
world ``explanatory'' of our various individual
sensations and emotions, but a world
\PageSep{51}
also, not essentially dependent upon any
particular sensations or upon any particular
individual. Is such a world merely but
one huge fairy tale? But fairy tales are
fantastic and arbitrary: if in truth there
be such a world, it ought to submit itself
to an exact description, which determines
accurately its various parts and their mutual
relations. Now, to a large degree, this
scientific world does submit itself to this
test and allow its events to be explored
and predicted by the apparatus of abstract
mathematical ideas. It certainly seems that
here we have an inductive verification of
our initial assumption. It must be admitted
that no inductive proof is conclusive; but
if the whole idea of a world which has
existence independently of our particular perceptions
of it be erroneous, it requires careful
explanation why the attempt to characterise
it, in terms of that mathematical remnant
of our ideas which would apply to it, should
issue in such a remarkable success.

It would take us too far afield to enter into
\index{Parallelogram Law|EtSeq}%
\index{Vectors|EtSeq}%
a detailed explanation of the other laws of
motion. The remainder of this chapter must
be devoted to the explanation of remarkable
ideas which are fundamental, both to mathematical
physics and to pure mathematics:
these are the ideas of vector quantities and
the parallelogram law for vector addition. We
\PageSep{52}
have seen that the essence of motion is that
a body was at~$A$ and is now at~$C$. This transference
from $A$ to~$C$ requires two distinct
elements to be settled before it is completely
determined, namely its magnitude (\ie\ the
length~$AC$) and its direction. Now anything,
like this transference, which is completely
given by the determination of a magnitude
\Figure[2in]{6}
and a direction is called a vector. For
example, a velocity requires for its definition
the assignment of a magnitude and of a
direction. It must be of so many miles per
hour in such and such a direction. The existence
and the independence of these two
elements in the determination of a velocity
are well illustrated by the action of the captain
of a ship, who communicates with different subordinates
respecting them: he tells the chief
engineer the number of knots at which he is
to steam, and the helmsman the compass
\PageSep{53}
bearing of the course which he is to keep.
Again the rate of change of velocity, that is
velocity added per unit time, is also a vector
quantity: it is called the acceleration. Similarly
a force in the dynamical sense is another
vector quantity. Indeed, the vector nature
of forces follows at once according to dynamical
principles from that of velocities and
accelerations; but this is a point which we
need not go into. It is sufficient here to say
that a force acts on a body with a certain
magnitude in a certain direction.

Now all vectors can be graphically represented
by straight lines. All that has to be
done is to arrange: (i)~a scale according to
which units of length correspond to units of
magnitude of the vector---for example, one
inch to a velocity of $10$~miles per~hour in the
case of velocities, and one inch to a force of
$10$~tons weight in the case of forces---and (ii)~a
direction of the line on the diagram corresponding
to the direction of the vector. Then
a line drawn with the proper number of inches
of length in the proper direction represents the
required vector on the arbitrarily assigned scale
of magnitude. This diagrammatic representation
of vectors is of the first importance. By
its aid we can enunciate the famous ``parallelogram
law'' for the addition of vectors of the
same kind but in different directions.

Consider the vector~$AC$ in \Fig[figure]{6} as representative
\PageSep{54}
\index{Transportation, Vector of|EtSeq}%
of the changed position of a body
from $A$ to~$C$: we will call this the vector of
transportation. It will be noted that, if the
reduction of physical phenomena to mere
changes in positions, as explained above, is
correct, all other types of physical vectors are
really reducible in some way or other to this
single type. Now the final transportation
from $A$ to~$C$ is equally well effected by a
transportation from $A$ to~$B$ and a transportation
from $B$ to~$C$, or, completing the parallelogram
$ABCD$, by a transportation from $A$ to~$D$
and a transportation from $D$ to~$C$. These
transportations as thus successively applied
are said to be added together. This is simply
a definition of what we mean by the addition
of transportations. Note further that, considering
parallel lines as being lines drawn in
the same direction, the transportations $B$~to~$C$
and $A$~to~$D$ may be conceived as the same
transportation applied to bodies in the two
initial positions $B$ and~$A$. With this conception
we may talk of the transportation
$A$~to~$D$ as applied to a body in any position,
for example at~$B$. Thus we may say that
the transportation $A$~to~$C$ can be conceived
as the sum of the two transportations $A$~to~$B$
and $A$~to~$D$ applied in any order. Here
we have the parallelogram law for the addition
of transportations: namely, if the
transportations are $A$~to~$B$ and $A$~to~$D$,
\PageSep{55}
complete the parallelogram $ABCD$, and then
the sum of the two is the diagonal~$AC$.

All this at first sight may seem to be
very artificial. But it must be observed
that nature itself presents us with the idea.
For example, a steamer is moving in the
direction~$AD$ (\Chg{cf.}{\cf}\ \Fig[fig.]{6}) and a man walks
across its deck. If the steamer were still,
in one minute he would arrive at~$B$; but
during that minute his starting point~$A$ on
the deck has moved to~$D$, and his path on
the deck has moved from $AB$ to~$DC$. So
that, in fact, his transportation has been from
$A$ to~$C$ over the surface of the sea. It is,
however, presented to us analysed into the
sum of two transportations, namely, one from
$A$ to~$B$ relatively to the steamer, and one
from $A$ to~$D$ which is the transportation of
the steamer.

By taking into account the element of time,
namely one minute, this diagram of the man's
transportation~$AC$ represents his velocity.
For if $AC$~represented so many feet of transportation,
it now represents a transportation
of so many feet per minute, that is to say, it
represents the velocity of the man. Then
$AB$ and $AD$ represent two velocities, namely,
his velocity relatively to the steamer, and the
velocity of the steamer, whose ``sum'' makes
up his complete velocity. It is evident that
diagrams and definitions concerning transportations
\PageSep{56}
are turned into diagrams and definitions
concerning velocities by conceiving
the diagrams as representing transportations
per unit time. Again, diagrams and definitions
concerning velocities are turned into
diagrams and definitions concerning accelerations
\Figure[3in]{7}
by conceiving the diagrams as representing
velocities added per unit time.

Thus by the addition of vector velocities
and of vector accelerations, we mean the
addition according to the parallelogram law.

Also, according to the laws of motion a
force is fully represented by the vector
acceleration it produces in a body of given
mass. Accordingly, forces will be said to be
added when their joint effect is to be reckoned
according to the parallelogram law.

Hence for the fundamental vectors of
\PageSep{57}
science, namely transportations, velocities,
and forces, the addition of any two of the same
kind is the production of a ``resultant''
vector according to the rule of the parallelogram
law.

By far the simplest type of parallelogram
is a rectangle, and in pure mathematics it is
\index{Rectangle}%
the relation of the single vector~$AC$ to the
two component vectors, $AB$~and~$AD$, at right
angles (\Chg{cf.}{\cf}\ \Fig[fig.]{7}), which is continually recurring.
Let $x$,~$y$, and $r$~units represent the
lengths of $AB$,~$AD$, and~$AC$, and let $m$~units
of angle represent the magnitude of the angle
$BAC$. Then the relations between $x$,~$y$,~$r$,
and~$m$, in all their many aspects are the continually
recurring topic of pure mathematics;
and the results are of the type required for
application to the fundamental vectors of
mathematical physics. This diagram is the
chief bridge over which the results of pure
mathematics pass in order to obtain application
to the facts of nature.
\PageSep{58}


\Chapter{V}{The Symbolism of Mathematics}

\First{We} now return to pure mathematics, and
consider more closely the apparatus of ideas
out of which the science is built. Our first
concern is with the symbolism of the science,
and we start with the simplest and universally
known symbols, namely those of arithmetic.

Let us assume for the present that we have
\index{Arabic Notation|EtSeq}%
sufficiently clear ideas about the integral
numbers, represented in the Arabic notation
by $0$,~$1$, $2$,~\dots, $9$, $10$, $11$,~\dots\Add{,} $100$, $101$,~\dots\ and
so on. This notation was introduced into
Europe through the Arabs, but they apparently
obtained it from Hindoo sources. The
first known work\footnote
  {For the detailed historical facts relating to pure
  mathematics, I am chiefly indebted to \Title{A Short History
\index{Ball, W. W. R.}%
  of Mathematics}, by W.~W.~R. Ball.}
% [** TN: http://www.gutenberg.org/ebooks/31246]
in which it is systematically
explained is a work by an Indian mathematician,
Bhaskara (born 1114~\AD). But
\index{Bhaskara}%
the actual numerals can be traced back to the
seventh century of our era, and perhaps were
originally invented in Tibet. For our present
\PageSep{59}
purposes, however, the history of the notation
is a detail. The interesting point to notice
is the admirable illustration which this
numeral system affords of the enormous importance
of a good notation. By relieving
the brain of all unnecessary work, a good
notation sets it free to concentrate on more
advanced problems, and in effect increases
the mental power of the race. Before the
introduction of the Arabic notation, multiplication
was difficult, and the division even of
integers called into play the highest mathematical
faculties. Probably nothing in the
modern world would have more astonished a
Greek mathematician than to learn that, under
the influence of compulsory education, a
large proportion of the population of Western
Europe could perform the operation of
division for the largest numbers. This fact
would have seemed to him a sheer impossibility.
The consequential extension of
the notation to decimal fractions was not
accomplished till the seventeenth century.
Our modern power of easy reckoning with
decimal fractions is the almost miraculous
result of the gradual discovery of a perfect
notation.

Mathematics is often considered a difficult
and mysterious science, because of the
numerous symbols which it employs. Of
course, nothing is more incomprehensible than
\PageSep{60}
a symbolism which we do not understand.
Also a symbolism, which we only partially
understand and are unaccustomed to use, is
difficult to follow. In exactly the same way
the technical terms of any profession or trade
are incomprehensible to those who have never
been trained to use them. But this is not
because they are difficult in themselves. On
the contrary they have invariably been introduced
to make things easy. So in mathematics,
granted that we are giving any serious
attention to mathematical ideas, the symbolism
is invariably an immense simplification.
It is not only of practical use, but is
of great interest. For it represents an analysis
of the ideas of the subject and an almost
pictorial representation of their relations to
each other. If anyone doubts the utility of
symbols, let him write out in full, without any
symbol whatever, the whole meaning of the
following equations which represent some of
\index{Algebra, Fundamental Laws of}%
the fundamental laws of algebra\footnotemark:---
\footnotetext{\Chg{Cf.}{\Cf}\ Note~A, \Pageref{noteA}.\Pagelabel{60}}
%[** TN: left-aligned in the original]
\begin{gather*}
x + y = y + x\Add{,}
\Tag{(1)} \\
(x + y) + z = x + (y + z)\Add{,}
\Tag{(2)} \\
x × y = y × x\Add{,}
\Tag{(3)} \\
(x × y) × z = x × (y × z)\Add{,}
\Tag{(4)} \\
x × (y + z) = (x × y) + (x × z)\Add{.}
\Tag{(5)}
\end{gather*}

Here \Eq{(1)}~and \Eq{(2)} are called the commutative
and associative laws for addition, \Eq{(3)}~and \Eq{(4)}
\PageSep{61}
are the commutative and associative laws for
multiplication, and \Eq{(5)}~is the distributive law
relating addition and multiplication. For example,
without symbols, \Eq{(1)}~becomes: If a
second number be added to any given number
the result is the same as if the first given
number had been added to the second number.

This example shows that, by the aid of symbolism,
we can make transitions in reasoning
almost mechanically by the eye, which otherwise
would call into play the higher faculties
of the brain.

It is a profoundly erroneous truism, repeated
by all copy-books and by eminent people when
they are making speeches, that we should
cultivate the habit of thinking of what we are
doing. The precise opposite is the case.
Civilization advances by extending the number
of important operations which we can
perform without thinking about them. Operations
of thought are like cavalry charges in
a battle---they are strictly limited in number,
they require fresh horses, and must only
be made at decisive moments.

One very important property for symbolism
to possess is that it should be concise, so as to
be visible at one glance of the eye and to be
rapidly written. Now we cannot place symbols
more concisely together than by placing
them in immediate juxtaposition. In a good
symbolism therefore, the juxtaposition of important
\PageSep{62}
symbols should have an important
meaning. This is one of the merits of the
Arabic notation for numbers; by means of
ten symbols, $0$,~$1$, $2$, $3$, $4$, $5$, $6$, $7$, $8$,~$9$, and by
simple juxtaposition it symbolizes any number
whatever. Again in algebra, when we have
two variable numbers $x$ and~$y$, we have to
make a choice as to what shall be denoted by
their juxtaposition~$xy$. Now the two most
important ideas on hand are those of addition
and multiplication. Mathematicians have
chosen to make their symbolism more concise
by defining $xy$ to stand for $x × y$. Thus the
laws \Eq{(3)},~\Eq{(4)}, and~\Eq{(5)} above are in general
written,
\[
xy = yx,\quad
(xy)z = x(yz),\quad
x(y + z) = xy + xz,
\]
thus securing a great gain in conciseness.
The same rule of symbolism is applied to the
juxtaposition of a definite number and a variable:
we write~$3x$ for $3 × x$, and $30x$ for $30 × x$.

It is evident that in substituting definite
numbers for the variables some care must be
taken to restore the~$×$, so as not to conflict
with the Arabic notation. Thus when we
substitute $2$~for~$x$ and $3$~for~$y$ in~$xy$, we must
write $2 × 3$ for~$xy$, and not~$23$ which means
$20 + 3$.

It is interesting to note how important for
the development of science a modest-looking
symbol may be. It may stand for the emphatic
presentation of an idea, often a very
\PageSep{63}
subtle idea, and by its existence make it easy
to exhibit the relation of this idea to all the
complex trains of ideas in which it occurs.
For example, take the most modest of all
symbols, namely,~$0$, which stands for the \emph{number}
\index{Zero|EtSeq}%
zero. The Roman notation for numbers
had no symbol for zero, and probably most
mathematicians of the ancient world would
have been horribly puzzled by the idea of the
number zero. For, after all, it is a very
subtle idea, not at all obvious. A great deal
of discussion on the meaning of the zero of
quantity will be found in philosophic works.
Zero is not, in real truth, more difficult or
subtle in idea than the other cardinal numbers.
What do we mean by~$1$ or by~$2$, or by~$3$?
But we are familiar with the use of these ideas,
though we should most of us be puzzled to
give a clear analysis of the simpler ideas
which go to form them. The point about zero
is that we do not need to use it in the operations
of daily life. No one goes out to buy
zero fish. It is in a way the most civilized
of all the cardinals, and its use is only forced
on us by the needs of cultivated modes of
thought. Many important services are rendered
by the symbol~$0$, which stands for the
number zero.

The symbol developed in connection with
the Arabic notation for numbers of which it
is an essential part. For in that notation the
\PageSep{64}
value of a digit depends on the position in
which it occurs. Consider, for example, the
digit~$5$, as occurring in the numbers $25$, $51$,
$3512$, $5213$. In the first number~$5$ stands for
five, in the second number $5$~stands for fifty,
in the third number for five hundred, and in
the fourth number for five thousand. Now,
when we write the number fifty-one in the
symbolic form~$51$, the digit~$1$ pushes the digit~$5$
along to the second place (reckoning from
right to left) and thus gives it the value fifty.
But when we want to symbolize fifty by itself,
we can have no digit~$1$ to perform this service;
we want a digit in the units place to add
nothing to the total and yet to push the~$5$
along to the second place. This service is
performed by~$0$, the symbol for zero. It is
extremely probable that the men who introduced
for this purpose had no definite conception
in their minds of the number zero.
They simply wanted a mark to symbolize the
fact that nothing was contributed by the
digit's place in which it occurs. The idea of
zero probably took shape gradually from a
desire to assimilate the meaning of this mark
to that of the marks, $1$, $2$,~\dots\Add{,}~$9$, which do represent
cardinal numbers. This would not
represent the only case in which a subtle idea
has been introduced into mathematics by a
symbolism which in its origin was dictated by
practical convenience.
\PageSep{65}

Thus the first use of~$0$ was to make the
arable notation possible---no slight service.
We can imagine that when it had been introduced
for this purpose, practical men, of the
sort who dislike fanciful ideas, deprecated the
silly habit of identifying it with a number
zero. But they were wrong, as such men
always are when they desert their proper
function of masticating food which others have
prepared. For the next service performed by
the symbol~$0$ essentially depends upon assigning
to it the function of representing the
number zero.

This second symbolic use is at first sight
so absurdly simple that it is difficult to make
a beginner realize its importance. Let us
start with a simple example. In \ChapRef{II}.\
we mentioned the correlation between two
variable numbers $x$ and $y$ represented by the
equation $x + y = 1$. This can be represented
in an indefinite number of ways; for example,
$x = 1 - y$, $y = 1 - x$, $2x + 3y - 1 = x + 2y$, and so
on. But the important way of stating it is
\[
x + y - 1 = 0.
\]
Similarly the important way of writing the
equation $x = 1$ is $x - 1 = 0$, and of representing
the equation $3x - 2 = 2x^{2}$ is $2x^{2} - 3x + 2 = 0$.
The point is that all the symbols which represent
variables, \eg\ $x$~and~$y$, and the symbols
\PageSep{66}
representing some definite number other than
zero, such as $1$ or $2$ in the examples above,
are written on the left-hand side, so that the
whole left-hand side is equated to the number
zero. The first man to do this is said to
have been Thomas Harriot, born at Oxford
\index{Harriot, Thomas}%
in 1560 and died in~1621. But what is the
importance of this simple symbolic procedure?
It made possible the growth of the
\index{Form, Algebraic|EtSeq}%
modern conception of \emph{algebraic form}.

This is an idea to which we shall have continually
to recur; it is not going too far to
say that no part of modern mathematics can
be properly understood without constant recurrence
to it. The conception of form is
so general that it is difficult to characterize
it in abstract terms. At this stage we shall
do better merely to consider examples. Thus
the equations $2x - 3 = 0$, $x - 1 = 0$, $5x - 6 = 0$,
are all equations of the same form, namely,
equations involving one unknown~$x$, which is
not multiplied by itself, so that $x^{2}$, $x^{3}$,~etc., do
not appear. Again $3x^{2} - 2x + 1 = 0$, $x^{2} - 3x + 2 = 0$,
$x^{2} - 4 = 0$, are all equations of the same
form, namely, equations involving one unknown~$x$
in which $x × x$, that is~$x^{2}$, appears. These
equations are called quadratic equations.
Similarly cubic equations, in which $x^{3}$~appears,
yield another form, and so on. Among the
three quadratic equations given above there
is a minor difference between the last equation,
\PageSep{67}
$x^{2} - 4 = 0$, and the preceding two equations,
due to the fact that~$x$ (as distinct
from~$x^{2}$) does not appear in the last and
does in the other two. This distinction is
very unimportant in comparison with the
great fact that they are all three quadratic
equations.

Then further there are the forms of equation
stating correlations between two variables;
for example, $x + y - 1 = 0$, $2x + 3y - 8 = 0$, and
so on. These are examples of what is called
the \emph{linear} form of equation. The reason for
this name of ``linear'' is that the graphic
method of representation, which is explained
at the end of \ChapRef{II}\Add{.}, always represents
such equations by a straight line. Then there
are other forms for two variables---for example,
the quadratic form, the cubic form, and so on.
But the point which we here insist upon is
that this study of form is facilitated, and,
indeed, made possible, by the standard method
of writing equations with the symbol~$0$ on
the right-hand side.

There is yet another function performed by~$0$
in relation to the study of form. Whatever
number $x$ may be, $0 × x = 0$, and $x + 0 = x$.
By means of these properties minor differences
of form can be assimilated. Thus the
difference mentioned above between the quadratic
equations $x^{2} - 3x + 2 = 0$, and $x^{2} - 4 = 0$,
can be obliterated by writing the latter
\PageSep{68}
equation in the form $x^{2} + (0 × x) - 4 = 0$. For,
by the laws stated above, $x^{2} + (0 × x) - 4 =
x^{2} + 0 - 4 = x^{2} - 4$. Hence the equation $x^{2} - 4 = 0$\Typo{,}{}
is merely representative of a particular
class of quadratic equations and belongs to
the same general form as does $x^{2} - 3x + 2 = 0$.

For these three reasons the symbol~$0$, representing
the number zero, is essential to
modern mathematics. It has rendered possible
types of investigation which would have
been impossible without it.

The symbolism of mathematics is in truth
the outcome of the general ideas which
dominate the science. We have now two
such general ideas before us, that of the variable
and that of algebraic form. The junction
of these concepts has imposed on mathematics
another type of symbolism almost quaint in
its character, but none the less effective. We
have seen that an equation involving two
variables, $x$~and~$y$, represents a particular
correlation between the pair of variables.
Thus $x + y - 1 = 0$ represents one definite correlation,
and $3x + 2y - 5 = 0$ represents another
definite correlation between the variables $x$
and~$y$; and both correlations have the form
of what we have called linear correlations.
But now, how can we represent \emph{any} linear
correlation between the variable numbers $x$
and~$y$? Here we want to symbolize \emph{any}
linear correlation; just as $x$~symbolizes \emph{any}
\PageSep{69}
number. This is done by turning the numbers
which occur in the definite correlation $3x + 2y - 5 = 0$
into letters. We obtain $ax + by -  c = 0$.
Here $a$,~$b$,~$c$, stand for variable numbers just
as do $x$ and~$y$: but there is a difference in the
use of the two sets of variables. We study
the general properties of the relationship between
$x$ and $y$ while $a$,~$b$, and~$c$ have unchanged
values. We do not determine what
the values of $a$,~$b$, and~$c$ are; but whatever
they are, they remain fixed while we study
the relation between the variables $x$ and $y$
for the whole group of possible values of $x$
and~$y$. But when we have obtained the properties
of this correlation, we note that, because
$a$,~$b$, and~$c$ have not in fact been determined,
we have proved properties which must
belong to \emph{any} such relation. Thus, by now
varying $a$,~$b$, and~$c$, we arrive at the idea that
$ax + by - c = 0$ represents a variable linear
correlation between $x$ and~$y$. In comparison
with $x$ and~$y$, the three variables $a$,~$b$, and~$c$
are called constants. Variables used in this
\index{Constants}%
way are sometimes also called parameters.
\index{Parameters}%

Now, mathematicians habitually save the
trouble of explaining which of their variables
are to be treated as ``constants,'' and which
as variables, considered as correlated in their
equations, by using letters at the end of the
alphabet for the ``variable'' variables, and
letters at the beginning of the alphabet for
\PageSep{70}
the ``constant'' variables, or parameters.
The two systems meet naturally about the
middle of the alphabet. Sometimes a word
or two of explanation is necessary; but as a
matter of fact custom and common sense are
usually sufficient, and surprisingly little confusion
is caused by a procedure which seems
so lax.

The result of this continual elimination of
definite numbers by successive layers of parameters
is that the amount of arithmetic performed
by mathematicians is extremely small.
Many mathematicians dislike all numerical
computation and are not particularly expert
at it. The territory of arithmetic ends where
the two ideas of ``variables'' and of ``algebraic
form'' commence their sway.
\PageSep{71}


\Chapter{VI}{Generalizations of Number}

\First{One} great peculiarity of mathematics is the
\index{Fractions|EtSeq}%
set of allied ideas which have been invented
in connection with the integral numbers from
which we started. These ideas may be called
extensions or generalizations of number. In
the first place there is the idea of fractions.
The earliest treatise on arithmetic which we
possess was written by an Egyptian priest,
named Ahmes, between 1700~\BC\ and 1100~\BC,
\index{Ahmes}%
and it is probably a copy of a much older
work. It deals largely with the properties of
fractions. It appears, therefore, that this
concept was developed very early in the history
of mathematics. Indeed the subject is
a very obvious one. To divide a field into
three equal parts, and to take two of the
parts, must be a type of operation which had
often occurred. Accordingly, we need not be
surprised that the men of remote civilizations
were familiar with the idea of two-thirds, and
\PageSep{72}
with allied notions. Thus as the first generalization
of number we place the concept of
fractions. The Greeks thought of this subject
rather in the form of ratio, so that a
Greek would naturally say that a line of
two feet in length bears to a line of three
feet in length the ratio of $2$~to~$3$. Under
the influence of our algebraic notation we
would more often say that one line was
two-thirds of the other in length, and would
think of two-thirds as a numerical multiplier.

In connection with the theory of ratio, or
\index{Incommensurable Ratios|EtSeq}%
\index{Ratio|EtSeq}%
fractions, the Greeks made a great discovery,
which has been the occasion of a large amount
of philosophical as well as mathematical
thought. They found out the existence of
``incommensurable'' ratios. They proved,
in fact, during the course of their geometrical
investigations that, starting with a line of any
length, other lines must exist whose lengths
do not bear to the original length the ratio
of any pair of integers---or, in other words,
that lengths exist which are not any exact
fraction of the original length.

For example, the diagonal of a square cannot
be expressed as any fraction of the side of the
same square; in our modern notation the
length of the diagonal is $\sqrt{2}$~times the length
of the side. But there is no fraction which
exactly represents~$\sqrt{2}$. We can approximate
\PageSep{73}
to~$\sqrt{2}$ as closely as we like, but we never
exactly reach its value. For example, $\dfrac{49}{25}$~is
just less than~$2$, and $\dfrac{9}{4}$~is greater than~$2$, so
that $\sqrt{2}$~lies between $\dfrac{7}{5}$ and~$\dfrac{3}{2}$. But the best
systematic way of approximating to~$\sqrt{2}$ in
obtaining a series of decimal fractions, each
bigger than the last, is by the ordinary method
of extracting the square root; thus the series
is $1$, $\dfrac{14}{10}$, $\dfrac{141}{100}$, $\dfrac{1414}{1000}$, and so on.

Ratios of this sort are called by the Greeks
incommensurable. They have excited from
the time of the Greeks onwards a great deal
of philosophic discussion, and the difficulties
connected with them have only recently been
cleared up.

We will put the incommensurable ratios
\index{Real Numbers|EtSeq}%
with the fractions, and consider the whole
set of integral numbers, fractional numbers,
and incommensurable numbers as forming
one class of numbers which we will call ``real
numbers.'' We always think of the real
numbers as arranged in order of magnitude,
starting from zero and going upwards, and
becoming indefinitely larger and larger as we
proceed. The real numbers are conveniently
\PageSep{74}
represented by points on a line. Let $OX$ be
\Diagram{pg76}
any line bounded at~$O$ and stretching away indefinitely
in the direction~$OX$. Take any convenient
point,~$A$, on it, so that $OA$~represents
the unit length; and divide off lengths $AB$,
$BC$, $CD$, and so on, each equal to~$OA$. Then
the point~$O$ represents the number~$0$, $A$~the
number~$1$, $B$~the number~$2$, and so on. In
fact the number represented by any point is
the measure of its distance from~$O$, in terms
of the unit length~$OA$. The points between
$O$ and~$A$ represent the proper fractions and
the incommensurable numbers less than~$1$;
the middle point of~$OA$ represents~$\dfrac{1}{2}$, that of~$AB$
represents~$\dfrac{3}{2}$, that of~$BC$ represents~$\dfrac{5}{2}$, and
so on. In this way every point on~$OX$ represents
some one real number, and every real
number is represented by some one point on~$OX$.

The series (or row) of points along~$OX$,
\index{Series|EtSeq}%
starting from~$O$ and moving regularly in the
direction from $O$ to~$X$, represents the real
numbers as arranged in an ascending order
\PageSep{75}
of size, starting from zero and continually
increasing as we go on.

All this seems simple enough, but even at
\index{Order, Type of|EtSeq}%
this stage there are some interesting ideas to
be got at by dwelling on these obvious facts.
Consider the series of points which represent
the integral numbers only, namely, the points,
$O$,~$A$, $B$, $C$, $D$,~etc. Here there is a first point~$O$,
a definite next point,~$A$, and each point,
such as $A$ or~$B$, has one definite immediate
predecessor and one definite immediate successor,
with the exception of~$O$, which has no
predecessor; also the series goes on indefinitely
without end. This sort of order is
called the type of order of the integers; its
essence is the possession of next-door neighbours
on either side with the exception of
No.~1 in the row. Again consider the integers
and fractions together, omitting the points
which correspond to the incommensurable
ratios. The sort of serial order which we now
obtain is quite different. There is a first
term~$O$; but no term has any immediate predecessor
or immediate successor. This is
easily seen to be the case, for between any
two fractions we can always find another
fraction intermediate in value. One very
simple way of doing this is to add the fractions
together and to halve the result. For example,
%[** textstyle fractions start here]
between $\frac{2}{3}$ and~$\frac{3}{4}$, the fraction $\frac{1}{2}(\frac{2}{3} + \frac{3}{4})$,
that is~$\frac{17}{24}$, lies; and between $\frac{2}{3}$ and $\frac{17}{24}$ the
\PageSep{76}
\index{Compact Series}%
fraction $\frac{1}{2}(\frac{2}{3} + \frac{17}{24})$, that is~$\frac{33}{48}$, lies; and so on
indefinitely. Because of this property the
series is said to be ``compact.'' There is no
end point to the series, which increases indefinitely
without limit as we go along the
line~$OX$. It would seem at first sight as
though the type of series got in this way from
the fractions, always including the integers,
would be the same as that got from all the
real numbers, integers, fractions, and incommensurables
taken together, that is, from all
the points on the line~$OX$. All that we have
hitherto said about the series of fractions
applies equally well to the series of all real
numbers. But there are important differences
which we now proceed to develop. The
absence of the incommensurables from the
series of fractions leaves an absence of endpoints
to certain classes. Thus, consider the
incommensurable~$\sqrt{2}$. In the series of real
numbers this stands between all the numbers
whose squares are less than~$2$, and all the
numbers whose squares are greater than~$2$.
But keeping to the series of fractions alone
and not thinking of the incommensurables, so
that we cannot bring in~$\sqrt{2}$, there is no fraction
which has the property of dividing off
the series into two parts in this way, \ie\ so
that all the members on one side have their
squares less than~$2$, and on the other side
greater than~$2$. Hence in the series of fractions
\PageSep{77}
there is a quasi-gap where $\sqrt{2}$~ought to
come. This presence of quasi-gaps in the
series of fractions may seem a small matter;
but any mathematician, who happens to read
this, knows that the possible absence of limits
\index{Limits}%
or maxima to a class of numbers, which yet
does not spread over the whole series of numbers,
is no small evil. It is to avoid this
difficulty that recourse is had to the incommensurables,
so as to obtain a complete series
with no gaps.

There is another even more fundamental
difference between the two series. We can
rearrange the fractions in a series like that of
the integers, that is, with a first term, and
such that each term has an immediate successor
and (except the first term) an immediate
predecessor. We can show how this can be
done. Let every term in the series of fractions
and integers be written in the fractional form
by writing $\frac{1}{1}$ for~$1$, $\frac{2}{1}$ for~$2$, and so on for all the
integers, excluding~$0$. Also for the moment
we will reckon fractions which are equal in
value but not reduced to their lowest terms
as distinct; so that, for example, until further
notice $\frac{2}{3}$, $\frac{4}{6}$, $\frac{6}{9}$, $\frac{8}{12}$, etc., are all reckoned as distinct.
Now group the fractions into classes
by adding together the numerator and denominator
of each term. For the sake of
brevity call this sum of the numerator and
denominator of a fraction its index. Thus $7$~is
\PageSep{78}
the index of~$\frac{4}{3}$, and also of~$\frac{3}{4}$, and of~$\frac{2}{5}$. Let
the fractions in each class be all fractions
which have some specified index, which may
therefore also be called the class index. Now
arrange these classes in the order of magnitude
of their indices. The first class has
the index~$2$, and its only member is~$\frac{1}{1}$; the
second class has the index~$3$, and its members
are $\frac{1}{2}$ and~$\frac{2}{1}$; the third class has the index~$4$,
and its members are $\frac{1}{3}$, $\frac{2}{2}$,~$\frac{3}{1}$; the fourth
class has the index~$5$, and its members are
$\frac{1}{4}$, $\frac{2}{3}$, $\frac{3}{2}$,~$\frac{4}{1}$; and so on. It is easy to see that
the number of members (still including fractions
not in their lowest terms) belonging to
any class is one less than its index. Also the
members of any one class can be arranged
in order by taking the first member to be the
fraction with numerator~$1$, the second member
to have the numerator~$2$, and so on, up to~$(n - 1)$
where $n$~is the index. Thus for the
class of index~$n$, the members appear in the
order\Typo{.}{}
%[** TN: Reformatted slightly from the original]
\[
\frac{1}{n - 1},\quad
\frac{2}{n - 2},\quad
\frac{3}{n - 3},\
\dots,\quad
\frac{n - 1}{1}.
\]
The members
of the first four classes have in fact been
mentioned in this order. Thus the whole set
of fractions have now been arranged in an
order like that of the integers. It runs thus
\begin{gather*}
\frac{1}{1},\
\frac{1}{2},\
\frac{2}{1},\
\frac{1}{3},\
\left[\frac{2}{2}\right],\
\frac{3}{1},\
\frac{1}{4},\
\frac{2}{3},\
\frac{3}{2},\
\frac{4}{1},\  \dots, \\
%\PageSep{79}
\frac{n - 1}{1},\
\frac{1}{n - 1},\
\frac{2}{n - 2},\
\frac{3}{n - 3},\
\dots,\
\frac{n - 1}{1},\
\frac{1}{n},
\end{gather*}
and so on.

Now we can get rid of all repetitions of
fractions of the same value by simply striking
them out whenever they appear after their
first occurrence. In the few initial terms
written down above, $\frac{2}{2}$~which is enclosed above
in square brackets is the only fraction not in
its lowest terms. It has occurred before as~$\frac{1}{1}$.
Thus this must be struck out. But the
series is still left with the same properties,
namely, (\textit{a})~there is a first term, (\textit{b})~each term
has next-door neighbours, (\textit{c})~the series goes
on without end.

It can be proved that it is not possible to
\index{Cantor, Georg}%
arrange the whole series of real numbers in
this way. This curious fact was discovered
by Georg Cantor, a German mathematician
still living; it is of the utmost importance
in the philosophy of mathematical ideas. We
are here in fact touching on the fringe of the
great problems of the meaning of continuity
and of infinity.

Another extension of number comes from
\index{Steps|EtSeq}%
the introduction of the idea of what has been
variously named an operation or a step,
names which are respectively appropriate
from slightly different points of view. We
will start with a particular case. Consider
\PageSep{80}
the statement $2 + 3 = 5$. We add $3$ to~$2$ and
obtain~$5$. Think of the operation of adding~$3$:
let this be denoted by~$+3$. Again $4 - 3 = 1$.
Think of the operation of subtracting~$3$:
let this be denoted by~$-3$. Thus instead
of considering the real numbers in themselves,
we consider the \emph{operations} of adding or subtracting
them: instead of~$\sqrt{2}$, we consider
$+\sqrt{2}$ and~$-\sqrt{2}$, namely the operations of
adding~$\sqrt{2}$ and of subtracting~$\sqrt{2}$. Then we
can add these operations, of course in a
different sense of addition to that in which we
add numbers. The sum of two operations is
the single operation which has the same effect
as the two operations applied successively.
In what order are the two operations to be
applied? The answer is that it is indifferent,
since for example
\[
2 + 3 + 1 = 2 + 1 + 3;
\]
so that the addition of the steps $+3$ and $+1$
is commutative.

Mathematicians have a habit, which is
puzzling to those engaged in tracing out
meanings, but is very convenient in practice,
of using the same symbol in different though
allied senses. The one essential requisite for
a symbol in their eyes is that, whatever its
possible varieties of meaning, the formal laws
for its use shall always be the same. In
\PageSep{81}
accordance with this habit the addition of
operations is denoted by~$+$ as well as the
addition of numbers. Accordingly we can
write
\[
(+3) + (+1) = +4;
\]
where the middle~$+$ on the left-hand side
denotes the addition of the operations $+3$
and~$+1$. But, furthermore, we need not be
so very pedantic in our symbolism, except in
the rare instances when we are directly tracing
meanings; thus we always drop the first~$+$
of a line and the brackets, and never write
two $+$~signs running. So the above equation
becomes
\[
3 + 1 = 4,
\]
which we interpret as simple numerical addition,
or as the more elaborate addition of
operations which is fully expressed in the
previous way of writing the equation, or
lastly as expressing the result of applying
the operation~$+1$ to the number~$3$ and obtaining
the number~$4$. Any interpretation
which is possible is always correct. But the
only interpretation which is always possible,
under certain conditions, is that of operations.
The other interpretations often give nonsensical
results.

This leads us at once to a question, which
must have been rising insistently in the
\PageSep{82}
reader's mind: What is the use of all this
elaboration? At this point our friend, the
practical man, will surely step in and insist on
sweeping away all these silly cobwebs of the
brain. The answer is that what the mathematician
is seeking is Generality. This is an
\index{Generality in Mathematics}%
idea worthy to be placed beside the notions
of the Variable and of Form so far as concerns
\index{Form, Algebraic}%
\index{Variable, The}%
its importance in governing mathematical
procedure. Any limitation whatsoever upon
the generality of theorems, or of proofs, or of
interpretation is abhorrent to the mathematical
instinct. These three notions, of the
variable, of form, and of generality, compose
a sort of mathematical trinity which preside
over the whole subject. They all really
spring from the same root, namely from the
abstract nature of the science.

Let us see how generality is gained by the
introduction of this idea of operations. Take
the equation $x + 1 = 3$; the solution is $x = 2$.
Here we can interpret our symbols as mere
numbers, and the recourse to ``operations''
is entirely unnecessary. But, if $x$~is a mere
number, the equation $x + 3 = 1$ is nonsense.
For $x$~should be the number of things which
remain when you have taken $3$~things away
from $1$~thing; and no such procedure is
possible. At this point our idea of algebraic
form steps in, itself only generalization under
another aspect. We consider, therefore, the
\PageSep{83}
\index{Positive and Negative Numbers|EtSeq}%
general equation of the same form as $x + 1 = 3$.
This equation is $x + a = b$, and its solution is
$x = b - a$. Here our difficulties become acute;
for this form can only be used for the numerical
interpretation so long as $b$~is greater than~$a$,
and we cannot say without qualification
that $a$ and $b$ may be any constants. In other
words we have introduced a limitation on
the variability of the ``constants'' $a$~and~$b$,
which we must drag like a chain throughout
all our reasoning. Really prolonged mathematical
investigations would be impossible
under such conditions. Every equation
would at last be buried under a pile of limitations.
But if we now interpret our symbols
as ``operations,'' all limitation vanishes like
magic. The equation $x + 1 = 3$ gives $x = +2$,
the equation $x + 3 = 1$ gives $x = -2$, the equation
$x + a = b$ gives $x = b - a$ which is an operation
of addition or subtraction as the case
may be. We need never decide whether $b - a$
represents the operation of addition or of
subtraction, for the rules of procedure with
the symbols are the same in either case.

It does not fall within the plan of this work
to write a detailed chapter of elementary
algebra. Our object is merely to make plain
the fundamental ideas which guide the formation
of the science. Accordingly we do not
further explain the detailed rules by which
the ``positive and negative numbers'' are
\PageSep{84}
multiplied and otherwise combined. We have
explained above that positive and negative
numbers are operations. They have also
been called ``steps.'' Thus $+3$~is the step
by which we go from $2$ to~$5$, and $-3$~is the
step backwards by which we go from $5$ to~$2$.
Consider the line~$OX$ divided in the way explained
in the earlier part of the chapter, so
that its points represent numbers. Then~$+2$
%[** TN: In original, negative numbers placed below axis, primed letters above]
\Diagram{pg86}
is the step from $O$ to~$B$, or from $A$ to~$C$, or
(if the divisions are taken backwards along~$OX'$)
from $C'$ to~$A'$, or from $D'$ to~$B'$, and so
on. Similarly $-2$ is the step from $O$ to~$B'$,
or from $B'$ to~$D'$, or from $B$ to~$O$, or from $C$
to~$A$.

We may consider the point which is reached
by a step from~$O$, as representative of that
step. Thus $A$~represents~$+1$, $B$~represents~$+2$,
$A'$~represents~$-1$, $B'$~represents~$-2$, and
so on. It will be noted that, whereas previously
with the mere ``unsigned'' real numbers
the points on one side of~$O$ only, namely along~$OX$,
were representative of numbers, now
with steps every point on the whole line
stretching on both sides of~$O$ is representative
of a step. This is a pictorial representation
of the superior generality introduced by the
positive and negative numbers, namely the
\PageSep{85}
operations or steps. These ``signed'' numbers
are also particular cases of what have
been called vectors (from the Latin \Foreign{veho}, I
\index{Vectors}%
draw or carry). For we may think of a
particle as carried from $O$ to~$A$, or from $A$
to~$B$.

In suggesting a few pages ago that the
practical man would object to the subtlety
involved by the introduction of the positive
and negative numbers, we were libelling that
excellent individual. For in truth we are on
the scene of one of his greatest triumphs. If
the truth must be confessed, it was the practical
man himself who first employed the actual
symbols $+$ and~$-$. Their origin is not very
certain, but it seems most probable that they
arose from the marks chalked on chests of
goods in German warehouses, to denote excess
or defect from some standard weight. The
earliest notice of them occurs in a book published
at Leipzig, in \AD~1489. They seem
first to have been employed in mathematics
by a German mathematician, Stifel, in a book
\index{Stifel}%
published at Nuremburg in 1544~\AD. But
then it is only recently that the Germans
have come to be looked on as emphatically
a practical nation. There is an old epigram
which assigns the empire of the sea to the
English, of the land to the French, and of the
clouds to the Germans. Surely it was from
the clouds that the Germans fetched $+$ and~$-$;
\PageSep{86}
the ideas which these symbols have
generated are much too important for the
welfare of humanity to have come from the
sea or from the land.

The possibilities of application of the positive
and negative numbers are very obvious.
If lengths in one direction are represented
by positive numbers, those in the opposite
direction are represented by negative numbers.
If a velocity in one direction is positive, that
in the opposite direction is negative. If a
rotation round a dial in the opposite direction
to the hands of a clock (anti-clockwise) is
positive, that in the clockwise direction is
negative. If a balance at the bank is positive,
an overdraft is negative. If vitreous
electrification is positive, resinous electrification
is negative. Indeed, in this latter case,
the terms positive electrification and negative
electrification, considered as mere names,
have practically driven out the other terms.
An endless series of examples could be given.
The idea of positive and negative numbers
has been practically the most successful of
mathematical subtleties.
\PageSep{87}


\Chapter{VII}{Imaginary Numbers}

\First{If} the mathematical ideas dealt with in the
\index{Imaginary Numbers|EtSeq}%
last chapter have been a popular success,
those of the present chapter have excited
almost as much general attention. But their
success has been of a different character, it
has been what the French term a \Foreign{succès de
scandale}. Not only the practical man, but
also men of letters and philosophers have expressed
their bewilderment at the devotion
of mathematicians to mysterious entities
which by their very name are confessed to be
imaginary. At this point it may be useful
to observe that a certain type of intellect
is always worrying itself and others by
discussion as to the applicability of technical
terms. Are the incommensurable numbers
properly called numbers? Are the positive
and negative numbers really numbers? Are
the imaginary numbers imaginary, and are
they numbers?---are types of such futile
questions. Now, it cannot be too clearly
understood that, in science, technical terms
are names arbitrarily assigned, like Christian
\PageSep{88}
names to children. There can be no question
of the names being right or wrong. They
may be judicious or injudicious; for they can
sometimes be so arranged as to be easy to
remember, or so as to suggest relevant and
important ideas. But the essential principle
involved was quite clearly enunciated in
Wonderland to Alice by Humpty Dumpty,
when he told her, à~propos of his use of words,
``I pay them extra and make them mean
what I like.'' So we will not bother as to
whether imaginary numbers are imaginary,
or as to whether they are numbers, but will
take the phrase as the arbitrary name of a
certain mathematical idea, which we will now
endeavour to make plain.

The origin of the conception is in every
way similar to that of the positive and negative
numbers. In exactly the same way it
is due to the three great mathematical ideas
of the variable, of algebraic form, and of
generalization. The positive and negative
numbers arose from the consideration of
equations like $x + 1 = 3$, $x + 3 = 1$, and the
general form $x + a = b$. Similarly the origin
of imaginary numbers is due to equations like
$x^{2} + 1 = 3$, $x^{2} + 3 = 1$, and $x^{2} + a = b$. Exactly
the same process is gone through. The equation
$x^{2} + 1 = 3$ becomes $x^{2} = 2$, and this has two
solutions, either $x = +\sqrt{2}$, or $x = -\sqrt{2}$. The
statement that there are these alternative
\PageSep{89}
solutions is usually written $x = ±\sqrt{2}$. So far
all is plain sailing, as it was in the previous
case. But now an analogous difficulty arises.
For the equation $x^{2} + 3 = 1$ gives $x^{2} = -2$ and
there is no positive or negative number which,
when multiplied by itself, will give a negative
square. Hence, if our symbols are to mean
the ordinary positive or negative numbers,
there is no solution to $x^{2} = -2$, and the equation
is in fact nonsense. Thus, finally taking
the general form $x^{2} + a = b$, we find the pair
of solutions $x = ±\sqrt{(b - a)}$, when, and only
when, $b$~is not less than~$a$. Accordingly we
cannot say unrestrictedly that the ``constants''
$a$~and~$b$ may be any numbers, that is,
the ``constants'' $a$~and~$b$ are not, as they
ought to be, independent unrestricted ``variables'';
and so again a host of limitations
and restrictions will accumulate round our
work as we proceed.

The same task as before therefore awaits
us: we must give a new interpretation to our
symbols, so that the solutions $±\sqrt{(b -  a)}$ for
the equation $x^{2} + a = b$ always have meaning.
In other words, we require an interpretation
of the symbols so that $\sqrt{a}$~always has meaning
whether $a$~be positive or negative. Of
course, the interpretation must be such that
all the ordinary formal laws for addition, subtraction,
multiplication, and division hold
good; and also it must not interfere with the
\PageSep{90}
generality which we have attained by the use
of the positive and negative numbers. In
fact, it must in a sense include them as
special cases. When $a$~is negative we may
write $-c^{2}$ for it, so that $c^{2}$~is positive. Then
\begin{align*}
\sqrt{a} &= \sqrt{(-c^{2})} = \sqrt{\{(-1) × c^{2}\}} \\
  &= \sqrt{(-1)} \sqrt{c^{2}} = c\sqrt{(-1)}.
\end{align*}
Hence, if we can so interpret our symbols that
$\sqrt{(-1)}$~has a meaning, we have attained our
object. Thus $\sqrt{(-1)}$~has come to be looked
on as the head and forefront of all the
imaginary quantities.

This business of finding an interpretation
for~$\sqrt{(-1)}$ is a much tougher job than the
analogous one of interpreting~$-1$. In fact,
while the easier problem was solved almost
instinctively as soon as it arose, it at first
hardly occurred, even to the greatest mathematicians,
that here a problem existed which
was perhaps capable of solution. Equations
like $x^{2} = -3$, when they arose, were simply
ruled aside as nonsense.

However, it came to be gradually perceived
during the eighteenth century, and even
earlier, how very convenient it would be if
an interpretation could be assigned to these
nonsensical symbols. Formal reasoning with
these symbols was gone through, merely
assuming that they obeyed the ordinary
\PageSep{91}
algebraic laws of transformation; and it was
seen that a whole world of interesting results
could be attained, if only these symbols might
legitimately be used. Many mathematicians
were not then very clear as to the logic of
their procedure, and an idea gained ground
that, in some mysterious way, symbols which
mean nothing can by appropriate manipulation
yield valid proofs of propositions. Nothing
can be more mistaken. A symbol
which has not been properly defined is not a
symbol at all. It is merely a blot of ink on
paper which has an easily recognized shape.
Nothing can be proved by a succession of
blots, except the existence of a bad pen or a
careless writer. It was during this epoch
that the epithet ``imaginary'' came to be
applied to~$\sqrt{(-1)}$. What these mathematicians
had really succeeded in proving were
a series of hypothetical propositions, of which
this is the blank form: If interpretations
exist for $\sqrt{(-1)}$ and for the addition, subtraction,
multiplication, and division of~$\sqrt{(-1)}$
which make the ordinary algebraic
rules (\eg\ $x + y = y + x$, etc.)\ to be satisfied,
then such and such results follows. It was
natural that the mathematicians should not
always appreciate the big ``If,'' which ought
to have preceded the statements of their results.

As may be expected the interpretation,
\PageSep{92}
when found, was a much more elaborate affair
than that of the negative numbers and the
reader's attention must be asked for some
careful preliminary explanation. We have
already come across the representation of a
point by two numbers. By the aid of the
\Figure{8}
positive and negative numbers we can now
represent the position of any point in a plane
by a pair of such numbers. Thus we take
the pair of straight lines $XOX'$ and $YOY'$, at
right angles, as the ``axes'' from which we
start all our measurements. Lengths measured
along $OX$ and $OY$ are positive, and
measured backwards along $OX'$ and $OY'$ are
negative. Suppose that a pair of numbers,
written in order, \eg~$(+3, +1)$, so that there
\PageSep{93}
\index{Ordered Couples|EtSeq}%
is a first number ($+3$~in the above example),
and a second number ($+1$~in the above example),
represents measurements from~$O$
along $XOX'$ for the first number, and along
$YOY'$ for the second number. Thus (\Chg{cf.}{\cf}\ \Fig[fig.]{9}) in
$(+3, +1)$ a length of $3$~units is to be measured
along $XOX'$ in the positive direction, that
is from~$O$ towards~$X$, and a length~$+1$
measured along $YOY'$ in the positive direction,
that is from~$O$ towards~$Y$. Similarly in
$(-3, +1)$ the length of $3$~units is to be
measured from~$O$ towards~$X'$, and of $1$~unit
from towards~$Y$. Also in $(-3, -1)$ the
two lengths are to be measured along $OX'$
and $OY'$ respectively, and in $(+3, -1)$ along
$OX$ and $OY'$ respectively. Let us for the
moment call such a pair of numbers an
``ordered couple.'' Then, from the two numbers
$1$~and~$3$, eight ordered couples can be
generated, namely
\begin{gather*}
(+1, +3),\ (-1, +3),\ (-1, -3),\ (+1, -3), \\
(+3, +1),\ (-3, +1),\ (-3, -1),\ (+3, -1).
\end{gather*}
Each of these eight ``ordered couples'' directs
a process of measurement along $XOX'$ and
$YOY'$ which is different from that directed
by any of the others.

The processes of measurement represented
by the last four ordered couples, mentioned
above, are given pictorially in the figure.
The lengths $OM$ and $ON$ together correspond
\PageSep{94}
to $(+3, +1)$, the lengths $OM'$ and $ON$
together correspond to $(-3, +1)$, $OM'$~and
$ON'$ together to $(-3, -1)$, and $OM$~and
$ON'$ together to $(+3, -1)$. But by completing
the various rectangles, it is easy to
see that the point~$P$ completely determines
and is determined by the ordered couple
\Figure{9}
$(+3, +1)$, the point~$P'$ by $(-3, +1)$, the
point~$P''$ by $(-3, -1)$, and the point~$P'''$ by
$(+3, -1)$. More generally in the previous
figure~(\FigNum{8}), the point~$P$ corresponds to the
ordered couple~$(x, y)$, where $x$~and~$y$ in the
figure are both assumed to be positive, the
point~$P'$ corresponds to $(x', y)$, where $x'$~in
the figure is assumed to be negative, $P''$~to
$(x' y')$, and $P'''$~to $(x, y')$. Thus an ordered
\PageSep{95}
couple $(x, y)$, where $x$~and~$y$ are any positive
or negative numbers, and the corresponding
point reciprocally determine each other. It
is convenient to introduce some names at this
juncture. In the ordered couple $(x, y)$ the
first number~$x$ is called the ``abscissa'' of the
\index{Abscissa}%
corresponding point, and the second number~$y$
is called the ``ordinate'' of the point, and
\index{Ordinate}%
the two numbers together are called the ``coordinates''
\index{Coordinates}%
of the point. The idea of determining
the position of a point by its ``coordinates''
was by no means new when the
theory of ``imaginaries'' was being formed.
It was due to Descartes, the great French
\index{Descartes}%
mathematician and philosopher, and appears
in his \Title{Discours} published at Leyden in 1637~\AD.
The idea of the ordered couple as a
thing on its own account is of later growth
and is the outcome of the efforts to interpret
imaginaries in the most abstract way possible.

It may be noticed as a further illustration
of this idea of the ordered couple, that the
point~$M$ in \Fig[fig.]{9} is the couple $(+3, 0)$, the
point~$N$ is the couple $(0, +1)$, the point~$M'$
the couple $(-3, 0)$, the point~$N'$ the couple
$(0, -1)$, the point~$O$ the couple~$(0, 0)$.

Another way of representing the ordered
couple $(x, y)$ is to think of it as representing
the dotted line~$OP$ (\Chg{cf.}{\cf}\ \Fig[fig.]{8}), rather than the
point~$P$. Thus the ordered couple represents
a line drawn from an ``origin,''~$O$, of a certain
\index{Origin}%
\PageSep{96}
\index{Steps}%
length and in a certain direction. The line~$OP$
may be called the vector line from $O$ to~$P$,
or the step from $O$ to~$P$. We see, therefore,
that we have in this chapter only extended
the interpretation which we gave formerly of
the positive and negative numbers. This
method of representation by vectors is very
\index{Vectors}%
useful when we consider the meaning to be
assigned to the operations of the addition and
multiplication of ordered couples.

{\Loosen We will now go on to this question, and
ask what meaning we shall find it convenient
to assign to the addition of the two ordered
couples $(x, y)$ and $(x', y')$. The interpretation
must, (\textit{a})~make the result of addition
to be another ordered couple, (\textit{b})~make the
operation commutative so that $(x, y) + (x', y') = (x', y') + (x, y)$,
(\textit{c})~make the operation
associative so that}
\[
\{(x, y) + (x', y')\} + (u, v) = (x, y) + \{(x', y') + (u, v)\},
\]
(\textit{d})~make the result of subtraction unique,
so that when we seek to determine the
unknown ordered couple $(x, y)$ so as to
satisfy the equation
\[
(x, y) + (a, b) = (c, d),
\]
there is one and only one answer which we
can represent by
\[
(x, y) = (c, d) - (a, b).
\]
\PageSep{97}
All these requisites are satisfied by taking
$(x, y) + (x', y')$ to mean the ordered couple
$(x + x', y + y')$. Accordingly by definition we
put
\[
(x, y) + (x', y') = (x + x', y + y').
\]
Notice that here we have adopted the mathematical
habit of using the same symbol~$+$ in
different senses. The $+$ on the left-hand side
of the equation has the new meaning of~$+$
which we are just defining; while the two~$+$'s
on the right-hand side have the meaning
of the addition of positive and negative numbers
(operations) which was defined in the
last chapter. No practical confusion arises
from this double use.

As examples of addition we have
\begin{align*}
(+3, +1) + (+2, +6) &= (+5, +7), \\
(+3, -1) + (-2, -6) &= (+1, -7), \\
(+3, +1) + (-3, -1) &= (0, 0).
\end{align*}

The meaning of subtraction is now settled
for us. We find that
\[
(x, y) - (u, v) = (x - u, y - v).
\]
Thus
\[
(+3, +2) - (+1, +1) = (+2, +1),
\]
and
\[
(+1, -2) - (+2, -4) = (-1, +2),
\]
and
\[
(-1, -2) - (+2, +3) = (-3, -5).
\]
\PageSep{98}

It is easy to see that
\[
(x, y) - (u, v) = (x, y) + (-u, -v).
\]
Also
\[
(x, y) - (x, y) = (0, 0).
\]
Hence $(0, 0)$~is to be looked on as the zero
ordered couple. For example
\[
(x, y) + (0, 0) = (x, y).
\]

The pictorial representation of the addition
of ordered couples is surprisingly easy.
\Figure{10}

{\Loosen Let $OP$ represent $(x, y)$ so that $OM = x$
and $PM = y$; let $OQ$ represent $(x_{1}, y_{1})$ so that
$OM_{1} = x_{1}$ and $QM_{1} = y_{1}$. Complete the parallelogram
$OPRQ$ by the dotted lines $PR$ and~$QR$,
then the diagonal~$OR$ is the ordered
couple $(x + x_{1}, y + y_{1})$. For draw $PS$ parallel
\PageSep{99}
to~$OX$; then evidently the triangles $OQM_{1}$
and $PRS$ are in all respects equal. Hence
$MM' = PS = x_{1}$, and $RS = QM_{1}$ and therefore}
\begin{gather*}
OM' = OM + MM' = x + x_{1}, \\
RM' = SM' + RS = y + y_{1}.
\end{gather*}

Thus $OR$~represents the ordered couple as
required. This figure can also be drawn with
$OP$ and $OQ$ in other quadrants.

It is at once obvious that we have here
come back to the parallelogram law, which
\index{Parallelogram Law}%
was mentioned in \ChapRef{VI}., on the laws of
motion, as applying to velocities and forces.
It will be remembered that, if $OP$ and $OQ$
represent two velocities, a particle is said to
be moving with a velocity equal to the two
velocities added together if it be moving with
the velocity~$OR$. In other words $OR$~is said
to be the resultant of the two velocities $OP$
and~$OQ$. Again forces acting at a point of a
body can be represented by lines just as
velocities can be; and the same parallelogram
law holds, namely, that the resultant of the
two forces $OP$ and $OQ$ is the force represented
by the diagonal~$OR$. It follows that we can
look on an ordered couple as representing a
velocity or a force, and the rule which we
have just given for the addition of ordered
couples then represents the fundamental laws
of mechanics for the addition of forces and
\PageSep{100}
velocities. One of the most fascinating
characteristics of mathematics is the surprising
way in which the ideas and results of
different parts of the subject dovetail into
each other. During the discussions of this
and the previous chapter we have been guided
merely by the most abstract of pure mathematical
considerations; and yet at the end
of them we have been led back to the most
fundamental of all the laws of nature, laws
which have to be in the mind of every engineer
as he designs an engine, and of every naval
architect as he calculates the stability of a
ship. It is no paradox to say that in our
most theoretical moods we may be nearest to
our most practical applications.
\PageSep{101}


\Chapter[Imaginary Numbers]
  {VIII}{Imaginary Numbers (\textit{C\MakeLowercase{ontinued}})}

\First{The} definition of the multiplication of
ordered couples is guided by exactly the same
considerations as is that of their addition.
The interpretation of multiplication must be
such that

\Eq{(\alpha)} the result is another ordered couple,

\Eq{(\beta)} the operation is commutative, so that
\[
(x, y) × (x', y') = (x', y') × (x, y),
\]

\Eq{(\gamma)} the operation is associative, so that
\[
\{(x, y) × (x', y')\} × (u, v) = (x, y) × \{(x', y') × (u, v)\},
\]

\Eq{(\delta)} must make the result of division unique
[with an exception for the case of the zero
couple $(0, 0)$], so that when we seek to determine
the unknown couple $(x, y)$ so as to
satisfy the equation
\[
(x, y) × (a, b) = (c, d),
\]
there is one and only one answer, which we
can represent by
\[
(x, y) = (c, d) ÷ (a, b),\quad\text{or by}\quad
(x, y) = \frac{(c, d)}{(a, b)}\Add{.}
\]
\PageSep{102}

\Eq{(\epsilon)} Furthermore the law involving both
addition and multiplication, called the distributive
law, must be satisfied, namely
\begin{multline*}
(x,y) × \{(a, b) + (c, d)\} \\
= \{(x, y) × (a, b)\} + \{(x, y) × (c, d)\}.
\end{multline*}

All these conditions \Eq{(\alpha)}, \Eq{(\beta)}, \Eq{(\gamma)}, \Eq{(\delta)}, \Eq{(\epsilon)} can
be satisfied by an interpretation which,
though it looks complicated at first, is capable
of a simple geometrical interpretation.

By definition we put
\[
(x, y) × (x', y') = \{(xx' - yy'), (xy' + x'y)\}\Add{.}
\Tag{(A)}
\]

This is the definition of the meaning of the
symbol~$×$ when it is written between two
ordered couples. It follows evidently from
this definition that the result of multiplication
is another ordered couple, and that the
value of the right-hand side of equation~\Eq{(A)}
is not altered by simultaneously interchanging
$x$~with~$x'$, and $y$~with~$y'$. Hence conditions
\Eq{(\alpha)} and \Eq{(\beta)} are evidently satisfied. The proof
of the satisfaction of \Eq{(\gamma)}, \Eq{(\delta)}, \Eq{(\epsilon)} is equally
easy when we have given the geometrical
interpretation, which we will proceed to do
in a moment. But before doing this it will
be interesting to pause and see whether we
have attained the object for which all this
elaboration was initiated.

We came across equations of the form
$x^{2} = -3$, to which no solutions could be
\PageSep{103}
assigned in terms of positive and negative real
numbers. We then found that all our difficulties
would vanish if we could interpret the
equation $x^{2} = -1$, \ie, if we could so define
$\sqrt{(-1)}$ that $\sqrt{(-1)} × \sqrt{(-1)} = -1$.

Now let us consider the three special
\index{Zero}%
ordered couples\footnote
  {For the future we follow the custom of omitting the
  $+$~sign wherever possible, thus $(1, 0)$ stands for $(+1, 0)$
  and $(0, 1)$ for $(0, +1)$.}
$(0, 0)$, $(1, 0)$, and $(0, 1)$.

We have already proved that
\[
(x, y) + (0, 0) = (x, y).
\]

Furthermore we now have
\[
(x, y) × (0, 0) = (0, 0).
\]

Hence both for addition and for multiplication
the couple $(0, 0)$ plays the part of zero in
elementary arithmetic and algebra; compare
the above equations with $x + 0 = x$, and
$x × 0 = 0$.

Again consider $(1, 0)$: this plays the part
of~$1$ in elementary arithmetic and algebra.
In these elementary sciences the special
characteristic of~$1$ is that $x × 1 = x$, for all
values of~$x$. Now by our law of multiplication
\[
(x, y) × (1, 0) = \{(x - 0), (y + 0)\} = (x, y).
\]

Thus $(1, 0)$ is the unit couple.
\PageSep{104}

Finally consider $(0, 1)$: this will interpret
for us the symbol~$\sqrt{(-1)}$. The symbol must
therefore possess the characteristic property
that $\sqrt{(-1)} × \sqrt{(-1)} = -1$. Now by the
law of multiplication for ordered couples
\[
(0, 1) × (0, 1) = \{(0 - 1), (0 + 0)\} = (-1, 0).
\]

But $(1, 0)$ is the unit couple, and $(-1, 0)$
is the negative unit couple; so that $(0, 1)$ has
the desired property. There are, however,
two roots of~$-1$ to be provided for, namely
$±\sqrt{(-1)}$. Consider $(0, -1)$; here again remembering
that $(-1)^{2} = 1$, we find, $(0, -1) × (0, -1) = (-1, 0)$.

Thus $(0, -1)$ is the other square root of~$\Typo{\sqrt{(-1)}}{-1}$.
Accordingly the ordered couples
$(0, 1)$ and $(0, -1)$ are the interpretations of
$±\sqrt{(-1)}$ in terms of ordered couples. But
which corresponds to which? Does $(0, 1)$
correspond to $+\sqrt{(-1)}$ and $(0, -1)$ to~$-\sqrt{(-1)}$,
or $(0, 1)$ to~$-\sqrt{(-1)}$, and $(0, -1)$
to~$+\sqrt{(-1)}$? The answer is that it is perfectly
indifferent which symbolism we adopt.

The ordered couples can be divided into
three types, (i)~the ``complex imaginary''
type~$(x, y)$, in which neither $x$ nor~$y$ is zero;
(ii)~the ``real'' type~$(x, 0)$; (iii)~the ``pure
imaginary'' type~$(0, y)$. Let us consider the
relations of these types to each other. First
multiply together the ``complex imaginary''
\PageSep{105}
couple $(x, y)$ and the ``real'' couple $(a, 0)$, we
find
\[
(a, 0) × (x, y) = (ax, ay).
\]

Thus the effect is merely to multiply each
term of the couple $(x, y)$ by the positive or
negative real number~$a$.

Secondly, multiply together the ``complex
imaginary'' couple $(x, y)$ and the ``pure
imaginary'' couple $(0, b)$, we find
\[
(0, b) × (x, y) = (-by, bx).
\]

Here the effect is more complicated, and is
best comprehended in the geometrical interpretation
to which we proceed after noting
three yet more special cases.

Thirdly, we multiply the ``real'' couple
$(a, 0)$ by the imaginary $(0, b)$ and obtain
\[
(a, 0) × (0, b) =(0, ab).
\]

Fourthly, we multiply the two ``real''
couples $(a, 0)$ and $(a', 0)$ and obtain
\[
(a, 0) × (a', 0) =( aa', 0).
\]

Fifthly, we multiply the two ``imaginary
couples'' $(0, b)$ and $(0, \Typo{b}{b'})$ and obtain
\[
(0, b) × (0, b') = (-bb', 0).
\]

We now turn to the geometrical interpretation,
beginning first with some special cases.
\PageSep{106}
Take the couples $(1, 3)$ and $(2, 0)$ and consider
the equation
\[
(2, 0) × (1, 3) = (2, 6)\Add{.}
\]
\Figure{11}

In the diagram (\Fig[fig.]{11}) the vector~$OP$ represents~$(1, 3)$,
and the vector~$ON$ represents~$(2, 0)$,
and the vector~$OQ$ represents~$(2, 6)$.
Thus the product $(2, 0) × (1, 3)$ is found geometrically
by taking the length of the vector~$OQ$
to be the product of the lengths of the
vectors $OP$ and~$ON$, and (in this case) by
producing $OP$ to~$Q$ to be of the required
length. Again, consider the product $(0, 2) × (1, 3)$,
we have
\[
(0, 2) × (1, 3) = (-6, 2)\Add{.}
\]

The vector~$ON_{1}$, corresponds to~$(0, 2)$ and
the vector~$OR$ to~$(-6,2)$. Thus $OR$ which
\PageSep{107}
represents the new product is at right angles
to~$OQ$ and of the same length. Notice that
we have the same law regulating the length
of~$OQ$ as in the previous case, namely, that
its length is the product of the lengths of
the two vectors which are multiplied together;
but now that we have $ON_{1}$ along the
``ordinate'' axis~$OY$, instead of $ON$ along
the ``abscissa'' axis~$OX$, the direction of~$OP$
has been turned through a right-angle.

Hitherto in these examples of multiplication
we have looked on the vector~$OP$ as modified
by the vectors $ON$ and~$ON_{1}$. We shall get
a clue to the general law for the direction by
inverting the way of thought, and by thinking
of the vectors $ON$ and~$ON_{1}$ as modified by
the vector~$OP$. The law for the length remains
unaffected; the resultant length is the
length of the product of the two vectors.
The new direction for the enlarged~$ON$ (\ie~$OQ$)
is found by rotating it in the (anti-clockwise)
direction of rotation from $OX$ towards~$OY$
through an angle equal to the angle~$XOP$:
it is an accident of this particular case that
this rotation makes $OQ$ lie along the line~$OP$.
Again consider the product of $ON_{1}$ and~$OP$;
the new direction for the enlarged~$ON_{1}$ (\ie~$OR$)
is found by rotating~$ON$ in the anti-clockwise
direction of rotation through an
angle equal to the angle~$XOP$, namely, the
angle~$N_{1}OR$ is equal to the angle~$XOP$.
\PageSep{108}

The general rule for the geometrical representation
of multiplication can now be enunciated
thus:
\Figure[3in]{12}

The product of the two vectors $OP$ and~$OQ$
is a vector~$OR$, whose length is the product
of the lengths of $OP$ and~$OQ$ and whose
direction~$OR$ is such that the angle~$XOR$ is
equal to the sum of the angles $XOP$ and~$XOQ$.

Hence we can conceive the vector~$OP$ as
making the vector~$OQ$ rotate through an
angle~$XOP$ (\ie\ $\text{the angle } QOR = \text{the angle } XOP$),
or the vector~$OQ$ as making the vector~$OP$
rotate through the angle~$XOQ$ (\ie $\text{the angle } POR = \text{the angle } XOQ$).

We do not prove this general law, as we
\PageSep{109}
should thereby be led into more technical
processes of mathematics than falls within the
design of this book. But now we can immediately
see that the associative law [numbered~\Eq{(\gamma)}
above] for multiplication is satisfied.
Consider first the length of the resultant
vector; this is got by the ordinary process
of multiplication for real numbers; and thus
the associative law holds for it.

Again, the direction of the resultant vector
is got by the mere addition of angles, and the
associative law holds for this process also.

So much for multiplication. We have now
rapidly indicated, by considering addition and
multiplication, how an algebra or ``calculus''
of vectors in one plane can be constructed,
which is such that any two vectors in the
plane can be added, or subtracted, and can
be multiplied, or divided one by the other.

We have not considered the technical details
of all these processes because it would
lead us too far into mathematical details;
but we have shown the general mode of procedure.
When we are interpreting our algebraic
symbols in this way, we are said to be
employing ``imaginary quantities'' or ``complex
\index{Complex Quantities}%
\index{Imaginary Quantities}%
quantities.'' These terms are mere
details, and we have far too much to think
about to stop to enquire whether they are or
are not very happily chosen.

%[** TN: [sic] "nett", variant spelling]
The nett result of our investigations is that
\PageSep{110}
any equations like $x + 3 = 2$ or $(x + 3)^{2} = -2$
can now always be interpreted into terms of
vectors, and solutions found for them. In
seeking for such interpretations it is well to
note that $3$~becomes $(3, 0)$, and $-2$~becomes
$(-2, 0)$, and $x$~becomes the ``unknown''
couple $(u, v)$: so the two equations become
respectively $(u, v) + (3, 0) = (2, 0)$, and
$\{(u, v) + (3, 0)\}^{2} = (-2, 0)$.

We have now completely solved the initial
difficulties which caught our eye as soon as
we considered even the elements of algebra.
The science as it emerges from the solution is
much more complex in ideas than that with
which we started. We have, in fact, created
a new and entirely different science, which
will serve all the purposes for which the old
science was invented and many more in addition.
But, before we can congratulate ourselves
on this result to our labours, we must
allay a suspicion which ought by this time to
have arisen in the mind of the student. The
question which the reader ought to be asking
himself is: Where is all this invention of new
interpretations going to end? It is true that
we have succeeded in interpreting algebra so
as always to be able to solve a quadratic
equation like $x^{2} - 2x + 4 = 0$; but there are
an endless number of other equations, for
example, $x^{3} - 2x + 4 = 0$, $x^{4} + x^{3} + 2 = 0$, and so
on without limit. Have we got to make a
\PageSep{111}
new science whenever a new equation appears?

Now, if this were the case, the whole of our
preceding investigations, though to some
minds they might be amusing, would in truth
be of very trifling importance. But the great
fact, which has made modern analysis possible,
is that, by the aid of this calculus of vectors,
every formula which arises can receive its
proper interpretation; and the ``unknown''
quantity in every equation can be shown to
indicate some vector. Thus the science is now
complete in itself as far as its fundamental
ideas are concerned. It was receiving its final
form about the same time as when the steam
engine was being perfected, and will remain
a great and powerful weapon for the achievement
of the victory of thought over things
when curious specimens of that machine
repose in museums in company with the
helmets and breastplates of a slightly earlier
epoch.
\PageSep{112}


\Chapter{IX}{Coordinate Geometry}

\First{The} methods and ideas of coordinate geometry
\index{Coordinate Geometry|EtSeq}%
have already been employed in the
previous chapters. It is now time for us to
consider them more closely for their own
sake; and in doing so we shall strengthen our
hold on other ideas to which we have attained.
In the present and succeeding chapters we
will go back to the idea of the positive and
negative real numbers and will ignore the
imaginaries which were introduced in the last
two chapters.

We have been perpetually using the idea
that, by taking two axes, $XOX'$ and~$YOY'$,
in a plane, any point~$P$ in that plane can be
determined in position by a pair of positive
or negative numbers $x$ and~$y$, where (\Chg{cf.}{\cf}\
\Fig[fig.]{13}) $x$~is the length~$OM$ and $y$~is the length~$PM$.
This conception, simple as it looks, is
the main idea of the great subject of coordinate
geometry. Its discovery marks a
momentous epoch in the history of mathematical
thought. It is due (as has been
\PageSep{113}
already said) to the philosopher Descartes,
\index{Descartes}%
and occurred to him as an important mathematical
method one morning as he lay in bed.
Philosophers, when they have possessed a
thorough knowledge of mathematics, have
been among those who have enriched the
\Figure{13}
science with some of its best ideas. On the
other hand it must be said that, with hardly
an exception, all the remarks on mathematics
made by those philosophers who have possessed
but a slight or hasty and late-acquired
knowledge of it are entirely worthless, being
either trivial or wrong. The fact is a curious
one; since the ultimate ideas of mathematics
\PageSep{114}
seem, after all, to be very simple, almost
childishly so, and to lie well within the
province of philosophical thought. Probably
their very simplicity is the cause of error; we
are not used to think about such simple
abstract things, and a long training is necessary
to secure even a partial immunity from
error as soon as we diverge from the beaten
track of thought.

The discovery of coordinate geometry, and
also that of projective geometry about the
same time, illustrate another fact which is
being continually verified in the history of
knowledge, namely, that some of the greatest
discoveries are to be made among the most
well-known topics. By the time that the
seventeenth century had arrived, geometry
had already been studied for over two thousand
years, even if we date its rise with the Greeks.
Euclid, taught in the University of Alexandria,
\index{Euclid}%
being born about 330~\BC; and he only
systematized and extended the work of a long
series of predecessors, some of them men of
genius. After him generation after generation
of mathematicians laboured at the improvement
of the subject. Nor did the
subject suffer from that fatal bar to progress,
namely, that its study was confined to a
narrow group of men of similar origin and
outlook---quite the contrary was the case;
by the seventeenth century it had passed
\PageSep{115}
through the minds of Egyptians and Greeks,
of Arabs and of Germans. And yet, after all
this labour devoted to it through so many
ages by such diverse minds its most important
secrets were yet to be discovered.

No one can have studied even the elements
of elementary geometry without feeling the
lack of some guiding method. Every proposition
has to be proved by a fresh display of ingenuity;
and a science for which this is true
lacks the great requisite of scientific thought,
namely, method. Now the especial point of
coordinate geometry is that for the first
time it introduced method. The remote
deductions of a mathematical science are not
of primary theoretical importance. The
science has not been perfected, until it consists
in essence of the exhibition of great allied
methods by which information, on any desired
topic which falls within its scope, can easily
be obtained. The growth of a science is not
primarily in bulk, but in ideas; and the more
the ideas grow, the fewer are the deductions
which it is worth while to write down. Unfortunately,
mathematics is always encumbered
by the repetition in text-books of
numberless subsidiary propositions, whose importance
has been lost by their absorption
into the role of particular cases of more
general truths---and, as we have already insisted,
generality is the soul of mathematics.
\PageSep{116}

Again, coordinate geometry illustrates
another feature of mathematics which has
already been pointed out, namely, that mathematical
sciences as they develop dovetail into
each other, and share the same ideas in common.
It is not too much to say that the
various branches of mathematics undergo a
perpetual process of generalization, and that
as they become generalized, they coalesce.
Here again the reason springs from the very
nature of the science, its generality, that is
to say, from the fact that the science deals
with the general truths which apply to all
things in virtue of their very existence as
things. In this connection the interest of coordinate
geometry lies in the fact that it
relates together geometry, which started as
the science of space, and algebra, which has
its origin in the science of number.

Let us now recall the main ideas of the two
sciences, and then see how they are related
by Descartes' method of coordinates. Take
\index{Descartes}%
algebra in the first place. We will not trouble
ourselves about the imaginaries and will
think merely of the real numbers with positive
or negative signs. The fundamental idea
is that of any number, the variable number,
which is denoted by a letter and not by any
definite numeral. We then proceed to the
consideration of correlations between variables.
For example, if $x$ and~$y$ are two variables,
\PageSep{117}
we may conceive them as correlated by
the equations $x + y = 1$, or by $x - y = 1$, or in
any one of an indefinite number of other ways.
This at once leads to the application of the
\index{Form, Algebraic}%
idea of algebraic form. We think, in fact, of
any correlation of some interesting type, thus
rising from the initial conception of variable
numbers to the secondary conception of
variable correlations of numbers. Thus we
generalize the correlation $x + y = 1$, into the
correlation $ax + by = c$. Here $a$~and $b$ and~$c$,
being letters, stand for any numbers and are
in fact themselves variables. But they are
the variables which determine the variable
correlation; and the correlation, when determined,
correlates the variable numbers $x$ and~$y$.
Variables, like $a$,~$b$, and~$c$ above, which
are used to determine the correlation are
called ``constants,'' or parameters. The use
\index{Constants}%
\index{Parameters}%
of the term ``constant'' in this connection
for what is really a variable may seem at first
sight to be odd; but it is really very natural.
For the mathematical investigation is concerned
with the relation between the variables
$x$ and~$y$, after $a$,~$b$,~$c$ are supposed to have been
determined. So in a sense, relatively to $x$
and~$y$, the ``constants'' $a$,~$b$, and~$c$ are constants.
Thus $ax + by = c$ stands for the general
example of a certain algebraic form, that is,
for a variable correlation belonging to a certain
class.
\PageSep{118}

Again we generalize $x^{2} + y^{2} = 1$ into $ax^{2} + by^{2} = c$,
or still further into $ax^{2} + 2hxy + by^{2} = c$,
or, still further, into $ax^{2} + 2hxy + by^{2} + 2gx + 2fy = c$.

Here again we are led to variable correlations
which are indicated by their various algebraic
forms.

Now let us turn to geometry. The name
of the science at once recalls to our minds
the thought of figures and diagrams exhibiting
triangles and rectangles and squares and
circles, all in special relations to each other.
The study of the simple properties of these
figures is the subject matter of elementary
geometry, as it is rightly presented to the
beginner. Yet a moment's thought will show
that this is not the true conception of the
subject. It may be right for a child to commence
his geometrical reasoning on shapes,
like triangles and squares, which he has cut
out with scissors. What, however, is a triangle?
It is a figure marked out and bounded
by three bits of three straight lines.

Now the boundary of spaces by bits of
lines is a very complicated idea, and not at
all one which gives any hope of exhibiting
the simple general conceptions which should
form the bones of the subject. We want
something more simple and more general. It
is this obsession with the wrong initial ideas---very
natural and good ideas for the creation
\PageSep{119}
of first thoughts on the subject---which was
the cause of the comparative sterility of the
study of the science during so many centuries.
Coordinate geometry, and Descartes its inventor,
must have the credit of disclosing the
true simple objects for geometrical thought.

In the place of a bit of a straight line, let
us think of the whole of a straight line
throughout its unending length in both directions.
This is the sort of general idea from
which to start our geometrical investigations.
The Greeks never seem to have found any
use for this conception which is now fundamental
in all modern geometrical thought.
Euclid always contemplates a straight line as
drawn between two definite points, and is
very careful to mention when it is to be produced
beyond this segment. He never thinks
of the line as an entity given once for all as a
whole. This careful definition and limitation,
so as to exclude an infinity not immediately
apparent to the senses, was very characteristic
of the Greeks in all their many
activities. It is enshrined in the difference
between Greek architecture and Gothic architecture,
and between the Greek religion and
the modern religion. The spire on a Gothic
cathedral and the importance of the unbounded
straight line in modern geometry
are both emblematic of the transformation of
the modern world.
\PageSep{120}

The straight line, considered as a whole,
is accordingly the root idea from which
modern geometry starts. But then other
sorts of lines occur to us, and we arrive at the
conception of the complete curve which at
every point of it exhibits some uniform characteristic,
just as the straight line exhibits
at all points the characteristic of straightness.
For example, there is the circle which
\index{Circle}%
at all points exhibits the characteristic of
being at a given distance from its centre, and
again there is the ellipse, which is an oval
\index{Ellipse}%
curve, such that the sum of the two distances
of any point on it from two fixed points, called
\index{Focus}%
its \emph{foci}, is constant for all points on the curve.
It is evident that a circle is merely a particular
case of an ellipse when the two foci are
superposed in the same point; for then the
sum of the two distances is merely twice the
radius of the circle. The ancients knew the
properties of the ellipse and the circle and, of
course, considered them as wholes. For example,
Euclid never starts with mere segments
(\ie,~bits) of circles, which are then prolonged.
He always considers the whole circle
as described. It is unfortunate that the
circle is not the true fundamental line in
geometry, so that his defective consideration
of the straight line might have been of less
consequence.

This general idea of a curve which at any
\PageSep{121}
\index{Locus|EtSeq}%
point of it exhibits some uniform property is
expressed in geometry by the term ``locus.''
A locus is the curve (or surface, if we do not
confine ourselves to a plane) formed by points,
all of which possess some given property.
To every property in relation to each other
which points can have, there corresponds
some locus, which consists of all the points
possessing the property. In investigating
the properties of a locus considered as a whole,
we consider \emph{any} point or points on the locus.
Thus in geometry we again meet with the
fundamental idea of the variable. Furthermore,
in classifying loci under such headings
as straight lines, circles, ellipses, etc., we again
find the idea of form.

Accordingly, as in algebra we are concerned
with variable numbers, correlations between
variable numbers, and the classification of
correlations into types by the idea of algebraic
form; so in geometry we are concerned with
variable points, variable points satisfying
some condition so as form to a locus, and the
classification of \emph{loci} into types by the idea of
conditions of the same form.

Now, the essence of coordinate geometry
is the identification of the algebraic correlation
with the geometrical locus. The point
on a plane is represented in algebra by its
two coordinates, $x$~and~$y$, and the condition
satisfied by any point on the locus is represented
\PageSep{122}
by the corresponding correlation
between $x$~and~$y$. Finally to correlations
expressible in some general algebraic form,
such as $ax + by = c$, there correspond loci of
some general type, whose geometrical conditions
are all of the same form. We
have thus arrived at a position where we
can effect a complete interchange in ideas
and results between the two sciences. Each
science throws light on the other, and itself
gains immeasurably in power. It is impossible
not to feel stirred at the thought
of the emotions of men at certain historic
moments of adventure and discovery---Columbus
\index{Columbus}%
when he first saw the Western
shore, Pizarro when he stared at the Pacific
\index{Pizarro}%
Ocean, Franklin when the electric spark came
\index{Franklin}%
from the string of his kite, Galileo when he
\index{Galileo}%
first turned his telescope to the heavens.
Such moments are also granted to students
in the abstract regions of thought, and high
among them must be placed the morning when
Descartes lay in bed and invented the method
\index{Descartes}%
of coordinate geometry.

When one has once grasped the idea of coordinate
geometry, the immediate question
which starts to the mind is, What sort of
loci correspond to the well-known algebraic
forms? For example, the simplest among
the general types of algebraic forms is $ax + by = c$.
The sort of locus which corresponds
\PageSep{123}
to this is a straight line, and conversely to
every straight line there corresponds an equation
of this form. It is fortunate that the
simplest among the geometrical loci should
correspond to the simplest among the algebraic
forms. Indeed, it is this general correspondence
of geometrical and algebraic simplicity
which gives to the whole subject its
power. It springs from the fact that the
connection between geometry and algebra is
not casual and artificial, but deep-seated and
essential. The equation which corresponds
to a locus is called the equation ``of'' (or
``to'') the locus. Some examples of equations
of straight lines will illustrate the subject.
\Figure[3.75in]{14}
\PageSep{124}

Consider $y - x = 0$; here the $a$,~$b$, and~$c$, of
the general form have been replaced by $-1$,~$1$,
and $0$ respectively. This line passes through
the ``origin,''~$O$, in the diagram and bisects
the angle~$XOY$. It is the line~$L'OL$ of the
diagram. The fact that it passes through the
origin,~$O$, is easily seen by observing that the
equation is satisfied by putting $x = 0$ and
$y = 0$ simultaneously, but $0$~and~$0$ are the coordinates
of~$O$. In fact it is easy to generalize
and to see by the same method that the
equation of any line through the origin is of
the form $ax + by = 0$. The locus of\Typo{}{ the} equation
$y + x = 0$ also passes through the origin and
bisects the angle~$X'OY$: it is the line~$L_{1}OL_{1}'$
of the diagram.

Consider $y - x = 1$: the corresponding locus
does not pass through the origin. We therefore
seek where it cuts the axes. It must cut
the axis of~$x$ at some point of coordinates
$x$~and~$0$. But putting $y = 0$ in the equation,
we get $x = -1$; so the coordinates of this
point~$(A)$ are $1$~and~$0$. Similarly the point~$(B)$
where the line cuts the axis~$OY$ are $0$~and~$1$.
The locus is the line~$AB$ in the figure and
is parallel to~$LOL'$. Similarly $y + x = 1$ is the
equation of line~$A_{1}B$ of the figure; and the
locus is parallel to~$L_{1}OL_{1}'$. It is easy to prove
the general theorem that two lines represented
by equations of the forms $ax + by = 0$ and
$ax + by = c$ are parallel.
\PageSep{125}

The group of loci which we next come upon
are sufficiently important to deserve a chapter
to themselves. But before going on to
them we will dwell a little longer on the main
ideas of the subject.

The position of any point~$P$ is determined
by arbitrarily choosing an origin,~$O$, two axes,
\index{Axes}%
$OX$~and~$OY$, at right-angles, and then by
noting its coordinates $x$~and~$y$, \ie\ $OM$ and~$PM$
(\cf\ \Fig[fig.]{13}). Also, as we have seen in the
last chapter, $P$~can be determined by the
``vector''~$OP$, where the idea of the vector
includes a determinate direction as well as a
determinate length. From an abstract
mathematical point of view the idea of an
arbitrary origin may appear artificial and
clumsy, and similarly for the arbitrarily
drawn axes, $OX$~and~$OY$. But in relation to
the application of mathematics to the event
of the Universe we are here symbolizing with
direct simplicity the most fundamental fact
respecting the outlook on the world afforded
to us by our senses. We each of us refer
our sensible perceptions of things to an origin
which we call ``here'': our location in a
particular part of space round which we
group the whole Universe is the essential fact
of our bodily existence. We can imagine
beings who observe all phenomena in all space
with an equal eye, unbiassed in favour of any
part. With us it is otherwise, a cat at our
\PageSep{126}
feet claims more attention than an earthquake
at Cape Horn, or than the destruction
of a world in the Milky Way. It is true that
in making a common stock of our knowledge
with our fellowmen, we have to waive something
of the strict egoism of our own individual
``here.'' We substitute ``nearly
here'' for ``here''; thus we measure miles
from the town hall of the nearest town, or
from the capital of the country. In measuring
the earth, men of science will put the
origin at the earth's centre; astronomers
\index{Origin}%
even rise to the extreme altruism of putting
their origin inside the sun. But, far as this
last origin may be, and even if we go further
to some convenient point amid the nearer
fixed stars, yet, compared to the immeasurable
infinities of space, it remains true that
our first procedure in exploring the Universe
is to fix upon an origin ``nearly here.''

Again the relation of the coordinates $OM$
and~$MP$ (\ie\ $x$~and~$y$) to the vector~$OP$ is an
instance of the famous parallelogram law, as
\index{Parallelogram Law}%
can easily be seen (\cf\ \Fig[fig.]{8}) by completing
the parallelogram~$OMPN$. The idea of the
``vector''~$OP$, that is, of a directed magnitude,
is the root-idea of physical science.
Any moving body has a certain magnitude
of velocity in a certain direction, that is to
say, its velocity is a directed magnitude, a
vector. Again a force has a certain magnitude
\PageSep{127}
and has a definite direction. Thus,
when in analytical geometry the ideas of the
``origin,'' of ``coordinates,'' and of ``vectors''
are introduced, we are studying the
abstract conceptions which correspond to the
fundamental facts of the physical world.
\PageSep{128}


\Chapter{X}{Conic Sections}

\First{When} the Greek geometers had exhausted,
\index{Conic Sections|EtSeq}%
as they thought, the more obvious and interesting
properties of figures made up of
straight lines and circles, they turned to
the study of other curves; and, with their
almost infallible instinct for hitting upon
things worth thinking about, they chiefly
devoted themselves to conic sections, that
is, to the curves in which planes would cut
the surfaces of circular cones. The man
who must have the credit of inventing the
study is Menaechmus (born 375~\BC\ and
\index{Menaechmus}%
died 325~\BC); he was a pupil of Plato
and one of the tutors of Alexander the
Great. Alexander, by the by, is a conspicuous
\index{Alexander the Great}%
example of the advantages of good
tuition, for another of his tutors was the
philosopher Aristotle. We may suspect that
\index{Aristotle}%
Alexander found Menaechmus rather a dull
teacher, for it is related that he asked for the
\PageSep{129}
proofs to be made shorter. It was to this
request that Menaechmus replied: ``In the
\index{Menaechmus}%
country there are private and even royal
roads, but in geometry there is only one road
for all.'' This reply no doubt was true
enough in the sense in which it would have
been immediately understood by Alexander.
But if Menaechmus thought that his proofs
could not be shortened, he was grievously
mistaken; and most modern mathematicians
would be horribly bored, if they were compelled
to study the Greek proofs of the properties
of conic sections. Nothing illustrates
better the gain in power which is obtained by
the introduction of relevant ideas into a
science than to observe the progressive
shortening of proofs which accompanies the
growth of richness in idea. There is a certain
type of mathematician who is always
rather impatient at delaying over the ideas
of a subject: he is anxious at once to get on
to the proofs of ``important'' problems. The
history of the science is entirely against him.
There are royal roads in science; but those
who first tread them are men of genius and
\index{Alexander the Great}%
not kings.

The way in which conic sections first presented
themselves to mathematicians was as
follows: think of a cone (\cf\ \Fig[fig.]{15}), whose
vertex (or point) is~$V$, standing on a circular
base~$STU$. For example, a conical shade to
\PageSep{130}
an electric light is often an example of such a
surface. Now let the ``generating'' lines
which pass through~$V$ and lie on the surface
be all produced backwards; the result is a
double cone, and $PQR$~is another circular cross
section on the opposite side of~$V$ to the cross
section~$STU$. The axis of the cone~$CVC'$
passes through all the centres of these circles
and is perpendicular to their planes, which
are parallel to each other. In the diagram
the parts of the curves which are supposed
to lie behind the plane of the paper are dotted
lines, and the parts on the plane or in front
of it are continuous lines. Now suppose this
double cone is cut by a plane not perpendicular
to the axis~$CVC'$, or at least not
necessarily perpendicular to it. Then three
cases can arise:---

(1) The plane may cut the cone in a closed
\index{Ellipse|EtSeq}%
oval curve, such as~$ABA'B'$ which lies entirely
on one of the two half-cones. In this
case the plane will not meet the other half-cone
at all. Such a curve is called an ellipse; it is
an oval curve. A particular case of such a
section of the cone is when the plane is perpendicular
to the axis~$CVC'$, then the section,
such as $STU$ or $PQR$, is a circle. Hence a
\index{Circle}%
circle is a particular case of the ellipse.

(2) The plane may be parallelled to a tangent
plane touching the cone along one of its ``generating''
lines as for example the plane of the
\PageSep{131}
\index{Parabola|EtSeq}%
curve $D_{1}A_{1}D_{1}'$ in the diagram is parallel to
the tangent plane touching the cone along the
generating line~$VS$; the curve is still confined
to one of the half-cones, but it is now not a
closed oval curve, it goes on endlessly as long
as the generating lines of the half-cone are
produced away from the vertex. Such a
conic section is called parabola.

(3) The plane may cut both the half-cones,
\index{Hyperbola|EtSeq}%
so that the complete curve consists of two
detached portions, or ``branches'' as they
are called, this case is illustrated by the two
branches $G_{2}A_{2}G_{2}'$ and $L_{2}A_{2}'L_{2}'$ which together
make up the curve. Neither branch is closed,
each of them spreading out endlessly as the
two half-cones are prolonged away from the
vertex. Such a conic section is called a
hyperbola.

There are accordingly three types of conic
sections, namely, ellipses, parabolas, and
hyperbolas. It is easy to see that, in a sense,
parabolas are limiting cases lying between
ellipses and hyperbolas. They form a more
special sort and have to satisfy a more particular
condition. These three names are
apparently due to Apollonius of Perga (born
\index{Apollonius of Perga}%
about 260~\BC, and died about 200~\BC), who
wrote a systematic treatise on conic sections
which remained the standard work till the
sixteenth century.
%[** TN: Moved to top of paragraph]
\Figure{15}

It must at once be apparent how awkward
\PageSep{132}
and difficult the investigation of the properties
of these curves must have been to the
Greek geometers. The curves are plane
curves, and yet their investigation involves
the drawing in perspective of a solid figure.
Thus in the diagram given above we have
practically drawn no subsidiary lines and yet
the figure is sufficiently complicated. The
\PageSep{133}
curves are plane curves, and it seems obvious
that we should be able to define them without
\Figure{16}
going beyond the plane into a solid figure.
At the same time, just as in the ``solid''
\Figure[2.5in]{17}
definition there is one uniform method of
definition---namely, the section of a cone by
\PageSep{134}
a plane---which yields three cases, so in any
``plane'' definition there also should be one
uniform method of procedure which falls into
three cases. Their shapes when drawn on
their planes are those of the curved lines in
the three figures \FigNum{16},~\FigNum{17}, and~\FigNum{18}. The
points $A$~and~$A'$ in the figures are called
%[** TN: Labels A, A', M, and line PM added to match the text]
\Figure[4in]{18}
the vertices and the line~$AA'$ the major axis.
It will be noted that a parabola (\cf\ \Fig[fig.]{17})
\index{Apollonius of Perga}%
\index{Vertex}%
has only one vertex. Apollonius proved\footnote
  {\Chg{Cf.}{\Cf}\ Ball, \Foreign{loc.\ cit.}, for this account of Apollonius and
  Pappus.}
that
the ratio of $PM^{2}$ to $AM·MA'$ $\left(\ie\ \dfrac{PM^{2}}{AM·\Typo{MA}{MA'}}\right)$
remains constant both for the ellipse and the
hyperbola (figs.\ \FigNum{16} and \FigNum{18}), and that the ratio
\PageSep{135}
of $PM^{2}$ to~$AM$ is constant for the parabola
of \Fig[fig.]{17}; and he bases most of his work
on this fact. We are evidently advancing
towards the desired uniform definition which
does not go out of the plane; but have not
yet quite attained to uniformity.

In the diagrams \FigNum{16} and~\FigNum{18}, two points, $S$
and~$S'$, will be seen marked, and in \Fig[diagram]{17}
one point,~$S$. These are the \emph{foci} of the curves,
and are points of the greatest importance.
Apollonius knew that for an ellipse the sum
of $SP$ and~$S'P$ (\ie\ $SP + S'P$) is constant as
$P$~moves on the curve, and is equal to~$AA'$.
Similarly for a hyperbola the difference $S'P - SP$
is constant, and equal to~$AA'$ when $P$~is
on one branch, and the difference $SP' - S'P'$
is constant and equal to~$AA'$ when $P'$~is on
the other branch. But no corresponding
point seemed to exist for the parabola.

Finally $500$~years later the last great Greek
geometer, Pappus of Alexandria, discovered
\index{Pappus}%
the final secret which completed this line of
thought. In the diagrams \FigNum{16} and~\FigNum{18} will be
seen two lines, $XN$~and~$X'N'$, and in \Fig[diagram]{17}
the single line,~$XN$. These are the directrices
of the curves, two each for the ellipse
and the hyperbola, and one for the parabola.
Each directrix corresponds to its nearer focus.
\index{Directrix}%
\index{Focus}%
The characteristic property of a focus,~$S$, and
its corresponding directrix,~$XN$, for any one
of the three types of curve, is that the ratio
\PageSep{136}
$SP$ to~$PN$ $\left(\ie\ \dfrac{SP}{PN}\right)$ is constant, where $PN$~is
the perpendicular on the directrix from~$P$,
and $P$~is any point on the curve. Here we
have finally found the desired property of the
curves which does not require us to leave
the plane, and is stated uniformly for all
three curves. For ellipses the ratio\footnote
  {\Chg{Cf.}{\Cf}\ Note~B, \Pageref{noteB}.\Pagelabel{136}}
is less
than~$1$, for parabolas it is equal to~$1$, and for
hyperbolas it is greater than~$1$.

When Pappus had finished his investigations,
\index{Pappus}%
he must have felt that, apart from
minor extensions, the subject was practically
exhausted; and if he could have foreseen
the history of science for more than a thousand
years, it would have confirmed his belief.
Yet in truth the really fruitful ideas in connection
with this branch of mathematics had
not yet been even touched on, and no one
had guessed their supremely important applications
in nature. No more impressive
warning can be given to those who would
confine knowledge and research to what is
apparently useful, than the reflection that
conic sections were studied for eighteen hundred
years merely as an abstract science,
without a thought of any utility other than
to satisfy the craving for knowledge on the
part of mathematicians, and that then at the
end of this long period of abstract study, they
\PageSep{137}
were found to be the necessary key with
which to attain the knowledge of one of the
most important laws of nature.

Meanwhile the entirely distinct study of
astronomy had been going forward. The
\index{Astronomy}%
great Greek astronomer Ptolemy (died 168~\AD)
\index{Ptolemy}%
published his standard treatise on the
subject in the University of Alexandria, explaining
the apparent motions among the
fixed stars of the sun and planets by the conception
of the earth at rest and the sun and
the planets circling round it. During the
next thirteen hundred years the number and
the accuracy of the astronomical observations
increased, with the result that the description
of the motions of the planets on
Ptolemy's hypothesis had to be made more
and more complicated. Copernicus (born
\index{Copernicus}%
1473~\AD\ and died 1543~\AD) pointed out
that the motions of these heavenly bodies
could be explained in a simpler manner if the
sun were supposed to rest, and the earth and
planets were conceived as moving round it.
However, he still thought of these motions as
essentially circular, though modified by a set
of small corrections arbitrarily superimposed
on the primary circular motions. So the
matter stood when Kepler was born at Stuttgart
\index{Kepler}%
in Germany in 1571~\AD. There were
two sciences, that of the geometry of conic
sections and that of astronomy, both of which
\PageSep{138}
had been studied from a remote antiquity
without a suspicion of any connection between
the two. Kepler was an astronomer,
\index{Kepler}%
but he was also an able geometer, and on the
subject of conic sections had arrived at ideas
in advance of his time\Add{.} He is only one of
many examples of the falsity of the idea that
success in scientific research demands an exclusive
absorption in one narrow line of study.
Novel ideas are more apt to spring from
an unusual assortment of knowledge---not
necessarily from vast knowledge, but from a
thorough conception of the methods and ideas
of distinct lines of thought. It will be remembered
that Charles Darwin was helped
\index{Darwin}%
to arrive at his conception of the law of
evolution by reading Malthus' famous \Title{Essay
\index{Malthus}%
on Population}, a work dealing with a different
subject---at least, as it was then
thought.

Kepler enunciated three laws of planetary
\index{Kepler's Laws}%
motion, the first two in~1609, and the third
ten years later. They are as follows:

(1) The orbits of the planets are ellipses,
the sun being in the focus.

(2) As a planet moves in its orbit, the
radius vector from the sun to the planet
sweeps out equal areas in equal times.

(3) The squares of the periodic times of the
several planets are proportional to the cubes
of their major axes.
\PageSep{139}

These laws proved to be only a stage towards
a more fundamental development of
ideas. Newton (born 1642~\AD\ and died
\index{Newton}%
1727~\AD) conceived the idea of universal
gravitation, namely, that any two pieces of
\index{Gravitation}%
matter attract each other with a force proportional
to the product of their masses and
inversely proportional to the square of their
distance from each other. This sweeping
general law, coupled with the three laws of
motion which he put into their final general
shape, proved adequate to explain all astronomical
phenomena, including Kepler's laws,
and has formed the basis of modern physics.
Among other things he proved that comets
might move in very elongated ellipses, or in
parabolas, or in hyperbolas, which are nearly
parabolas. The comets which return---such
as Halley's comet---must, of course, move in
\index{Halley}%
ellipses. But the essential step in the proof of
the law of gravitation, and even in the suggestion
of its initial conception, was the verification
of Kepler's laws connecting the
motions of the planets with the theory of
conic sections.

From the seventeenth century onwards the
abstract theory of the curves has shared in
the double renaissance of geometry due to
the introduction of coordinate geometry and
of projective geometry. In projective geometry
\index{Projective Geometry}%
the fundamental ideas cluster round
\PageSep{140}
the consideration of sets (or pencils, as they
\index{Pencils}%
are called) of lines passing through a common
point (the vertex of the ``pencil''). Now
(\cf\ \Fig[fig.]{19}) if $A$,~$B$, $C$,~$D$, be any four fixed
points on a conic section and $P$~be a variable
point on the curve, the pencil of lines $PA$,
\Figure[2.5in]{19}
$PB$, $PC$, and~$PD$, has a special property,
known as the constancy of its cross ratio. It
\index{Cross Ratio}%
will suffice here to say that cross ratio is a
fundamental idea in projective geometry.
For projective geometry this is really the definition
of the curves, or some analogous property
which is really equivalent to it. It
\PageSep{141}
will be seen how far in the course of ages of
study we have drifted away from the old
original idea of the sections of a circular cone.
We know now that the Greeks had got hold
of a minor property of comparatively slight
importance; though by some divine good
fortune the curves themselves deserved all
the attention which was paid to them. This
unimportance of the ``section'' idea is now
marked in ordinary mathematical phraseology
by dropping the word from their
names. As often as not, they are now
named merely ``conics'' instead of ``conic
sections.''

Finally, we come back to the point at
\index{Locus}%
which we left coordinate geometry in the last
chapter. We had asked what was the type
of \emph{loci} corresponding to the general algebraic
form $ax + by = c$, and had found that it was
the class of straight lines in the plane. We
had seen that every straight line possesses an
equation of this form, and that every equation
of this form corresponds to a straight line.
We now wish to go on to the next general
type of algebraic forms. This is evidently
to be obtained by introducing terms involving
$x^{2}$~and $xy$ and~$y^{2}$. Thus the new general
form must be written\Add{:}---
\[
ax^{2} + 2hxy + by^{2} + 2gx + 2fy + c = 0\Add{.}
\]
What does this represent? The answer is
\PageSep{142}
that (when it represents any locus) it always represents
a conic section, and, furthermore,
that the equation of every conic section can
always be put into this shape. The discrimination
of the particular sorts of conics as given
by this form of equation is very easy. It entirely
depends upon the consideration of $ab - h^{2}$,
where $a$,~$b$, and~$h$, are the ``constants'' as
written above. If $ab - h^{2}$ is a positive number,
the curve is an ellipse; if $ab - h^{2} = 0$, the curve
is a parabola: and if $ab - h^{2}$ is a negative
number, the curve is a hyperbola.

For example, put $a = b = 1$, $h = g = f = 0$,
$c = -4$. We then get the equation $x^{2} + y^{2} - 4 = 0$.
It is easy to prove that this is the equation
of a circle, whose centre is at the origin,
and radius is $2$~units of length. Now $ab - h^{2}$
becomes $1 × 1 - 0^{2}$, that is,~$1$, and is therefore
positive. Hence the circle is a particular
case of an ellipse, as it ought to be. Generalising,
the equation of any circle can be
put into the form $a(x^{2} + y^{2}) + 2gx + 2fy + c = 0$.
Hence $ab - h^{2}$ becomes $a^{2} - 0$, that is,~$a^{2}$,
which is necessarily positive. Accordingly
all circles satisfy the condition for ellipses.
The general form of the equation of a parabola
is
\[
(dx + ey)^{2} + 2gx + 2fy + c = 0,
\]
so that the terms of the second degree, as
\PageSep{143}
they are called, can be written as a perfect
square. For squaring out, we get
\[
d^{2} x^{2} + 2dexy + e^{2} y^{2} + 2gx + 2fy + c;
\]
so that by comparison $a = d^{2}$, $h = de$, $b = e^{2}$,
and therefore $ab - h^{2} = d^{2} e^{2} - (de)^{2} = 0$. Hence
the necessary condition is automatically satisfied.
The equation $2xy - 4 = 0$, where $a = b = g = f = 0$,
$h = 1$, $c = -4$, represents a hyperbola.
For the condition $ab - h^{2}$ becomes
$0 - 1^{2}$, that is,~$-1$, which is negative.

{\Loosen The limitation, introduced by saying that,
\index{Circular Cylinder}%
\emph{when the general equation represents any locus},
it represents a conic section, is necessary, because
some particular cases of the general
equation represent no real locus. For example
$x^{2} + y^{2} + 1 = 0$ can be satisfied by no
real values of $x$~and~$y$. It is usual to say that
the locus is now one composed of imaginary
points. But this idea of imaginary points in
geometry is really one of great complexity,
which we will not now enter into.}

Some exceptional cases are included in the
general form of the equation which may not
be immediately recognized as conic sections.
By properly choosing the constants the equation
can be made to represent two straight
lines. Now two intersecting straight lines
may fairly be said to come under the Greek
idea of a conic section. For, by referring to
\PageSep{144}
the picture of the double cone above, it will
be seen that some planes through the vertex,~$V$,
will cut the cone in a pair of straight lines
intersecting at~$V$. The case of two parallel
straight lines can be included by considering
a circular cylinder as a particular case of a
cone. Then a plane, which cuts it and is
parallel to its axis, will cut it in two parallel
straight lines. Anyhow, whether or no the
%[** TN: [sic] "Greek", not "Greeks"]
ancient Greek would have allowed these
special cases to be called conic sections, they
are certainly included among the curves represented
by the general algebraic form of
the second degree. This fact is worth noting;
for it is characteristic of modern mathematics
to include among general forms all sorts of
particular cases which would formerly have
received special treatment. This is due to
its pursuit of generality.
\PageSep{145}


\Chapter{XI}{Functions}

\First{The} mathematical use of the term function
%[** TN: Index entry reads "p. 144" in the original]
\index{Function|EtSeq}%
has been adopted also in common life. For
example, ``His temper is a function of his
digestion,'' uses the term exactly in this
mathematical sense. It means that a rule
can be assigned which will tell you what his
temper will be when you know how his
digestion is working. Thus the idea of a
``function'' is simple enough, we only have
to see how it is applied in mathematics to
variable numbers. Let us think first of some
concrete examples: If a train has been travelling
at the rate of twenty miles per hour, the
distance ($s$~miles) gone after any number of
hours, say~$t$, is given by $s = 20 × t$; and $s$~is
called a function of~$t$. Also $20 × t$ is the function
of~$t$ with which $s$~is identical. If John
is one year older than Thomas, then, when
Thomas is at any age of $x$~years, John's age
($y$~years) is given by $y = x + 1$; and $y$~is a
function of~$x$, namely, is the function~$x + 1$.

In these examples $t$ and~$x$ are called the
\PageSep{146}
\index{Argument of a Function}%
\index{Value of a Function}%
``arguments'' of the functions in which they
appear. Thus $t$~is the argument of the function
$20 × t$, and $x$~is the argument of the function
$x + 1$. If $s = 20 × t$, and $y = x + 1$, then $s$
and~$y$ are called the ``values'' of the functions
$20 × t$ and $x + 1$ respectively.

Coming now to the general case, we can
define a function in mathematics as a correlation
between two variable numbers, called
respectively the argument and the value of
the function, such that whatever value be
assigned to the ``argument of the function''
the ``value of the function'' is definitely
(\ie~uniquely) determined. The converse
is not necessarily true, namely, that when
the value of the function is determined
the argument is also uniquely determined.
Other functions of the argument~$x$ are $y = x^{2}$,
%[** TN: log, sin italicized throughout in the original]
$y = 2x^{2} + 3x + 1$, $y = x$, $y = \log x$, $y = \sin x$. The
last two functions of this group will be
readily recognizable by those who understand
a little algebra and trigonometry. It is not
worth while to delay now for their explanation,
as they are merely quoted for the sake
of example.

Up to this point, though we have defined
what we mean by a function in general, we
have only mentioned a series of special functions.
But mathematics, true to its general
methods of procedure, symbolizes the general
idea of any function. It does this by writing
\PageSep{147}
\index{Variable Function}%
$F(x)$, $f(x)$, $g(x)$, $\phi(x)$,~etc., for any function of~$x$,
where the argument~$x$ is placed in a bracket,
and some letter like $F$,~$f$, $g$, $\phi$,~etc., is prefixed
to the bracket to stand for the function.
This notation has its defects. Thus it obviously
clashes with the convention that the
single letters are to represent variable numbers;
since here $F$,~$f$, $g$, $\phi$,~etc., prefixed to a
bracket stand for variable functions. It
would be easy to give examples in which we
can only trust to common sense and the context
to see what is meant. One way of
evading the confusion is by using Greek
letters (\eg~$\phi$ as above) for functions; another
way is to keep to $f$~and~$F$ (the initial
letter of function) for the functional letter,
and, if other variable functions have to be
symbolized, to take an adjacent letter like~$g$.

With these explanations and cautions, we
write $y = f(x)$, to denote that $y$~is the value of
some undetermined function of the argument~$x$;
where $f(x)$ may stand for anything such
as $x + 1$, $x^{2} - 2x + 1$, $\sin x$, $\log x$, or merely for
$x$~itself. The essential point is that when $x$~is
given, then $y$~is thereby definitely determined.
It is important to be quite clear as
to the generality of this idea. Thus in $y = f(x)$,
we may determine, if we choose, $f(x)$~to
mean that when $x$~is an integer, $f(x)$~is zero,
and when $x$~has any other value, $f(x)$~is~$1$.
Accordingly, putting $y = f(x)$, with this choice
\PageSep{148}
for the meaning of~$f$, $y$~is either $0$ or~$1$ according
as the value of~$x$ is integral or otherwise.
Thus $f(1) = 0$, $f(2) = 0$, $f(\frac{2}{3}) = 1$, $f(\sqrt{2}) = 1$, and
so on. This choice for the meaning of~$f(x)$
gives a perfectly good function of the argument~$x$
according to the general definition of
a function.

A function, which after all is only a sort
\index{Graphs|EtSeq}%
of correlation between two variables, is represented
like other correlations by a graph,
that is in effect by the methods of coordinate
geometry. For example, \Fig[fig.]{2} in \ChapRef{II}.\
is the graph of the function~$\dfrac{1}{v}$ where $v$~is the
argument and $p$~the value of the function.
In this case the graph is only drawn for
positive values of~$v$, which are the only values
possessing any meaning for the physical application
considered in that chapter. Again
in \Fig[fig.]{14} of \ChapRef{IX}.\ the whole length of
the line~$AB$, unlimited in both directions, is
the graph of the function~$x + 1$, where $x$~is the
argument and $y$~is the value of the function;
and in the same figure the unlimited line~$A_{1}B$
is the graph of the function~$1 - x$, and
the line~$LOL'$ is the graph of the function~$x$,
$x$~being the argument and $y$~the value of the
function.

These functions, which are expressed by
simple algebraic formulæ, are adapted for representation
by graphs. But for some functions
\PageSep{149}
this representation would be very
misleading without a detailed explanation, or
might even be impossible. Thus, consider the
function mentioned above, which has the value~$1$
for all values of its argument~$x$, except
those which are integral, \eg\ except for $x = 0$,
$x = 1$, $x = 2$, etc., when it has the value~$0$.
Its appearance on a graph would be that of
the straight line~$ABA'$ drawn parallel to the
\Figure{20}
axis~$XOX'$ at a distance from it of $1$~unit of
length. But the points, $B$,~$C_{1}$, $C_{2}$, $C_{3}$, $C_{4}$,~etc.,
corresponding to the values $0$,~$1$, $2$, $3$, $4$,~etc., of
the argument~$x$, are to be omitted, and instead
of them the points $O$,~$B_{1}$, $B_{2}$, $B_{3}$, $B_{4}$,~etc.,
on the axis~$OX$, are to be taken. It is easy
to find functions for which the graphical representation
is not only inconvenient but
impossible. Functions which do not lend
themselves to graphs are important in the
\PageSep{150}
higher mathematics, but we need not concern
ourselves further about them here.

The most important division between functions
\index{Continuous Functions|EtSeq}%
\index{Discontinuous Functions|EtSeq}%
is that between continuous and discontinuous
functions. A function is continuous
when its value only alters gradually for
gradual alterations of the argument, and is
discontinuous when it can alter its value by
sudden jumps. Thus the two functions $x + 1$
and $1 - x$, whose graphs are depicted as
straight lines in \Fig[fig.]{14} of \ChapRef{IX}., are continuous
functions, and so is the function~$\dfrac{1}{v}$,
depicted in \ChapRef{II}., if we only think of
positive values of~$v$. But the function depicted
in \Fig[fig.]{20} of this chapter is discontinuous
since at the values $x = 1$, $x = 2$, etc., of its
argument, its value gives sudden jumps.

Let us think of some examples of functions
presented to us in nature, so as to get into
our heads the real bearing of continuity and
discontinuity. Consider a train in its journey
along a railway line, say from Euston Station,
the terminus in London of the London and
North-Western Railway. Along the line in
order lie the stations of Bletchley and Rugby.
Let $t$~be the number of hours which the train
has been on its journey from Euston, and $s$~be
the number of miles passed over. Then $s$~is
a function of~$t$, \ie~is the variable value
corresponding to the variable argument~$t$.
\PageSep{151}
If we know the circumstances of the train's
run, we know~$s$ as soon as any special value
of~$t$ is given. Now, miracles apart, we may
confidently assume that $s$~is a continuous
function of~$t$. It is impossible to allow for
the contingency that we can trace the train
continuously from Euston to Bletchley, and
that then, without any intervening time, however
short, it should appear at Rugby. The
idea is too fantastic to enter into our calculation:
it contemplates possibilities not to be
found outside the \Title{Arabian Nights}; and even
in those tales sheer discontinuity of motion
hardly enters into the imagination, they do
not dare to tax our credulity with anything
more than very unusual speed. But unusual
speed is no contradiction to the great law of
continuity of motion which appears to hold
in nature. Thus light moves at the rate of
about $190,000$ miles per~second and comes to
us from the sun in seven or eight minutes;
but, in spite of this speed, its distance travelled
is always a continuous function of the time.

It is not quite so obvious to us that the
velocity of a body is invariably a continuous
function of the time. Consider the train at
any time~$t$: it is moving with some definite
velocity, say $v$~miles per~hour, where $v$~is
zero when the train is at rest in a station and
is negative when the train is backing. Now
we readily allow that $v$~cannot change its
\PageSep{152}
value suddenly for a big, heavy train. The
train certainly cannot be running at forty
miles per hour from 11.45~a.m.\ up to noon,
and then suddenly, without any lapse of time,
commence running at $50$~miles per~hour. We
at once admit that the change of velocity
will be a gradual process. But how about
sudden blows of adequate magnitude? Suppose
two trains collide; or, to take smaller
objects, suppose a man kicks a football. It
certainly appears to our sense as though the
football began suddenly to move. Thus, in
the case of velocity our senses do not revolt
at the idea of its being a discontinuous function
of the time, as they did at the idea of the
train being instantaneously transported from
Bletchley to Rugby. As a matter of fact,
if the laws of motion, with their conception
of mass, are true, there is no such thing as
discontinuous velocity in nature. Anything
that appears to our senses as discontinuous
change of velocity must, according to them,
be considered to be a case of gradual change
which is too quick to be perceptible to us.
It would be rash, however, to rush into the
generalization that no discontinuous functions
are presented to us in nature. A man who,
trusting that the mean height of the land
above sea-level between London and Paris
was a continuous function of the distance
from London, walked at night on Shakespeare's
\PageSep{153}
Cliff by Dover in contemplation of
the Milky Way, would be dead before he had
had time to rearrange his ideas as to the
necessity of caution in scientific conclusions.

It is very easy to find a discontinuous
function, even if we confine ourselves to the
\Figure{21}
simplest of the algebraic formulæ. For example,
take the function $y = \dfrac{1}{x}$, which we
have already considered in the form $p = \dfrac{1}{v}$,
where $v$~was confined to positive values. But
\PageSep{154}
now let $x$ have any value, positive or negative.
The graph of the function is exhibited in \Fig[fig.]{21}.
Suppose $x$ to change continuously from
a large negative value through a numerically
decreasing set of negative values up to~$0$, and
thence through the series of increasing positive
values. Accordingly, if a moving point,~$M$,
represents~$x$ on~$XOX'$, $M$~starts at the
extreme left of the axis~$XOX'$ and successively
moves through $M_{1}$,~$M_{2}$, $M_{3}$, $M_{4}$,~etc.
The corresponding points on the function are
$P_{1}$,~$P_{2}$, $P_{3}$, $P_{4}$,~etc. It is easy to see that
there is a point of discontinuity at $x = 0$, \ie~at
the origin~$O$. For the value of the function
on the negative (left) side of the origin becomes
endlessly great, but negative, and the
function reappears on the positive (right)
side as endlessly great but positive. Hence,
however small we take the length~$M_{2} M_{3}$,
there is a finite jump between the values of
the function at $M_{2}$ and~$M_{3}$. Indeed, this case
has the peculiarity that the smaller we take the
length between $M_{2}$ and~$M_{3}$, so long as they
enclose the origin, the bigger is the jump in
value of the function between them. This
graph brings out, what is also apparent in
\Fig[fig.]{20} of this chapter, that for many functions
the discontinuities only occur at isolated
points, so that by restricting the values of the
argument we obtain a continuous function for
these remaining values. Thus it is evident
\PageSep{155}
from \Fig[fig.]{21} that in $y = \dfrac{1}{x}$, if we keep to positive
values only and exclude the origin, we obtain
a continuous function. Similarly the same
function, if we keep to negative values only,
excluding the origin, is continuous. Again
the function which is graphed in \Fig[fig.]{20} is continuous
between $B$ and~$C_{1}$, and between $C_{1}$
and~$C_{2}$, and between $C_{2}$ and $C_{3}$, and so on,
always in each case excluding the end points.
It is, however, easy to find functions such that
their discontinuities occur at all points. For
example, consider a function~$f(x)$, such that
when $x$~is any fractional number $f(x) = 1$, and
when $x$~is any incommensurable number
$f(x) = 2$. This function is discontinuous at all
points.

Finally, we will look a little more closely
at the definition of continuity given above.
We have said that a function is continuous
when its value only alters gradually for
gradual alterations of the argument, and is
discontinuous when it can alter its value by
sudden jumps. This is exactly the sort of
definition which satisfied our mathematical
forefathers and no longer satisfies modern
mathematicians. It is worth while to spend
some time over it; for when we understand
the modern objections to it, we shall have
gone a long way towards the understanding
of the spirit of modern mathematics. The
\PageSep{156}
whole difference between the older and the
newer mathematics lies in the fact that vague
half-metaphorical terms like ``gradually''
are no longer tolerated in its exact statements.
Modern mathematics will only admit statements
and definitions and arguments which
exclusively employ the few simple ideas about
number and magnitude and variables on
which the science is founded. Of two numbers
one can be greater or less than the
other; and one can be such and such a multiple
of the other; but there is no relation of
``graduality'' between two numbers, and
hence the term is inadmissible. Now this
may seem at first sight to be great pedantry.
To this charge there are two answers. In
the first place, during the first half of the
nineteenth century it was found by some
great mathematicians, especially Abel in
\index{Abel}%
Sweden, and Weierstrass in Germany, that
\index{Weierstrass}%
large parts of mathematics as enunciated in
the old happy-go-lucky manner were simply
wrong. Macaulay in his essay on Bacon
\index{Bacon}%
\index{Macaulay}%
contrasts the certainty of mathematics with
the uncertainty of philosophy; and by way
of a rhetorical example he says, ``There has
been no reaction against Taylor's theorem.''
\index{Taylor's Theorem}%
He could not have chosen a worse example.
For, without having made an examination of
English text-books on mathematics contemporary
with the publication of this essay, the
\PageSep{157}
\index{Taylor's Theorem}%
assumption is a fairly safe one that Taylor's
theorem was enunciated and proved wrongly
in every one of them. Accordingly, the
anxious precision of modern mathematics is
necessary for accuracy. In the second place
it is necessary for research. It makes for
clearness of thought, and thence for boldness
of thought and for fertility in trying new
combinations of ideas. When the initial
statements are vague and slipshod, at every
subsequent stage of thought common sense
has to step in to limit applications and to
explain meanings. Now in creative thought
common sense is a bad master. Its sole
criterion for judgment is that the new ideas
shall look like the old ones. In other words
it can only act by suppressing originality.

In working our way towards the precise
definition of continuity (as applied to functions)
let us consider more closely the statement
that there is no relation of ``graduality''
between numbers. It may be asked, Cannot
one number be only slightly greater than
another number, or in other words, cannot
the difference between the two numbers be
small? The whole point is that in the abstract,
apart from some arbitrarily assumed
application, there is no such thing as a great
or a small number. A million miles is a
small number of miles for an astronomer
investigating the fixed stars, but a million
\PageSep{158}
pounds is a large yearly income. Again, one-quarter
is a large fraction of one's income to
give away in charity, but is a small fraction
of it to retain for private use. Examples can
be accumulated indefinitely to show that
great or small in any absolute sense have no
abstract application to numbers. We can
say of two numbers that one is greater or
smaller than another, but not without specification
of particular circumstances that any
one number is great or small. Our task
therefore is to define continuity without any
mention of a ``small'' or ``gradual'' change
in value of the function.

In order to do this we will give names to
some ideas, which will also be useful when
we come to consider limits and the differential
calculus.

An ``interval'' of values of the argument~$x$
\index{Interval|EtSeq}%
of a function~$f(x)$ is all the values lying
between some two values of the argument.
For example, the interval between $x = 1$ and
$x = 2$ consists of all the values which~$x$ can
take lying between $1$ and~$2$, \ie\ it consists of
all the real numbers between $1$ and~$2$. But
the bounding numbers of an interval need
not be integers. An interval of values of the
argument \emph{contains} a number~$a$, when $a$~is a
member of the interval. For example, the
interval between $1$ and~$2$ contains $\frac{3}{2}$, $\frac{5}{3}$, $\frac{7}{4}$, and
so on.
\PageSep{159}

A set of numbers approximates to a number~$a$
\index{Standard of Approximation|EtSeq}%
within a \emph{standard}~$k$, when the numerical
difference between $a$ and every number of the
set is less than~$k$. Here $k$~is the ``standard
of approximation.'' Thus the set of numbers
$3$,~$4$, $6$,~$8$, approximates to the number~$5$
within the standard~$4$. In this case the
standard~$4$ is not the smallest which could
have been chosen, the set also approximates
%[** TN: Original uses center dot for decimal point]
to~$5$ within any of the standards $3.1$ or $3.01$
or~$3.001$. Again, the numbers, $3.1$, $3.141$,
$3.1415$, $3.14159$ approximate to $3.13102$ within
the standard~$.032$, and also within the
smaller standard~$.03103$.

These two ideas of an interval and of
\index{Neighbourhood|EtSeq}%
approximation to a number within a standard
are easy enough; their only difficulty is that
they look rather trivial. But when combined
with the next idea, that of the ``neighbourhood''
of a number, they form the foundation
of modern mathematical reasoning. What
do we mean by saying that something is true
for a function~$f(x)$ in the neighbourhood of
the value~$a$ of the argument~$x$? It is this
fundamental notion which we have now got to
make precise.

The values of a function~$f(x)$ are said to
possess a characteristic in the ``neighbourhood
of~$a$'' when some interval can be found,
which (i)~contains the number~$a$ not as an
end-point, and (ii)~is such that every value
\PageSep{160}
of the function for arguments, other than~$a$,
lying within that interval possesses the characteristic.
The value~$f(a)$ of the function for
the argument~$a$ may or may not possess the
characteristic. Nothing is decided on this
point by statements about the \emph{neighbourhood}
of~$a$.

For example, suppose we take the particular
function~$x^{2}$. Now \emph{in the neighbourhood of~$2$},
the values of~$x^{2}$ are less than~$5$. For we can
find an interval, \eg\ from $1$ to~$2.1$, which
(i)~contains $2$ not as an end-point, and (ii)~is
such that, for values of~$x$ lying within it, $x^{2}$~is
less than~$5$.

Now, combining the preceding ideas we
know what is meant by saying that \emph{in the
neighbourhood of~$a$} the function~$f(x)$ approximates
to~$c$ within the \emph{standard}~$k$. It means
that some interval can be found which (i)~includes
$a$ not as an end-point, and (ii)~is such
that all values of~$f(x)$, where $x$~lies in the interval
and is not~$a$, differ from~$c$ by less than~$k$. For
example, in the neighbourhood of~$2$, the function~$\sqrt{x}$
approximates to~$1.41425$ within the
standard~$.0001$. This is true because the
square root of~$1.99996164$ is~$1.4142$ and the
square root of~$2.00024449$ is~$1.4143$; hence
for values of~$x$ lying in the interval
$1.99996164$ to~$2.00024449$, which contains $2$
not as an end-point, the values of the function~$\sqrt{x}$
all lie between $1.4142$ and $1.4143$, and
\PageSep{161}
they therefore all differ from~$1.41425$ by less
than~$.0001$. In this case we can, if we like,
fix a smaller standard of approximation,
namely $.000051$ or $.0000501$. Again, to take
another example, in the neighbourhood of~$2$
the function~$x^{2}$ approximates to~$4$ within the
standard~$.5$. For $(1.9)^{2} = 3.61$ and $(2.1)^{2} = 4.41$,
and thus the required interval $1.9$ to~$2.1$,
containing $2$ not as an end-point, has
been found. This example brings out the
fact that statements about a function~$f(x)$ in
the neighbourhood of a number~$a$ are distinct
from statements about the value of~$f(x)$ when
$x = a$. The production of an \emph{interval}, throughout
which the statement is true, is required.
Thus the mere fact that $2^{2} = 4$ does not by
itself justify us in saying that in the \emph{neighbourhood}
of~$2$ the function~$x^{2}$ is equal to~$4$.
This statement would be untrue, because no
interval can be produced with the required
property. Also, the fact that $2^{2} = 4$ does not
by itself justify us in saying that in the
\emph{neighbourhood} of~$2$ the function~$x^{2}$ approximates
to~$4$ within the standard~$.5$; although
as a matter of fact, the statement has just
been proved to be true.

If we understand the preceding ideas, we
understand the foundations of modern
mathematics. We shall recur to analogous
ideas in the chapter on Series, and again
in the chapter on the Differential Calculus.
\PageSep{162}
\index{Continuous Functions@Continuous Functions (\emph{defined})}%
Meanwhile, we are now prepared to define
``continuous functions.'' A function~$f(x)$
is ``continuous'' at a value~$a$ of its argument,
when in the neighbourhood of~$a$
its values approximate to~$f(a)$ (\ie~to its
value at~$a$) within \emph{every} standard of approximation.

This means that, whatever standard~$k$ be
chosen, in the neighbourhood of~$a$ $f(x)$~approximates
to~$f(a)$ within the standard~$k$.
For example, $x^{2}$~is continuous at the value~$2$
of its argument,~$x$, because however $k$~be
chosen we can always find an interval, which
(i)~contains $2$ not as an end-point, and (ii)~is
such that the values of~$x^{2}$ for arguments lying
within it approximate to~$4$ (\ie~$2^{2}$) within
the standard~$k$. Thus, suppose we choose
the standard~$.1$; now $(1.999)^{2} = 3.996001$,
and $(2.01)^{2} = 4.0401$, and both these numbers
differ from~$4$ by less than~$.1$. Hence, within
the interval $1.999$ to $2.01$ the values of~$x^{2}$
approximate to~$4$ within the standard~$.1$.
Similarly an interval can be produced for any
other standard which we like to try.

Take the example of the railway train. Its
velocity is continuous as it passes the signal
box, if whatever velocity you like to assign
(say one-millionth of a mile per hour) an interval
of time can be found extending before
and after the instant of passing, such that at
all instants within it the train's velocity
\PageSep{163}
differs from that with which the train passed
the box by less than one-millionth of a mile
per hour; and the same is true whatever
other velocity be mentioned in the place of
one-millionth of a mile per hour.
\PageSep{164}


\Chapter{XII}{Periodicity in Nature}

\First{The} whole life of Nature is dominated by
\index{Periodicity|EtSeq}%
the existence of periodic events, that is, by
the existence of successive events so analogous
to each other that, without any straining of
language, they may be termed recurrences of
the same event. The rotation of the earth
produces the successive days. It is true that
each day is different from the preceding days,
however abstractly we define the meaning of
a day, so as to exclude casual phenomena.
But with a sufficiently abstract definition of
a day, the distinction in properties between
two days becomes faint and remote from
practical interest; and each day may then
be conceived as a recurrence of the phenomenon
of one rotation of the earth. Again the
path of the earth round the sun leads to the
yearly recurrence of the seasons, and imposes
another periodicity on all the operations of
nature. Another less fundamental periodicity
is provided by the phases of the moon.
In modern civilized life, with its artificial light,
these phases are of slight importance, but in
\PageSep{165}
ancient times, in climates where the days are
burning and the skies clear, human life was
apparently largely influenced by the existence of
moonlight. Accordingly our divisions into
weeks and months, with their religious associations,
have spread over the European races from
Syria and Mesopotamia, though independent
observances following the moon's phases are
found amongst most nations. It is, however,
through the tides, and not through its phases
of light and darkness, that the moon's periodicity
has chiefly influenced the history of
the earth.

Our bodily life is essentially periodic.
It is dominated by the beatings of the
heart, and the recurrence of breathing.
The presupposition of periodicity is indeed
fundamental to our very conception of life.
We cannot imagine a course of nature in
which, as events progressed, we should be
unable to say: ``This has happened before.''
The whole conception of experience as a guide
to conduct would be absent. Men would
always find themselves in new situations
possessing no substratum of identity with
anything in past history. The very means of
measuring time as a quantity would be absent.
Events might still be recognized as occurring
in a series, so that some were earlier and
others later. But we now go beyond this
bare recognition. We can not only say that
\PageSep{166}
\index{Time|EtSeq}%
three events, $A$,~$B$,~$C$, occurred in this order,
so that $A$~came before~$B$, and $B$~before~$C$;
but also we can say that the length of time
between the occurrences of $A$ and~$B$ was
twice as long as that between $B$ and~$C$. Now,
quantity of time is essentially dependent on
observing the number of natural recurrences
which have intervened. We may say
that the length of time between $A$ and~$B$ was
so many days, or so many months, or so
many years, according to the type of recurrence
to which we wish to appeal. Indeed,
at the beginning of civilization, these three
modes of measuring time were really distinct.
It has been one of the first tasks of science
among civilized or semi-civilized nations, to
fuse them into one coherent measure. The
full extent of this task must be grasped. It
is necessary to determine, not merely what
number of days (\eg~$365.25$\dots) go to some
one year, but also previously to determine that
the same number of days do go to the successive
years. We can imagine a world in
which periodicities exist, but such that no two
are coherent. In some years there might be
$200$~days and in others~$350$. The determination
of the broad general consistency of the
more important periodicities was the first step
in natural science. This consistency arises
from no abstract intuitive law of thought;
it is merely an observed fact of nature
\PageSep{167}
guaranteed by experience. Indeed, so far is
it from being a necessary law, that it is not
even exactly true There are divergencies in
every case. For some instances these divergencies
are easily observed and are therefore
immediately apparent. In other cases it requires
the most refined observations and
astronomical accuracy to make them apparent.
Broadly speaking, all recurrences depending
on living beings, such as the beatings
of the heart, are subject in comparison with
other recurrences to rapid variations. The
great stable obvious recurrences---stable in
the sense of mutually agreeing with great
accuracy---are those depending on the motion
of the earth as a whole, and on similar motions
of the heavenly bodies.

We therefore assume that these astronomical
\index{Laws of Motion|EtSeq}%
recurrences mark out equal intervals of
time. But how are we to deal with their
discrepancies which the refined observations
of astronomy detect? Apparently we are
reduced to the arbitrary assumption that one
or other of these sets of phenomena marks out
equal times---\eg\ that either all days are of
equal length, or that all years are of equal
length. This is not so: some assumptions
must be made, but the assumption which
underlies the whole procedure of the astronomers
in determining the measure of time is
that the laws of motion are exactly verified.
\PageSep{168}
Before explaining how this is done, it is interesting
to observe that this relegation of
the determination of the measure of time to
the astronomers arises (as has been said) from
the stable consistency of the recurrences with
which they deal. If such a superior consistency
had been noted among the recurrences
characteristic of the human body, we
should naturally have looked to the doctors
of medicine for the regulation of our clocks.

In considering how the laws of motion
come into the matter, note that two inconsistent
modes of measuring time will yield
different variations of velocity to the same
body. For example, suppose we define an
hour as one twenty-fourth of a day, and take
the case of a train running uniformly for two
hours at the rate of twenty miles per hour.
Now take a grossly inconsistent measure of
time, and suppose that it makes the first hour
to be twice as long as the second hour. Then,
according to this other measure of duration,
the time of the train's run is divided into
two parts, during each of which it has traversed
the same distance, namely, twenty
miles; but the duration of the first part is
twice as long as that of the second part.
Hence the velocity of the train has not been
uniform, and on the average the velocity
during the second period is twice that during
the first period. Thus the question as to
\PageSep{169}
whether the train has been running uniformly
or not entirely depends on the standard of
time which we adopt.

Now, for all ordinary purposes of life on the
earth, the various astronomical recurrences
may be looked on as absolutely consistent;
and, furthermore assuming their consistency,
and thereby assuming the velocities and
changes of velocities possessed by bodies, we
find that the laws of motion, which have
been considered above, are almost exactly
verified. But only \emph{almost} exactly when we
come to some of the astronomical phenomena.
We find, however, that by assuming slightly
different velocities for the rotations and
motions of the planets and stars, the laws
would be exactly verified. This assumption
is then made; and we have, in fact thereby,
adopted a measure of time, which is indeed
defined by reference to the astronomical
phenomena, but not so as to be consistent
with the uniformity of any one of them. But
the broad fact remains that the uniform flow
of time on which so much is based, is itself
dependent on the observation of periodic
events.

Even phenomena, which on the surface
seem casual and exceptional, or, on the other
hand, maintain themselves with a uniform
persistency, may be due to the remote influence
of periodicity. Take for example, the
\PageSep{170}
principle of resonance. Resonance arises
\index{Resonance}%
when two sets of connected circumstances
have the same periodicities. It is a dynamical
law that the small vibrations of all bodies
when left to themselves take place in definite
times characteristic of the body. Thus a
pendulum with a small swing always vibrates
in some definite time, characteristic of its shape
and distribution of weight and length. A more
complicated body may have many ways of
vibrating; but each of its modes of vibration
will have its own peculiar ``period.'' Those
\index{Period}%
periods of vibration of a body are called its
``free'' periods. Thus a pendulum has but
one period of vibration, while a suspension
bridge will have many. We get a musical
instrument, like a violin string, when the
periods of vibration are all simple submultiples
of the longest; \ie~if $t$~seconds be the longest
period, the others are $\frac{1}{2}t$, $\frac{1}{3}t$, and so on, where
any of these smaller periods may be absent.
Now, suppose we excite the vibrations of a
body by a cause which is itself periodic;
then, if the period of the cause is very nearly
that of one of the periods of the body, that
mode of vibration of the body is very violently
excited; even although the magnitude of the
exciting cause is small. This phenomenon is
called ``resonance.'' The general reason is
easy to understand. Any one wanting to
upset a rocking stone will push ``in tune''
\PageSep{171}
with the oscillations of the stone, so as always
to secure a favourable moment for a push.
If the pushes are out of tune, some increase
the oscillations, but others check them. But
when they are in tune, after a time all the
pushes are favourable. The word ``resonance''
\index{Resonance}%
comes from considerations of sound:
but the phenomenon extends far beyond the
region of sound. The laws of absorption and
emission of light depend on it, the ``tuning''
of receivers for wireless telegraphy, the comparative
importance of the influences of
planets on each other's motion, the danger
to a suspension bridge as troops march over
it in step, and the excessive vibration of some
ships under the rhythmical beat of their
machinery at certain speeds. This coincidence
of periodicities may produce steady
phenomena when there is a constant association
of the two periodic events, or it may
produce violent and sudden outbursts when
the association is fortuitous and temporary.

Again, the characteristic and constant
periods of vibration mentioned above are
the underlying causes of what appear to
us as steady excitements of our senses. We
work for hours in a steady light, or we listen
to a steady unvarying sound. But, if modern
science be correct, this steadiness has no
counterpart in nature. The steady light is
due to the impact on the eye of a countless
\PageSep{172}
number of periodic waves in a vibrating ether,
and the steady sound to similar waves in a
vibrating air. It is not our purpose here to
explain the theory of light or the theory of
sound. We have said enough to make it
evident that one of the first steps necessary
to make mathematics a fit instrument for the
investigation of Nature is that it should be
able to express the essential periodicity of
things. If we have grasped this, we can
understand the importance of the mathematical
conceptions which we have next to
consider, namely, periodic functions.
\PageSep{173}


\Chapter{XIII}{Trigonometry}

\First{Trigonometry} did not take its rise from
\index{Trigonometry|EtSeq}%
the general consideration of the periodicity of
nature. In this respect its history is analogous
to that of conic sections, which also had
their origin in very particular ideas. Indeed,
a comparison of the histories of the two
sciences yields some very instructive analogies
and contrasts. Trigonometry, like conic sections,
had its origin among the Greeks. Its
inventor was Hipparchus (born about 160~\BC),
\index{Hipparchus}%
a Greek astronomer, who made his
observations at Rhodes. His services to
astronomy were very great, and it left his
\index{Astronomy}%
hands a truly scientific subject with important
results established, and the right method of
progress indicated. Perhaps the invention
of trigonometry was not the least of these
services to the main science of his study. The
next man who extended trigonometry was
Ptolemy, the great Alexandrian astronomer,
\index{Ptolemy}%
whom we have already mentioned. We now
\PageSep{174}
see at once the great contrast between conic
sections and trigonometry. The origin of
trigonometry was practical; it was invented
because it was necessary for astronomical research.
The origin of conic sections was
purely theoretical. The only reason for its
initial study was the abstract interest of the
ideas involved. Characteristically enough
conic sections were invented about $150$~years
earlier than trigonometry, during the very
best period of Greek thought. But the importance
of trigonometry, both to the theory
and the application of mathematics, is only
one of innumerable instances of the fruitful
ideas which the general science has gained
from its practical applications.

We will try and make clear to ourselves
what trigonometry is, and why it should be
generated by the scientific study of astronomy.
\index{Astronomy}%
In the first place: What are the measurements
which can be made by an astronomer?
They are measurements of time and measurements
of angles. The astronomer may adjust
a telescope (for it is easier to discuss the
familiar instrument of modern astronomers)
so that it can only turn about a fixed axis
pointing east and west; the result is that
the telescope can only point to the south, with
a greater or less elevation of direction, or, if
turned round beyond the zenith, point to the
north. This is the transit instrument, the
\PageSep{175}
great instrument for the exact measurement
of the times at which stars are due south or
due north. But indirectly this instrument
measures angles. For when the time elapsed
between the transits of two stars has been
noted, by the assumption of the uniform
rotation of the earth, we obtain the angle
through which the earth has turned in that
period of time. Again, by other instruments,
the angle between two stars can be directly
measured. For if $E$~is the eye of the astronomer,
\Figure[2in]{22}
and $EA$~and $EB$ are the directions in
which the stars are seen, it is easy to devise
instruments which shall measure the angle~$AEB$.
Hence, when the astronomer is forming
a survey of the heavens, he is, in fact,
measuring angles so as to fix the relative
directions of the stars and planets at any instant.
Again, in the analogous problem of
\PageSep{176}
\index{Surveys|EtSeq}%
\index{Triangle|EtSeq}%
land-surveying, angles are the chief subject
of measurements. The direct measurements
of length are only rarely possible with any
accuracy; rivers, houses, forests, mountains,
and general irregularities of ground all get in
the way. The survey of a whole country will
depend only on one or two direct measurements
of length, made with the greatest
elaboration in selected places like Salisbury
Plain. The main work of a survey is the
measurement of angles. For example, $A$,~$B$,
and~$C$ will be conspicuous points in the district
\Figure[2in]{23}
surveyed, say the tops of church towers.
These points are visible each from the others.
Then it is a very simple matter at~$A$ to
measure the angle~$BAC$, and at~$B$ to measure
the angle~$ABC$, and at~$C$ to measure the angle~$BCA$.
Theoretically, it is only necessary to
measure two of these angles; for, by a well-known
proposition in geometry, the sum of
the three angles of a triangle amounts to two
\PageSep{177}
right-angles, so that when two of the angles
are known, the third can be deduced. It is
better, however, in practice to measure all
three, and then any small errors of observation
can be checked. In the process of map-making
a country is completely covered with
triangles in this way. This process is called
triangulation, and is the fundamental process
\index{Triangulation}%
in a survey.

Now, when all the angles of a triangle are
\index{Similarity|EtSeq}%
known, the shape of the triangle is known---that
is, the shape as distinguished from the
size. We here come upon the great principle
of geometrical similarity. The idea is very
familiar to us in its practical applications.
We are all familiar with the idea of a plan
drawn to scale. Thus if the scale of a plan
be an inch to a yard, a length of three inches
in the plan means a length of three yards in
the original. Also the shapes depicted in the
plan are the shapes in the original, so that a
right-angle in the original appears as a right-angle
in the plan. Similarly in a map, which
is only a plan of a country, the proportions
of the lengths in the map are the proportions
of the distances between the places indicated,
and the directions in the map are the directions
in the country. For example, if in the
map one place is north-north-west of the
other, so it is in reality; that is to say, in a
map the angles are the same as in reality.
\PageSep{178}
\index{Scale of a Map}%
Geometrical similarity may be defined thus:
Two figures are similar (i)~if to any point
in one figure a point in the other figure
corresponds, so that to every line there is a
corresponding line, and to every angle a
corresponding angle, and (ii)~if the lengths
of corresponding lines are in a fixed proportion,
and the magnitudes of corresponding
angles are the same. The fixed proportion
of the lengths of corresponding lines in a map
(or plan) and in the original is called the scale
of the map. The scale should always be
indicated on the margin of every map and
plan. It has already been pointed out that
two triangles whose angles are respectively
equal are similar. Thus, if the two triangles
\Figure{24}
$ABC$ and~$DEF$ have the angles at $A$ and $D$
equal, and those at $B$ and~$E$, and those at $C$
and~$F$, then $DE$~is to~$AB$ in the same proportion
\PageSep{179}
as $EF$~is to~$BC$, and as $FD$~is to~$CA$.
But it is not true of other figures that similarity
is guaranteed by the mere equality of
angles. Take for example, the familiar cases
of a rectangle and a square. Let $ABCD$~be
a square, and $ABEF$~be a rectangle. Then
all the corresponding angles are equal. But
\Figure[2.75in]{25}
whereas the side~$AB$ of the square is equal to
the side~$AB$ of the rectangle, the side~$BC$ of
the square is about half the size of the side~$BE$
of the rectangle. Hence it is not true
that the square $ABCD$ is similar to the rectangle
$ABEF$. This peculiar property of the
triangle, which is not shared by other rectilinear
figures, makes it the fundamental
figure in the theory of similarity. Hence in
surveys, triangulation is the fundamental
process; and hence also arises the word ``trigonometry,''
\PageSep{180}
\index{Circle|EtSeq}%
derived from the two Greek
words \Foreign{trigonon} a triangle and \Foreign{metria} measurement.
The fundamental question from which
trigonometry arose is this: Given the magnitudes
of the angles of a triangle, what can be
stated as to the relative magnitudes of the
sides. Note that we say ``\emph{relative} magnitudes
of the sides,'' since by the theory of similarity
it is only the proportions of the sides which
are known. In order to answer this question,
certain functions of the magnitudes of
an angle, considered as the argument, are introduced.
In their origin these functions
were got at by considering a right-angled triangle,
and the magnitude of the angle was
defined by the length of the arc of a circle.
In modern elementary books, the fundamental
position of the arc of the circle as defining
the magnitude of the angle has been
pushed somewhat to the background, not to
the advantage either of theory or clearness
of explanation. It must first be noticed
that, in relation to similarity, the circle holds
the same fundamental position among curvilinear
figures, as does the triangle among
rectilinear figures. Any two circles are similar
figures; they only differ in scale. The
lengths of the circumferences of two circles,
such as $APA'$ and $A_{1} P_{1} A_{1}'$ in the \Fig[fig.]{26} are
in proportion to the lengths of their radii.
Furthermore, if the two circles have the same
\PageSep{181}
centre~$O$, as do the two circles in \Fig[fig.]{26}, then
the arcs $AP$ and $A_{1} P_{1}$ intercepted by the
arms of any angle~$AOP$, are also in proportion
to their radii. Hence the ratio of the
\Figure{26}
length of the arc~$AP$ to the length of the
radius~$OP$, that is $\dfrac{\text{arc } AP}{\text{radius } OP}$ is a number which
is quite independent of the length~$OP$, and is
the same as the fraction $\dfrac{\text{arc } A_{1} P_{1}}{\text{radius } OP_{1}}$. This fraction
of ``arc divided by radius'' is the proper
theoretical way to measure the magnitude of
\PageSep{182}
\index{Cosine|EtSeq}%
\index{Sine|EtSeq}%
an angle; for it is dependent on no arbitrary
unit of length, and on no arbitrary way of
dividing up any arbitrarily assumed angle,
such as a right-angle. Thus the fraction~$\dfrac{AP}{OA}$
represents the magnitude of the angle~$AOP$.
Now draw $PM$ perpendicularly to~$OA$. Then
the Greek mathematicians called the line~$PM$
the sine of the arc~$AP$, and the line~$OM$ the
cosine of the arc~$AP$. They were well aware
that the importance of the relations of these
various lines to each other was dependent on
the theory of similarity which we have just
expounded. But they did not make their
definitions express the properties which arise
from this theory. Also they had not in their
heads the modern general ideas respecting
functions as correlating pairs of variable numbers,
nor in fact were they aware of any
modern conception of algebra and algebraic
analysis. Accordingly, it was natural to
them to think merely of the relations between
certain lines in a diagram. For us the case
is different: we wish to embody our more
powerful ideas.

Hence, in modern mathematics, instead
of considering the arc~$AP$, we consider
the fraction~$\dfrac{AP}{OP}$, which is a number the
same for all lengths of~$OP$; and, instead of
considering the lines $PM$ and~$OM$, we consider
\PageSep{183}
the fractions $\dfrac{PM}{OP}$ and~$\dfrac{OM}{OP}$, which again
are numbers not dependent on the length of~$OP$,
\ie~not dependent on the scale of our
diagrams. Then we define the number $\dfrac{PM}{OP}$
to be the \emph{sine} of the number $\dfrac{PA}{OP}$, and the
number $\dfrac{OM}{OP}$ to be the \emph{cosine} of the number
$\dfrac{PA}{OP}$. These fractional forms are clumsy to
print; so let us put $u$ for the fraction~$\dfrac{AP}{OP}$,
which represents the magnitude of the angle~$AOP$,
and put $v$ for the fraction~$\dfrac{PM}{OM}$, and $w$~for
the fraction~$\dfrac{OM}{OP}$. Then $u$,~$v$,~$w$, are numbers,
and, since we are talking of \emph{any} angle~$AOP$,
they are variable numbers. But a
correlation exists between their magnitudes,
so that when $u$ (\ie\ the angle~$AOP$) is given
the magnitudes of $v$~and~$w$ are definitely determined.
Hence $v$~and~$w$ are functions of the
argument~$u$. We have called $v$ the \emph{sine} of~$u$,
and $w$ the \emph{cosine} of~$u$. We wish to adapt
the general functional notation $y = f(x)$ to
these special cases: so in modern mathematics
%[** TN: Function names italicized in the original]
we write \Chg{$\sin$}{``$\sin$''} for~``$f$'' when we want to
\PageSep{184}
indicate the special function of ``sine,'' and
``$\cos$'' for~``$f$'' when we want to indicate
the special function of ``cosine.'' Thus, with
the above meanings for $u$,~$v$,~$w$, we get
\[
v = \sin u,\quad\text{and}\quad
w = \cos u,
\]
where the brackets surrounding the~$x$ in~$f(x)$
are omitted for the special functions. The
meaning of these functions $\sin$ and $\cos$ as
correlating the pairs of numbers $u$~and~$v$, and
$u$~and~$w$ is, that the functional relations are to
be found by constructing (\cf\ \Fig[fig.]{26}) an angle~$AOP$,
whose measure ``$AP$~divided by~$OP$''
is equal to~$u$, and that then $v$~is the number
given by ``$PM$~divided by~$OP$'' and $w$~is the
number given by ``$OM$~divided by~$OP$.''

It is evident that without some further definitions
we shall get into difficulties when the
number~$u$ is taken too large. For then the arc~$AP$
may be greater than one-quarter of the
circumference of the circle, and the point~$M$
(\cf\ figs.\ \FigNum{26} and~\FigNum{27}) may fall between $O$ and~$A'$
and not between $O$ and~$A$. Also $P$~may be
below the line~$AOA'$ and not above it as in
\Fig[fig.]{26}. In order to get over this difficulty
we have recourse to the ideas and conventions
of coordinate geometry in making our
complete definitions of the sine and cosine.
Let one arm~$OA$ of the angle be the axis~$OX$,
and produce the axis backwards to
obtain its negative part~$OX'$. Draw the
\PageSep{185}
other axis~$YOY'$ perpendicular to it. Let
any point~$P$ at a distance~$r$ from~$O$ have
coordinates $x$ and~$y$. These coordinates are
both positive in the first ``quadrant'' of
the plan, \eg\ the coordinates $x$ and~$y$ of~$P$
\Figure{27}
in \Fig[fig.]{27}. In the other quadrants, either
one or both of the coordinates are negative,
for example, $x'$~and~$y$ for~$P'$, and $x'$ and~$y'$
for~$P''$, and $x$ and~$y'$ for~$P'''$ in \Fig[fig.]{27}, where
$x'$ and~$y'$ are both negative numbers. The
positive angle~$POA$ is the arc~$AP$ divided
by~$r$, its sine is~$\dfrac{y}{r}$ and its cosine is~$\dfrac{x}{r}$; the positive
\PageSep{186}
angle~$AOP'$ is the arc~$ABP'$ divided by~$r$,
its sine is~$\dfrac{y}{r}$ and cosine~$\dfrac{x'}{r}$; the positive angle~$AOP''$
is the arc $ABA'P''$ divided by~$r$, its
sine is~$\dfrac{y'}{r}$ and its cosine is~$\dfrac{x'}{r}$; the positive
angle~$AOP'''$ is the arc $ABA'B'P'''$ divided
by~$r$, its sine is~$\dfrac{y'}{r}$ and its cosine is~$\dfrac{x}{r}$.

But even now we have not gone far enough.
For suppose we choose~$u$ to be a number
greater than the ratio of the whole circumference
of the circle to its radius. Owing to
the similarity of all circles this ratio is the
same for all circles. It is always denoted in
mathematics by the symbol~$2\pi$, where $\pi$~is
the Greek form of the letter~\Foreign{p} and its
name in the Greek alphabet is ``pi.'' It can
be proved that $\pi$~is an incommensurable
number, and that therefore its value cannot
be expressed by any fraction, or by any
terminating or recurring decimal. Its value
to a few decimal places is~$3.14159$; for many
purposes a sufficiently accurate approximate
value is~$\dfrac{22}{7}$. Mathematicians can easily calculate~$\pi$
to any degree of accuracy required,
just as~$\sqrt{2}$ can be so calculated. Its value
has been actually given to $707$~places of
\PageSep{187}
decimals. Such elaboration of calculation is
merely a curiosity, and of no practical or
theoretical interest. The accurate determination
of~$\pi$ is one of the two parts of
the famous problem of squaring the circle.
\index{Squaring the Circle}%
The other part of the problem is, by the
theoretical methods of pure geometry to
describe a straight line equal in length to the
circumference. Both parts of the problem
are now known to be impossible; and the
insoluble problem has now lost all special
practical or theoretical interest, having become
absorbed in wider ideas.

After this digression on the value of~$\pi$, we
now return to the question of the general
definition of the magnitude of an angle, so as
to be able to produce an angle corresponding
to any value~$u$. Suppose a moving point,~$Q$,
to start from~$A$ on~$OX$ (\Chg{cf.}{\cf}\ \Fig[fig.]{27}), and to rotate
in the positive direction (anti-clockwise, in
the figure considered) round the circumference
of the circle for any number of times, finally
resting at any point, \eg~at $P$ or~$P'$ or~$P''$ or~$P'''$.
Then the total length of the curvilinear
circular path traversed, divided by the radius
of the circle,~$r$, is the generalized definition of
a positive angle of \emph{any} size. Let $x$,~$y$ be the
coordinates of the point in which the point~$Q$
rests, \ie~in one of the four alternative positions
mentioned in \Fig[fig.]{27}; $x$~and~$y$ (as here used) will
either \Typo{}{be} $x$~and~$y$, or $x'$~and~$y$, or $x'$~and~$y'$, or $x$~and~$y'$.
\PageSep{188}
Then the sign of this generalized
angle is~$\dfrac{y}{r}$ and its cosine is~$\dfrac{x}{r}$. With these
definitions the functional relations $v = \sin u$
and $w = \cos u$, are at last defined for all positive
real values of~$u$. For negative values of~$u$
we simply take rotation of~$Q$ in the opposite
(clockwise) direction; but it is not worth our
while to elaborate further on this point, now
that the general method of procedure has
been explained.

These functions of sine and cosine, as thus
defined, enable us to deal with the problems
concerning the triangle from which Trigonometry
took its rise. But we are now in a
position to relate Trigonometry to the wider
idea of Periodicity of which the importance
\index{Periodicity}%
was explained in the last chapter. It is easy
to see that the functions $\sin u$ and $\cos u$ are
periodic functions of~$u$. For consider the
position,~$P$ (in \Fig[fig.]{27}), of a moving point,~$Q$,
which has started from~$A$ and revolved round
the circle. This position,~$P$, marks the angles
$\dfrac{\text{arc } AP}{r}$, and $2\pi + \dfrac{\text{arc } AP}{r}$, and $4\pi + \dfrac{\text{arc } AP}{r}$,
and $6\pi + \dfrac{\text{arc } AP}{r}$, and so on indefinitely. Now,
all these angles have the same sine and cosine,
namely, $\dfrac{y}{r}$~and~$\dfrac{x}{r}$. Hence it is easy to see that,
\PageSep{189}
\index{Period|EtSeq}%
if $u$ be chosen to have any value, the arguments
$u$~and~$2\pi + u$, and $4\pi + u$, and $6\pi + u$,
and $8\pi + u$ and so on indefinitely, have all the
same values for the corresponding sines and
cosines. In other words,
\begin{alignat*}{4}
\sin u &= \sin(2\pi + u) &&= \sin(4\pi + u) &&= \sin(6\pi + u) &&= \text{etc.}; \\
\cos u &= \cos(2\pi + u) &&= \cos(4\pi + u) &&= \cos(6\pi + u) &&= \text{etc.}
\end{alignat*}
This fact is expressed by saying that $\sin u$ and
$\cos u$ are periodic functions with their period
equal to~$2\pi$.

The graph of the function $y = \sin x$ (notice
that we now abandon $v$~and~$u$ for the more
familiar $y$~and~$x$) is shown in \Fig[fig.]{28}. We take
on the axis of~$x$ any arbitrary length at pleasure
to represent the number~$\pi$, and on the axis
of~$y$ any arbitrary length at pleasure to represent
the number~$1$. The numerical values of
the sine and cosine can never exceed unity.
The recurrence of the figure after periods of~$2\pi$
will be noticed. This graph represents the
simplest style of periodic function, out of
which all others are constructed. The cosine
gives nothing fundamentally different from the
sine. For it is easy to prove that $\cos x = \sin(x + \dfrac{\pi}{2})$;
hence it can be seen that the
graph of $\cos x$ is simply \Fig[fig.]{28} modified by
\PageSep{190}
drawing the axis of~$OY$ through the point
on~$OX$ marked~$\dfrac{\pi}{2}$, instead of drawing it in
its actual position on the figure.

It is easy to construct a `sine' function in
\Figure{28}
which the period has any assigned value~$a$.
For we have only to write
\[
y = \sin \frac{2\pi x}{a},
\]
and then
\[
\sin \frac{2\pi (x + a)}{a}
%[** TN: Changed curly braces to parentheses]
  = \sin \left(\frac{2\pi x}{a} + 2\pi\right)
  = \sin \frac{2\pi x}{a}.
\]
Thus the period of this new function is now~$a$.
Let us now give a general definition of what
\PageSep{191}
we mean by a periodic function. The function~$f(x)$
is periodic, with the period~$a$, if (i)~for \emph{any}
value of~$x$ we have $f(x) = f(x + a)$, and (ii)~there
is no number~$b$ smaller than~$a$ such that for
\emph{any} value of~$x$, $f(x) = f(x + b)$.

The second clause is put into the definition
because when we have $\sin \dfrac{2\pi x}{a}$, it is not only
periodic in the period~$a$, but also in the periods
$2a$ and~$3a$, and so on; this arises since
\[
\sin \frac{2\pi (x + 3a)}{a}
  = \sin \left(\frac{2\pi x}{a} + 6\pi\right)
  = \sin \frac{2\pi x}{a}.
\]
So it is the smallest period which we want to
get hold of and call \emph{the} period of the function.
The greater part of the abstract theory of
periodic functions and the whole of the applications
of the theory to Physical Science are
dominated by an important theorem called
Fourier's Theorem; namely that, if $f(x)$ be a
\index{Fourier's Theorem}%
periodic function with the period~$a$ and if $f(x)$
also satisfies certain conditions, which practically
are always presupposed in functions suggested
by natural phenomena, then $f(x)$ can
be written as the sum of a set of terms in the
form\Pagelabel{191}
\begin{multline*}
c_{0} + c_{1} \sin \left(\frac{2\pi x}{a} + e_{1}\right)
  + c_{2} \sin \left(\frac{4\pi x}{a} + e_{2}\right) \\
  + c_{3} \sin \left(\frac{6\pi x}{a} + e_{3}\right) + \text{etc.}
\end{multline*}
\PageSep{192}
In this formula $c_{0}$,~$c_{1}$, $c_{2}$, $c_{3}$,~etc., and also
$e_{1}$,~$e_{2}$, $e_{3}$,~etc., are constants, chosen so as to
suit the particular function. Again we have
to ask, How many terms have to be chosen?
And here a new difficulty arises: for we can
prove that, though in some particular cases a
definite number will do, yet in general all we
can do is to approximate as closely as we like
to the value of the function by taking more
and more terms. This process of gradual
approximation brings us to the consideration
of the theory of infinite series, an essential
part of mathematical theory which we will
consider in the next chapter.

The above method of expressing a periodic
\index{Harmonic Analysis}%
function as a sum of sines is called the ``harmonic
analysis'' of the function. For example,
at any point on the sea coast the tides
rise and fall periodically. Thus at a point
near the Straits of Dover there will be two
daily tides due to the rotation of the earth.
The daily rise and fall of the tides are complicated
by the fact that there are two tidal
waves, one coming up the English Channel,
and the other which has swept round the
North of Scotland, and has then come southward
down the North Sea. Again some high
tides are higher than others: this is due to
the fact that the Sun has also a tide-generating
influence as well as the Moon. In this way
monthly and other periods are introduced.
\PageSep{193}
We leave out of account the exceptional influence
of winds which cannot be foreseen.
The general problem of the harmonic analysis
of the tides is to find sets of terms like those
in the expression on \Pageref[page]{191} above, such that
each set will give with approximate accuracy
the contribution of the tide-generating influences
of one ``period'' to the height of the
tide at any instant. The argument~$x$ will
therefore be the \emph{time} reckoned from any convenient
commencement.

Again, the motion of vibration of a violin
string is submitted to a similar harmonic
analysis, and so are the vibrations of the
ether and the air, corresponding respectively
to waves of light and waves of sound. We
are here in the presence of one of the fundamental
processes of mathematical physics---namely,
nothing less than its general method
of dealing with the great natural fact of
Periodicity.
\PageSep{194}


\Chapter{XIV}{Series}

\First{No} part of Mathematics suffers more from
\index{Order|EtSeq}%
\index{Series|EtSeq}%
the triviality of its initial presentation to
beginners than the great subject of series.
Two minor examples of series, namely arithmetic
and geometric series, are considered;
these examples are important because they
are the simplest examples of an important
general theory. But the general ideas are
never disclosed; and thus the examples, which
exemplify nothing, are reduced to silly trivialities.

The general mathematical idea of a series
is that of a set of things ranged in order, that
is, in sequence; This meaning is accurately
represented in the common use of the term.
Consider for example, the series of English
Prime Ministers during the nineteenth century,
arranged in the order of their first tenure of
that office within the century. The series
commences with William Pitt, and ends with
\index{Pitt, William}%
\index{Rosebery, Lord}%
Lord Rosebery, who, appropriately enough,
is the biographer of the first member. We
\PageSep{195}
might have considered other serial orders for
the arrangement of these men; for example,
according to their height or their weight.
These other suggested orders strike us as
trivial in connection with Prime Ministers,
and would not naturally occur to the mind;
but abstractly they are just as good orders
as any other. When one order among terms
is very much more important or more obvious
than other orders, it is often spoken of as \emph{the}
order of those terms. Thus \emph{the} order of the
integers would always be taken to mean their
order as arranged in order of magnitude. But
of course there is an indefinite number of
other ways of arranging them. When the
number of things considered is finite, the
number of ways of arranging them in order is
called the number of their permutations. The
number of permutations of a set of $n$~things,
where $n$~is some finite integer, is
\[
n × (n - 1) × (n - 2) × (n - 3) × \dots × 4 × 3 × 2 × 1\Add{,}
\]
that is to say, it is the product of the first $n$
integers; this product is so important in
mathematics that a special symbolism, is used
for it, and it is always written~`$n!$\Add{.}' Thus,
$2! = 2 × 1 = 2$, and $3! = 3 × 2 × 1 = 6$, and $4! = 4 × 3 × 2 × 1 = 24$,
and $5! = 5 × 4 × 3 × 2 × 1 = 120$.
As $n$~increases, the value of~$n!$ increases very
quickly; thus $100!$~is a hundred times as
large as~$99!$\Add{.}
\PageSep{196}

It is easy to verify in the case of small
values of~$n$ that $n!$ is the number of ways
of arranging $n$~things in order. Thus consider
two things $a$ and~$b$; these are capable
of the two orders $ab$ and~$ba$, and $2! = 2$.

Again, take three things $a$,~$b$, and~$c$; these
are capable of the six orders, $abc$, $acb$, $bac$,
$bca$, $cab$, $cba$, and $3! = 6$. Similarly for the
twenty-four orders in which four things $a$,~$b$,~$c$,
and~$d$, can be arranged.

When we come to the infinite sets of things---like
\index{Order, Type of}%
the sets of all the integers, or all the
fractions, or all the real numbers for instance---we
come at once upon the complications of
the theory of order-types. This subject was
touched upon in \ChapRef{VI}. in considering
the possible orders of the integers, and of the
fractions, and of the real numbers. The
whole question of order-types forms a comparatively
new branch of mathematics of
great importance. We shall not consider it
any further. All the infinite series which we
consider now are of the same order-type as
the integers arranged in ascending order of
magnitude, namely, with a first term, and
such that each term has a couple of next-door
neighbours, one on either side, with the
exception of the first term which has, of
course, only one next-door neighbour. Thus,
if $m$~be any integer (not zero), there will be
always an $m$th~term. A series with a finite
\PageSep{197}
number of terms (say $n$~terms) has the same
characteristics as far as next-door neighbours
are concerned as an infinite series; it only
differs from infinite series in having a last
term, namely, the~$n$th.

The important thing to do with a series of
numbers---using for the future ``series'' in
the restricted sense which has just been mentioned---is
to add its successive terms together.

Thus if $u_{1}$,~$u_{2}$, $u_{3}$,~\dots\Add{,} $u_{n}$,~\dots\ are respectively
the $1$st,~$2$nd, $3$rd, $4$th,~\dots\Add{,} $n$th,~\dots\
terms of a series of numbers, we form successively
the series $u_{1}$, $u_{1} + u_{2}$, $u_{1} + u_{2} + u_{3}$, $u_{1} + u_{2} + u_{3} + u_{4}$,
and so on; thus the sum of the
$1$st $n$~terms may be written\Typo{.}{}
\[
u_{1} + u_{2} + u_{3} + \dots + u_{n}.
\]

If the series has only a finite number of
\index{Approximation|EtSeq}%
terms, we come at last in this way to the
sum of the whole series of terms. But, if
the series has an infinite number of terms,
this process of successively forming the sums
of the terms never terminates; and in this
sense there is no such thing as the sum of an
infinite series.

But why is it important successively to add
the terms of a series in this way? The answer
is that we are here symbolizing the fundamental
mental process of approximation.
This is a process which has significance far
\PageSep{198}
beyond the regions of mathematics. Our
limited intellects cannot deal with complicated
material all at once, and our method of
arrangement is that of approximation. The
statesman in framing his speech puts the
dominating issues first and lets the details
fall naturally into their subordinate places.
There is, of course, the converse artistic
method of preparing the imagination by the
presentation of subordinate or special details,
and then gradually rising to a crisis. In
either way the process is one of gradual summation
of effects; and this is exactly what
is done by the successive summation of the
terms of a series. Our ordinary method of
stating numbers is such a process of gradual
summation, at least, in the case of large
numbers. Thus $568,213$ presents itself to
the mind as\Add{:}---
\[
500,000 + 60,000 + 8,000 + 200 + 10 + 3\Add{.}
\]

In the case of decimal fractions this is so
more avowedly. Thus $3.14159$ is\Add{:}---
\[
3 + \tfrac{1}{10} + \tfrac{4}{100} + \tfrac{1}{1000} + \tfrac{5}{10000} + \tfrac{9}{100000}\Add{.}
\]
Also, $3$ and~$3 + \frac{1}{10}$, and $3 + \tfrac{1}{10} + \tfrac{4}{100}$, and
$3 + \tfrac{1}{10} + \tfrac{4}{100} + \tfrac{1}{1000}$,
and $3 + \tfrac{1}{10} + \tfrac{4}{100} + \tfrac{1}{1000} + \tfrac{5}{10000}$ are
successive approximations to the complete result
$3.14159$. If we read $568,213$ backwards
from right to left, starting with the $3$~units,
\PageSep{199}
we read it in the artistic way, gradually preparing
the mind for the crisis of~$500,000$.

The ordinary process of numerical multiplication
proceeds by means of the summation
of a series, Consider the computation
\[
\begin{array}{*{6}{@{}c@{}}}
 & & &3&4&2 \\
 & & &6&5&8 \\
\cline{4-6}
\Strut
 & &2&7&3&6 \\
 &1&7&1&0&  \\
2&0&5&2& &  \\
\cline{1-6}
\Strut
2&2&5&0&3&6
\end{array}
\]

Hence the three lines to be added form a
series of which the first term is the upper
line. This series follows the artistic method
of presenting the most important term last,
not from any feeling for art, but because of
the convenience gained by keeping a firm
hold on the units' place, thus enabling us to
omit some~$0$'s, formally necessary.

But when we approximate by gradually
\index{Limit of a Series|EtSeq}%
adding the successive terms of an infinite
series, what are we approximating to? The
difficulty is that the series has no ``sum'' in
the straightforward sense of the word, because
the operation of adding together its terms
can never be completed. The answer is that
we are approximating to the \emph{limit} of the
summation of the series, and we must now
\PageSep{200}
proceed to explain what the ``limit'' of a
series is.

The summation of a series approximates to
a limit when the sum of any number of its
terms, provided the number be large enough,
is as nearly equal to the limit as you care to
approach. But this description of the meaning
of approximating to a limit evidently will
not stand the vigorous scrutiny of modern
mathematics. What is meant by \emph{large
enough}, and by \emph{nearly equal}, and by \emph{care to
approach}? All these vague phrases must be
explained in terms of the simple abstract
ideas which alone are admitted into pure
mathematics.

Let the successive terms of the series be
$u_{1}$,~$u_{2}$, $u_{3}$, $u_{4}$,~\dots, $u_{n}$, etc., so that $u_{n}$~is the
$n$th~term of the series. Also let $s_{n}$ be the
sum of the $1$st $n$~terms, whatever $n$~may be.
So that\Add{:}---
\begin{gather*}
s_{1} = u_{1},\quad
s_{2} = u_{1} + u_{2},\quad
s_{3} = u_{1} + u_{2} + u_{3},\quad\text{and} \\
s_{n} = u_{1} + u_{2} + u_{3} + \dots + u_{n}.
\end{gather*}

Then the terms $s_{1}$,~$s_{2}$, $s_{3}$,~\dots\Add{,} $s_{n}$,~\dots\ form
a new series, and the formation of this series
is the process of summation of the original
series. Then the ``approximation'' of the
\emph{summation} of the original series to a ``limit''
means the ``approximation of the \emph{terms} of
this new series to a limit.'' And we have
\PageSep{201}
now to explain what we mean by the approximation
to a limit of the terms of a series.

Now, remembering the definition (given in
\ChapRef[chapter]{XII}.)\ of a \emph{standard of approximation},
\index{Standard of Approximation|EtSeq}%
\index{Sum to Infinity|EtSeq}%
the idea of a limit means this: $l$~is
the limit of the terms of the series $s_{1}$,~$s_{2}$,
$s_{3}$,~\dots\Add{,} $s_{n}$,~\dots, if, corresponding to each
real number~$k$, taken as a standard of
approximation, a term~$s_{n}$ of the series can
be found so that all succeeding terms (\ie\
$s_{n+1}$, $s_{n+2}$, etc.)\ approximate to~$l$ within
that standard of approximation. If another
smaller standard~$k^{1}$ be chosen, the term~$s_{n}$
may be too early in the series, and a
later term~$s_{m}$ with the above property will
then be found.

If this property holds, it is evident that as
you go along to series $s_{1}$,~$s_{2}$, $s_{3}$,~\dots, $s_{n}$,~\dots\
from left to right, after a time you come to
terms \emph{all} of which are nearer to~$l$ than any
number which you may like to assign. In
other words you approximate to~$l$ as closely
as you like. The close connection of this
definition of the limit of a series with the
definition of a continuous function given in
\ChapRef[chapter]{XI}.\ will be immediately perceived.

Then coming back to the original series $u_{1}$,~$u_{2}$,
$u_{3}$,~\dots, $u_{n}$,~\dots, the limit of the terms of
the series $s_{1}$,~$s_{2}$, $s_{3}$,~\dots, $s_{n}$,~\dots, is called
the ``sum to infinity'' of the original series.
But it is evident that this use of the word
\PageSep{202}
``sum'' is very artificial, and we must not
assume the analogous properties to those of
the ordinary sum of a finite number of terms
without some special investigation.

Let us look at an example of a ``sum to
infinity.'' Consider the recurring decimal
$.1111\dots$. This decimal is merely a way of
symbolizing the ``sum to infinity'' of the series
$.1$, $.01$, $.001$, $.0001$, etc. The corresponding
series found by summation is $s_{1} = .1$,
$s_{2} = .11$, $s_{3} = .111$, $s_{4} = .1111$, etc. The limit
of the terms of this series is~$\frac{1}{9}$; this is easy to
see by simple division, for
\[
\tfrac{1}{9}
  = .1 + \tfrac{1}{90}
  = .11 + \tfrac{1}{900}
  = .111 + \tfrac{1}{9000} = \text{etc.}
\]
Hence, if $\frac{3}{17}$ is given (the $k$ of the definition),
$.1$~and \emph{all} succeeding terms differ from~$\frac{1}{9}$ by
less than~$\frac{3}{17}$; if $\frac{1}{1000}$ is given (another choice
for the $k$ of the definition), $.111$ and all
succeeding terms differ from~$\frac{1}{9}$ by less than~$\frac{1}{1000}$;
and so on, whatever choice for~$k$ be
made.

It is evident that nothing that has been
said gives the slightest idea as to how the
``sum to infinity'' of a series is to be
found. We have merely stated the conditions
which such a number is to satisfy. Indeed,
a general method for finding in all
cases the sum to infinity of a series is intrinsically
out of the question, for the simple reason
that such a ``sum,'' as here defined, does not
always exist. Series which possess a sum to
\PageSep{203}
\index{Convergent|EtSeq}%
\index{Divergent|EtSeq}%
infinity are called \emph{convergent}, and those which
do not possess a sum to infinity are called
\emph{divergent}.

An obvious example of a divergent series
is $1$,~$2$, $3$,~\dots, $n$~\dots\Add{,} \ie~the series of integers
in their order of magnitude. For
whatever number~$l$ you try to take as its
sum to infinity, and whatever standard of
approximation~$k$ you choose, by taking
enough terms of the series you can always
make their sum differ from~$l$ by more than~$k$.
Again, another example of a divergent
series is $1$,~$1$, $1$,~etc., \ie~the series of
which each term is equal to~$1$. Then the
sum of $n$~terms is~$n$, and this sum grows
without limit as $n$~increases. Again, another
example of a divergent series is $1$,~$-1$, $1$,~$-1$,
$1$,~$-1$, etc., \ie~the series in which the terms
are alternately $1$ and~$-1$. The sum of an
odd number of terms is~$1$, and of an even
number of terms is~$0$. Hence the terms of
the series $s_{1}$,~$s_{2}$, $s_{3}$,~\dots\Add{,} $s_{n}$,~\dots\ do not approximate
to a limit, although they do not
increase without limit.

It is tempting to suppose that the condition
for $u_{1}$,~$u_{2}$,~\dots\Add{,} $u_{n}$,~\dots\ to have a sum
to infinity is that $u_{n}$~should decrease indefinitely
as $n$~increases. Mathematics would
be a much easier science than it is, if this
were the case. Unfortunately the supposition
is not true.
\PageSep{204}

For example the series
\[
1,\quad
\frac{1}{2},\quad
\frac{1}{3},\quad
\frac{1}{4},\ \dots,\quad
\frac{1}{n},\ \dots
\]
is divergent. It is easy to see that this is
the case; for consider the sum of $n$~terms
%[** TN: "(n + 1)^{th} term" in the original
beginning at the $(n + 1)$th term. These $n$~terms
are $\dfrac{1}{n + 1}$, $\dfrac{1}{n + 2}$, $\dfrac{1}{n + 3}$,~\dots\Add{,} $\dfrac{1}{2n}$: there
are $n$~of them and $\dfrac{1}{2n}$~is the least among them.
Hence their sum is greater than $n$~times~$\dfrac{1}{2n}$,
\ie~is greater than~$\dfrac{1}{2}$. Now, without
altering the sum to infinity, if it exist, we
can add together neighbouring terms, and
obtain the series
\[
1,\quad
\tfrac{1}{2},\quad
\tfrac{1}{3} + \tfrac{1}{4},\quad
\tfrac{1}{5} + \tfrac{1}{6} + \tfrac{1}{7} + \tfrac{1}{8},\quad
\text{etc.},
\]
that is, by what has been said above, a series
whose terms after the~$2$nd are greater than
those of the series,
\[
1,\quad
\tfrac{1}{2},\quad
\tfrac{1}{2},\quad
\tfrac{1}{2},\quad
\text{etc.},
\]
where all the terms after the first are equal.
But this series is divergent. Hence the
original series is divergent.\footnote
  {\Chg{Cf.}{\Cf}\ Note~C, \Pageref{noteC}.\Pagelabel{204}}

This question of divergency shows how
careful we must be in arguing from the properties
\PageSep{205}
of the sum of a finite number of terms
to that of the sum of an infinite series. For
the most elementary property of a finite
number of terms is that of course they
possess a sum: but even this fundamental
property is not necessarily possessed by an
infinite series. This caution merely states
that we must not be misled by the suggestion
of the technical term ``\emph{sum} of an infinite
series.'' It is usual to indicate the sum of
the infinite series
\[
u_{1},\quad
u_{2},\quad
u_{3},\ \dots\Add{,}\quad
u_{n}\Add{,}\ \dots
\]
by
\[
u_{1} + u_{2} + u_{3} + \dots + u_{n} + \dots\Add{.}
\]

We now pass on to a generalization of the
idea of a series, which mathematics, true to
its method, makes by use of the variable.
Hitherto, we have only contemplated series
in which each definite term was a definite
number. But equally well we can generalize,
and make each term to be some mathematical
expression containing a variable~$x$. Thus
we may consider the series $1$,~$x$, $x^{2}$, $x^{3}$,~\dots,
$x^{n}$,~\dots, and the series
\[
x,\quad
\frac{x^{2}}{2},\quad
\frac{x^{3}}{3},\ \dots,\quad
\frac{x^{n}}{n},\ \dots\Add{.}
\]

In order to symbolize the general idea of
any such function, conceive of a function of~$x$,
$f_{n}(x)$~say, which involves in its formation
a variable integer~$n$, then, by giving~$n$ the
\PageSep{206}
values $1$,~$2$, $3$,~etc., in succession, we get the
series
\[
f_{1}(x),\quad
f_{2}(x),\quad
f_{3}(x),\ \dots,\quad
f_{n}(x),\dots\Add{.}
\]
Such a series may be convergent for some
values of~$x$ and divergent for others. It is,
in fact, rather rare to find a series involving a
variable~$x$ which is convergent for all values
of~$x$,---at least in any particular instance it is
very unsafe to assume that this is the case.
For example, let us examine the simplest of
all instances, namely, the ``geometrical''
\index{Geometrical Series|EtSeq}%
series
\[
1,\quad x,\quad x^{2},\quad x^{3},\ \dots,\quad x^{n},\ \dots\Add{.}
\]
The sum of $n$~terms is given by
\[
s_{n} = 1 + x + x^{2} + x^{3} + \dots + x^{n}.
\]

Now multiply both sides by~$x$ and we get
\[
xs_{n} = x + x^{2} + x^{3} + x^{4} + \dots + x^{n} + x^{n+1}\Add{.}
\]
Now subtract the last line from the upper
line and we get
\[
s_{n}(1 - x) = s_{n} - xs_{n} = 1 - x^{n+1},
\]
and hence (if $x$~be not equal to~$1$)
\[
s_{n} = \frac{1 - x^{n+1}}{1 - x}
  = \frac{1}{1 - x} - \frac{x^{n+1}}{1 - x}\Add{.}
\]
Now if $x$~be numerically less than~$1$, for sufficiently
large values of~$n$, $\dfrac{x^{n+1}}{1 - x}$~is always numerically
\PageSep{207}
less than~$k$, however $k$~be chosen. Thus,
if $x$~be numerically less than~$1$, the series $1$,~$x$,
$x^{2}$,~\dots\Add{,} $x^{n}$,~\dots\ is convergent, and $\dfrac{1}{1 - x}$~is its
limit. This statement is symbolized by
\[
\frac{1}{1 - x} = 1 + x + x^{2} + \dots + x^{n} + \dots,\quad
(-1 < x < 1).
\]
But if $x$~is numerically greater than~$1$, or
numerically equal to~$1$, the series is divergent.
In other words, if $x$~lie between $-1$ and~$+1$,
the series is convergent; but if $x$~be equal
to~$-1$ or~$+1$, or if $x$~lie outside the interval
$-1$~to~$+1$, then the series is divergent. Thus
the series is convergent at all ``points''
within the interval $-1$~to~$+1$, exclusive of
the end points.

At this stage of our enquiry another question
arises. Suppose that the series
\[
f_{1}(x) + f_{2}(x) + f_{3}(x) + \dots + f_{n}(x) + \dots
\]
is convergent for all values of~$x$ lying within
the interval $a$~to~$b$, \ie~the series is convergent
for any value of~$x$ which is greater than~$a$ and
less than~$b$. Also, suppose we want to be
sure that in approximating to the limit we
add together enough terms to come within
some standard of approximation~$k$. Can we
always state some number of terms, say~$n$,
such that, if we take $n$~or more terms to
form the sum, then \emph{whatever} value $x$~has
\PageSep{208}
within the interval we have satisfied the
desired standard of approximation?

Sometimes we can and sometimes we cannot
\index{Non-Uniform Convergence|EtSeq}%
\index{Uniform Convergence|EtSeq}%
do this for each value of~$k$. When we
can, the series is called uniformly convergent
throughout the interval, and when we cannot
do so, the series is called non-uniformly convergent
throughout the interval. It makes
a great difference to the properties of a series
whether it is or is not uniformly convergent
through an interval. Let us illustrate the
matter by the simplest example and the
simplest numbers.

Consider the geometric series
\[
1 + x + x^{2} + x^{3} + \dots + x^{n} + \dots\Add{.}
\]

It is convergent throughout the interval
$-1$~to~$+1$, excluding the end values $x = ±1$.

But it is not uniformly convergent throughout
this interval. For if $s_{n}(x)$~be the sum of
$n$~terms, we have proved that the difference
between $s_{n}(x)$ and the limit~$\dfrac{1}{1 - x}$ is~$\dfrac{x^{n+1}}{1 - x}$.
Now suppose $n$~be any given number of terms,
say~$20$, and let $k$~be any assigned standard
of approximation, say~$.001$. Then, by taking
$x$~near enough to~$+1$ or near enough to~$-1$,
we can make the numerical value of~$\dfrac{x^{21}}{1 - x}$ to
be greater than~$.001$. Thus $20$~terms will
\PageSep{209}
not do over the whole interval, though it is
more than enough over some parts of it.

The same reasoning can be applied whatever
other number we take instead of~$20$,
and whatever standard of approximation instead
of~$.001$. Hence the geometric series
$1 + x + x^{2} + x^{3} + \dots + x^{n} + \dots$ is non-uniformly
convergent over its \emph{whole} interval of
convergence $-1$~to~$+1$. But if we take any
smaller interval lying at both ends within the
interval $-1$~to~$+1$, the geometric series is
uniformly convergent within it. For example,
take the interval $0$~to~$+\frac{1}{10}$. Then any
value for~$n$ which makes $\dfrac{x^{n+1}}{1 - x}$~numerically
less than~$k$ \emph{at} these limits for~$x$ also serves
for all values of~$x$ between these limits, since
it so happens that $\dfrac{x^{n+1}}{1 - x}$ diminishes in numerical
value as $x$~diminishes in numerical value.
For example, take $k = .001$; then, putting
$x = \frac{1}{10}$, we find:\Add{---}
\begin{alignat*}{3}
&\text{for $n = 1$,}\quad & \frac{x^{n+1}}{1 - x}
  &= \frac{(\frac{1}{10})^{2}}{1 - \frac{1}{10}}
  &&= \tfrac{1}{90} = .0111\dots, \\
%
&\text{for $n = 2$,}\quad & \frac{x^{n+1}}{1 - x}
  &= \frac{(\frac{1}{10})^{3}}{1 - \frac{1}{10}}
  &&= \tfrac{1}{900} = .00111\dots, \\
%
&\text{for $n = 3$,}\quad & \frac{x^{n+1}}{1 - x}
  &= \frac{(\frac{1}{10})^{4}}{1 - \frac{1}{10}}
  &&= \tfrac{1}{9000} = .000111\dots\Typo{,}{.}
\end{alignat*}

Thus three terms will do for the whole interval,
\PageSep{210}
though, of course, for some parts of
the interval it is more than is necessary.
Notice that, because $1 + x + x^{2} + \dots + x^{n} + \dots$
is convergent (though not uniformly)
throughout the interval $-1$~to~$+1$,
for each value of~$x$ in the interval some number
of terms~$n$ can be found which will satisfy
a desired standard of approximation; but,
as we take $x$ nearer and nearer to either end
value $+1$ or~$-1$, larger and larger values of~$n$
have to be employed.

It is curious that this important distinction
between uniform and non-uniform convergence
was not published till 1847 by Stokes---afterwards,
\index{Stokes, Sir George}%
Sir~George Stokes---and later, independently
in~1850 by Seidel, a German
\index{Seidel}%
mathematician.

The critical points, where non-uniform convergence
comes in, are not necessarily at the
limits of the interval throughout which convergence
holds. This is a speciality belonging
to the geometric series.

In the case of the geometric series $1 + x + x^{2} + \dots + x^{n} + \dots$,
a simple algebraic
expression~$\dfrac{1}{1 - x}$ can be given for its limit in
its interval of convergence. But this is not
always the case. Often we can prove a series
to be convergent within a certain interval,
though we know nothing more about its
limit except that it is the limit of the series.
\PageSep{211}
But this is a very good way of defining a
function; \viz.\ as the limit of an infinite convergent
series, and is, in fact, the way in which
most functions are, or ought to be, defined.

Thus, the most important series in elementary
\index{Exponential Series|EtSeq}%
analysis is
\[
1 + x + \frac{x^{2}}{2!} + \frac{x^{3}}{3!} + \dots + \frac{x^{n}}{n!} + \dots,
\]
where $n!$ has the meaning defined earlier in
this chapter. This series can be proved to
be absolutely convergent for \emph{all} values of~$x$,
and to be uniformly convergent within any
interval which we like to take. Hence it has
all the comfortable mathematical properties
which a series should have. It is called the
exponential series. Denote its sum to infinity
by~$\exp x$. Thus, by definition,
\[
\exp x = 1 + x + \frac{x^{2}}{2!} + \frac{x^{3}}{3!} + \dots
  + \frac{x^{n}}{n!} + \dots\Add{.}
\]
$\exp x$ is called the exponential function.

It is fairly easy to prove, with a little
knowledge of elementary mathematics, that
\[
(\exp x) × (\exp y) = \exp(x + y).
\Tag{(A)}
\]
In other words that
\begin{multline*}
(\exp x) × (\exp y) \\
  = 1 + (x + y) + \frac{(x + y)^{2}}{2!} + \frac{(x + y)^{3}}{3!} + \dots
  + \frac{(x + y)^{n}}{n!} + \dots\Add{.}
\end{multline*}
\PageSep{212}

This property~\Eq{(A)} is an example of what
is called an addition-theorem. When any
\index{Addition-Theorem}%
function [say~$f(x)$] has been defined, the first
thing we do is to try to express $f(x + y)$ in terms
of known functions of $x$~only, and known functions
of $y$~only. If we can do so, the result
is called an addition-theorem. Addition-theorems
play a great part in mathematical
analysis. Thus the addition-theorem for the
sine is given by
\[
\sin(x + y) = \sin x \cos y + \cos x \sin y,
\]
and for the cosine by
\[
\cos(x + y) = \cos x \cos y - \sin x \sin y.
\]

As a matter of fact the best ways of defining
$\sin x$ and $\cos x$ are not by the elaborate
geometrical methods of the previous chapter,
but as the limits respectively of the series
\[
x - \frac{x^{3}}{3!} + \frac{x^{5}}{5!} - \frac{x^{7}}{7!} + \text{etc.} \dots,
\]
and
\[
1 - \frac{x^{2}}{2!} + \frac{x^{4}}{4!} - \frac{x^{6}}{6!} + \text{etc.} \dots,
\]
so that we put
\begin{align*}
\sin x &= x - \frac{x^{3}}{3!} + \frac{x^{5}}{5!} - \frac{x^{7}}{7!} + \text{etc.} \dots, \\
\cos x &= 1 - \frac{x^{2}}{2!} + \frac{x^{4}}{4!} - \frac{x^{6}}{6!} + \text{etc.} \dots\Typo{,}{.}
\end{align*}
\PageSep{213}

These definitions are equivalent to the geometrical
definitions, and both series can be
proved to be convergent for all values of~$x$,
and uniformly convergent throughout any
interval. These series for sine and cosine
have a general likeness to the exponential
series given above. They are, indeed, intimately
connected with it by means of the
theory of imaginary numbers explained in
Chapters \ChapNum{VII}.\ and~\ChapNum{VIII}.
\Figure{29}

The graph of the exponential function is
given in \Fig[fig.]{29}. It cuts the axis~$OY$ at the
point $y = 1$, as evidently it ought to do, since
when $x = 0$ every term of the series except
the first is zero. The importance of the exponential
function is that it represents any
changing physical quantity whose rate of
increase at any instant is a uniform percentage
of its value at that instant. For
\PageSep{214}
example, the above graph represents the size
at any time of a population with a uniform
birth-rate, a uniform death-rate, and no emigration,
where the $x$ corresponds to the time
reckoned from any convenient day, and the
$y$ represents the population to the proper
scale. The scale must be such that $OA$~represents
the population at the date which is
taken as the origin. But we have here come
upon the idea of ``rates of increase'' which
is the topic for the next chapter.

An important function nearly allied to the
\index{Normal Error, Curve of}%
exponential function is found by putting~$-x^{2}$
for~$x$ as the argument in the exponential function.
%[** TN: Omitted period following "exp"]
We thus get $\exp (-x^{2})$. The graph
$y = \exp(-x^{2})$ is given in \Fig[fig.]{30}.
\Figure{30}

The curve, which is something like a cocked
hat, is called the curve of normal error. Its
\PageSep{215}
corresponding function is vitally important
to the theory of statistics, and tells us in
many cases the sort of deviations from the
average results which we are to expect.

Another important function is found by
combining the exponential function with the
sine, in this way:\Add{---}
\[
y = \exp(-cx) × \sin \frac{2\pi x}{p}\Add{.}
\]
\Figure{31}

Its graph is given in \Fig[fig.]{31}. The points
$A$,~$B$, $O$, $C$, $D$, $E$,~$F$, are placed at equal intervals~$\frac{1}{2}p$,
and an unending series of them
should be drawn forwards and backwards.
This function represents the dying away of
vibrations under the influence of friction or of
``damping'' forces. Apart from the friction,
the vibrations would be periodic, with a
period~$p$; but the influence of the friction
\PageSep{216}
makes the extent of each vibration smaller
than that of the preceding by a constant percentage
of that extent. This combination
of the idea of ``periodicity'' (which requires
\index{Periodicity}%
the sine or cosine for its symbolism) and of
``constant percentage'' (which requires the
exponential function for its symbolism) is the
reason for the form of this function, namely,
its form as a product of a sine-function into
an exponential function.
\PageSep{217}


\Chapter{XV}{The Differential Calculus}

\First{The} invention of the differential calculus
\index{Differential Calculus|EtSeq}%
marks a crisis in the history of mathematics.
The progress of science is divided between
periods characterized by a slow accumulation
of ideas and periods, when, owing to the new
material for thought thus patiently collected,
some genius by the invention of a new method
or a new point of view, suddenly transforms
the whole subject on to a higher level. These
contrasted periods in the progress of the
history of thought are compared by Shelley
to the formation of an avalanche.
\begin{verse}
\footnotesize
\index{Shelley (quotation from)}%
The sun-awakened avalanche! whose mass, \\
Thrice sifted by the storm, had gathered there \\
Flake after flake,---in heaven-defying minds \\
As thought by thought is piled, till some great truth \\
Is loosened, and the nations echo round, \\
\dotfill
\end{verse}

The comparison will bear some pressing.
The final burst of sunshine which awakens
the avalanche is not necessarily beyond comparison
in magnitude with the other powers
of nature which have presided over its slow
\PageSep{218}
formation. The same is true in science. The
genius who has the good fortune to produce
the final idea which transforms a whole
region of thought, does not necessarily excel
all his predecessors who have worked at the
preliminary formation of ideas. In considering
the history of science, it is both silly and
ungrateful to confine our admiration with a
gaping wonder to those men who have made
the final advances towards a new epoch\Add{.}

In the particular instance before us, the
\index{Leibniz|EtSeq}%
\index{Newton|EtSeq}%
subject had a long history before it assumed
its final form at the hands of its
two inventors. There are some traces of its
methods even among the Greek mathematicians,
and finally, just before the actual
production of the subject, Fermat (born 1601~\AD,
\index{Fermat}%
and died 1665~\AD), a distinguished
French mathematician, had so improved on
previous ideas that the subject was all but
created by him. Fermat, also, may lay
claim to be the joint inventor of coordinate
geometry in company with his contemporary
and countryman, Descartes. It was, in fact,
\index{Descartes}%
Descartes from whom the world of science
received the new ideas, but Fermat had certainly
arrived at them independently.

We need not, however, stint our admiration
either for Newton or for Leibniz. Newton
was a mathematician and a student of
physical science, Leibniz was a mathematician
\PageSep{219}
and a philosopher, and each of them
in his own department of thought was one of
the greatest men of genius that the world
has known. The joint invention was the
occasion of an unfortunate and not very
creditable dispute. Newton was using the
methods of Fluxions, as he called the subject,
\index{Fluxions}%
in~1666, and employed it in the composition
of his \Title{Principia}, although in the work as
printed any special algebraic notation is
avoided. But he did not print a direct statement
of his method till~1693. Leibniz published
his first statement in~1684. He was
accused by Newton's friends of having got
it from a MS. by Newton, which he had been
shown privately. Leibniz also accused Newton
of having plagiarized from him. There
is now not very much doubt but that both
should have the credit of being independent
discoverers. The subject had arrived at a
stage in which it was ripe for discovery, and
there is nothing surprising in the fact that
two such able men should have independently
hit upon it.

These joint discoveries are quite common
in science. Discoveries are not in general
made before they have been led up to
by the previous trend of thought, and by
that time many minds are in hot pursuit
of the important idea. If we merely keep
to discoveries in which Englishmen are
\PageSep{220}
concerned, the simultaneous enunciation of
the law of natural selection by Darwin and
\index{Darwin}%
Wallace, and the simultaneous discovery of
\index{Wallace}%
Neptune by Adams and the French astronomer,
\index{Adams}%
Leverrier, at once occur to the mind.
\index{Leverrier}%
The disputes, as to whom the credit ought to
be given, are often influenced by an unworthy
spirit of nationalism. The really inspiring
reflection suggested by the history of mathematics
is the unity of thought and interest
among men of so many epochs, so many nations,
and so many races. Indians, Egyptians,
Assyrians, Greeks, Arabs, Italians, Frenchmen,
Germans, Englishmen, and Russians, have
all made essential contributions to the progress
of the science. Assuredly the jealous
exaltation of the contribution of one particular
nation is not to show the larger spirit.

The importance of the differential calculus
\index{Rate of Increase of Functions|EtSeq}%
arises from the very nature of the subject,
which is the systematic consideration of the
rates of increase of functions. This idea is
immediately presented to us by the study of
nature; velocity is the rate of increase of the
distance travelled, and acceleration is the
rate of increase of velocity. Thus the fundamental
idea of change, which is at the basis of
our whole perception of phenomena, immediately
suggests the enquiry as to the rate of
change. The familiar terms of ``quickly''
and ``slowly'' gain their meaning from a tacit
\PageSep{221}
reference to rates of change. Thus the differential
calculus is concerned with the very
key of the position from which mathematics
can be successfully applied to the explanation
of the course of nature.

This idea of the rate of change was certainly
in Newton's mind, and was embodied in the
\Figure{32}
language in which he explained the subject.
It may be doubted, however, whether this
point of view, derived from natural phenomena,
was ever much in the minds of the preceding
mathematicians who prepared the subject
for its birth. They were concerned with the
more abstract problems of drawing tangents
\index{Tangents}%
to curves, of finding the lengths of curves, and
of finding the areas enclosed by curves. The
\PageSep{222}
last two problems, of the rectification of curves
and the quadrature of curves as they are
named, belong to the Integral Calculus, which
\index{Integral Calculus}%
is however involved in the same general subject
as the Differential Calculus.

The introduction of coordinate geometry
\index{Tangents}%
makes the two points of view coalesce. For
(\Chg{cf.}{\cf}\ \Fig[fig.]{32}) let $AQP$ be any curved line and let
$PT$ be the tangent at the point~$P$ on it. Let
the axes of coordinates be $OX$ and~$OY$; and
let $y = f(x)$ be the equation to the curve, so that
$OM = x$, and $PM = y$. Now let $Q$ be any
moving point on the curve, with coordinates
$x_{1}$,~$y_{1}$; then $y_{1} = f(x_{1})$. And let $Q'$ be the point
on the tangent with the same abscissa~$x_{1}$;
suppose that the coordinates of~$Q'$ are $x_{1}$ and~$y'$.
Now suppose that $N$~moves along the
axis~$OX$ from left to right with a uniform
velocity; then it is easy to see that the ordinate~$y'$
of the point~$Q'$ on the tangent~$TP$ also
increases uniformly as $Q'$~moves along the
tangent in a corresponding way. In fact it is
easy to see that the ratio of the rate of increase
of~$Q'N$ to the rate of increase of~$ON$ is in the
ratio of $Q'N$ to~$TN$, which is the same at all
points of the straight line. But the rate of
increase of~$QN$, which is the rate of increase
of~$f(x_{1})$, varies from point to point of the curve
so long as it is not straight. As $Q$~passes
through the point~$P$, the rate of increase of~$f(x_{1})$
(where $x_{1}$~coincides with~$x$ for the moment)
\PageSep{223}
is the same as the rate of increase of~$y'$ on the
tangent at~$P$. Hence, if we have a general
method of determining the rate of increase
of a function~$f(x)$ of a variable~$x$, we can
determine the slope of the tangent at any
point $(x, y\Typo{,}{})$ on a curve, and thence can
draw it. Thus the problems of drawing tangents
to a curve, and of determining the
rates of increase of a function are really
identical.

It will be noticed that, as in the cases of
Conic Sections and Trigonometry, the more
artificial of the two points of view is the one
in which the subject took its rise. The really
fundamental aspect of the science only rose
into prominence comparatively late in the
day. It is a well-founded historical generalization,
that the last thing to be discovered
in any science is what the science is really
about. Men go on groping for centuries,
guided merely by a dim instinct and a puzzled
curiosity, till at last ``some great truth is
loosened.''

Let us take some special cases in order to
familiarize ourselves with the sort of ideas
which we want to make precise. A train is
in motion---how shall we determine its velocity
at some instant, let us say, at noon? We can
take an interval of five minutes which includes
noon, and measure how far the train has gone
in that period. Suppose we find it to be five
\PageSep{224}
miles, we may then conclude that the train
was running at the rate of $60$~miles per~hour.
But five miles is a long distance, and we
cannot be sure that just at noon the train
was moving at this pace. At noon it may
have been running $70$~miles per~hour, and
afterwards the \Typo{break}{brake} may have been put on.
It will be safer to work with a smaller interval,
say one minute, which includes noon, and to
measure the space traversed during that
period. But for some purposes greater
accuracy may be required, and one minute
may be too long. In practice, the necessary
inaccuracy of our measurements makes it
useless to take too small a period for measurement.
But in theory the smaller the period
the better, and we are tempted to say that
for ideal accuracy an infinitely small period
is required. The older mathematicians, in
particular Leibniz, were not only tempted,
but yielded to the temptation, and did say
it. Even now it is a useful fashion of speech,
provided that we know how to interpret it
into the language of common sense. It is
curious that, in his exposition of the foundations
of the calculus, Newton, the natural
scientist, is much more philosophical than
Leibniz, the philosopher, and on the other
hand, Leibniz provided the admirable notation
which has been so essential for the progress
of the subject.
\PageSep{225}

Now take another example within the region
of pure mathematics. Let us proceed to find
the rate of increase of the function~$x^{2}$ for
any value~$x$ of its argument. We have not
yet really defined what we mean by rate of
increase. We will try and grasp its meaning
in relation to this particular case. When $x$~increases
to $x + h$, the function~$x^{2}$ increases to
$(x + h)^{2}$; so that the total increase has been
$(x + h)^{2} - x^{2}$, due to an increase~$h$ in the argument.
Hence throughout the interval $x$~to
$(x + h)$ the average increase of the function per
unit increase of the argument is $\dfrac{(x + h)^{2} - x^{2}}{h}$.
But
\[
(x + h)^{2} = x^{2} + 2hx + h^{2},
\]
and therefore
\[
\frac{(x + h)^{2} - x^{2}}{h} = \frac{2hx + h^{2}}{h} = 2x + h.
\]
Thus $2x + h$ is the average increase of the
function~$x^{2}$ per unit increase in the argument,
the average being taken over by the interval
$x$~to~$x + h$. But $2x + h$ depends on~$h$, the size
of the interval. We shall evidently get what
we want, namely the \emph{rate} of increase at the
value~$x$ of the argument, by diminishing~$h$
more and more. Hence \emph{in the limit} when $h$~has
\PageSep{226}
\index{Infinitely Small Quantities|EtSeq}%
\emph{decreased indefinitely}, we say that $2x$~is the
rate of increase of~$x^{2}$ at the value~$x$ of the
argument.

Here again we are apparently driven up
against the idea of infinitely small quantities
in the use of the words ``in the limit when $h$~has
decreased indefinitely.'' Leibniz held that,
mysterious as it may sound, there were actually
existing such things as infinitely small
quantities, and of course infinitely small numbers
corresponding to them. Newton's language
and ideas were more on the modern
lines; but he did not succeed in explaining
the matter with such explicitness so as to be
evidently doing more than explain Leibniz's
ideas in rather indirect language. The real
explanation of the subject was first given by
Weierstrass and the Berlin School of mathematicians
\index{Weierstrass}%
about the middle of the nineteenth
century. But between Leibniz and Weierstrass
a copious literature, both mathematical
and philosophical, had grown up round these
mysterious infinitely small quantities which
mathematics had discovered and philosophy
proceeded to explain. Some philosophers,
\index{Berkeley, Bishop}%
Bishop Berkeley, for instance, correctly denied
the validity of the whole idea, though for
reasons other than those indicated here. But
the curious fact remained that, despite all
criticisms of the foundations of the subject,
there could be no doubt but that the mathematical
\PageSep{227}
procedure was substantially right. In
fact, the subject was right, though the explanations
were wrong. It is this possibility of
being right, albeit with entirely wrong explanations
as to what is being done, that so
often makes external criticism---that is so far
as it is meant to stop the pursuit of a method---singularly
barren and futile in the progress of
science. The instinct of trained observers,
and their sense of curiosity, due to the fact
that they are obviously getting at something,
are far safer guides. Anyhow the general
effect of the success of the Differential Calculus
was to generate a large amount of bad philosophy,
centring round the idea of the infinitely
small. The relics of this verbiage
may still be found in the explanations of
many elementary mathematical text-books on
the Differential Calculus. It is a safe rule to
apply that, when a mathematical or philosophical
author writes with a misty profundity,
he is talking nonsense.
\medskip

Newton would have phrased the question
\index{Limit of a Function|EtSeq}%
by saying that, as $h$~approaches zero, in the
limit $2x + h$ becomes~$2x$. It is our task so to
explain this statement as to show that it does
not in reality covertly assume the existence
of Leibniz's infinitely small quantities. In
reading over the Newtonian method of statement,
it is tempting to seek simplicity by
\PageSep{228}
saying that $2x + h$ is~$2x$, when $h$~is zero. But
this will not do; for it thereby abolishes the
interval from $x$ to~$x + h$, over which the average
increase was calculated. The problem is, how
to keep an interval of length~$h$ over which to
calculate the average increase, and at the same
time to treat~$h$ as if it were zero. Newton did
this by the conception of a limit, and we now
\index{Weierstrass}%
proceed to give Weierstrass's explanation of
its real meaning.

In the first place notice that, in discussing
$2x + h$, we have been considering~$x$ as fixed in
value and $h$~as varying. In other words $x$~has
been treated as a ``constant'' variable,
or parameter, as explained in \ChapRef{IX}.;
and we have really been considering $2x + h$ as
a function of the argument~$h$. Hence we can
generalize the question on hand, and ask
what we mean by saying that the function~$f(h)$
tends to the limit~$l$, say, as its argument~$h$
tends to the value zero. But again we shall
see that the special value \emph{zero} for the argument
does not belong to the essence of the subject;
and again we generalize still further, and ask,
what we mean by saying that the function~$f(h)$
tends to the limit~$l$ as $h$~tends to the value~$a$.

Now, according to the Weierstrassian explanation
the whole idea of $h$~tending to the
value~$a$, though it gives a sort of metaphorical
picture of what we are driving at, is really off
the point entirely. Indeed it is fairly obvious
\PageSep{229}
that, as long as we retain anything like ``$h$~tending
to~$a$,'' as a fundamental idea, we are
really in the clutches of the infinitely small;
for we imply the notion of $h$~being infinitely
near to~$a$. This is just what we want to get
rid of.

Accordingly, we shall yet again restate our
phrase to be explained, and ask what we
mean by saying that the limit of the function~$f(h)$
at~$a$ is~$l$.

The limit of~$f(h)$ at~$a$ is a property of the
\index{Standard of Approximation|EtSeq}%
neighbourhood of~$a$, where ``neighbourhood''
is used in the sense defined in \ChapRef{XI}.\
during the discussion of the continuity of
functions. The value of the function~$f(h)$ at~$a$
is~$f(a)$; but the limit is distinct in idea
from the value, and may be different from
it, and may exist when the value has not
been defined. We shall also use the term
``standard of approximation'' in the sense
in which it is defined in \ChapRef{XI}. In
fact, in the definition of ``continuity'' given
towards the end of that chapter we have
practically defined a limit. The definition of
a limit is:---

A function~$f(x)$ has the limit~$l$ at a value~$a$
of its argument~$x$, when in the neighbourhood
of~$a$ its values approximate to~$l$ within
\emph{every} standard of approximation.

Compare this definition with that already
given for continuity, namely:---
\PageSep{230}

A function~$f(x)$ is continuous at a value~$a$
of its argument, when in the neighbourhood
of~$a$ its values approximate to its value at~$a$
within \emph{every} standard of approximation.

It is at once evident that a function is continuous
at~$a$ when (i)~it possesses a limit at~$a$,
and (ii)~that limit is equal to its value at~$a$.
Thus the illustrations of continuity which
have been given at the end of \ChapRef{XI}.\ are
illustrations of the idea of a limit, namely,
they were all directed to proving that $f(a)$~was
the limit of~$f(x)$ at~$a$ for the functions
considered and the value of~$a$ considered. It
is really more instructive to consider the
limit at a point where a function is not continuous.
For example, consider the function
of which the graph is given in \Fig[fig.]{20} of \ChapRef{XI}.
This function~$f(x)$ is defined to have
the value~$1$ for all values of the argument
except the integers $0$,~$1$, $2$, $3$,~etc., and for these
integral values it has the value~$0$. Now let
us think of its limit when $x = 3$. We notice
that in the definition of the limit the value
of the function at~$a$ (in this case, $a = 3$) is excluded.
But, excluding~$f(3)$, the values of~$f(x)$,
when $x$~lies within any interval which
(i)~contains $3$ not as an end-point, and (ii)~does
not extend so far as $2$ and~$4$, are all
equal to~$1$; and hence these values approximate
to~$1$ within every standard of approximation.
Hence $1$~is the limit of~$f(x)$ at the
\PageSep{231}
value~$3$ of the argument~$x$, but by definition
$f(3) = 0$.

This is an instance of a function which
possesses both a value and a limit at the
value~$3$ of the argument, but the value is not
equal to the limit. At the end of \ChapRef{XI}.\
the function~$x^{2}$ was considered at the
value~$2$ of the argument. Its value at~$2$ is~$2^{2}$,
\ie~$4$, and it was proved that its limit is also~$4$.
Thus here we have a function with a
value and a limit which are equal.

Finally we come to the case which is essentially
important for our purposes, namely, to
a function which possesses a limit, but no
defined value at a certain value of its argument.
We need not go far to look for
such a function, $\dfrac{2x}{x}$~will serve our purpose.
Now in any mathematical book, we might
find the equation, $\dfrac{2x}{x} = 2$, written without
hesitation or comment. But there is a difficulty
in this; for when $x$~is zero, $\dfrac{2x}{x} = \dfrac{0}{0}$; and
$\dfrac{0}{0}$~has no defined meaning. Thus the value
of the function~$\dfrac{2x}{x}$ at $x = 0$ has no defined
\PageSep{232}
meaning. But for every other value of~$x$,
the value of the function~$\dfrac{2x}{x}$ is~$2$. Thus the
limit of~$\dfrac{2x}{x}$ at $x = 0$ is~$2$, and it has no value
at $x = 0$. Similarly the limit of~$\dfrac{x^{2}}{x}$ at $x = a$ is~$a$
whatever $a$~may be, so that the limit of~$\dfrac{x^{2}}{x}$
at $x = 0$ is~$0$. But the value of~$\dfrac{x^{2}}{x}$ at $x = 0$
takes the form~$\dfrac{0}{0}$, which has no defined
meaning. Thus the function~$\dfrac{x^{2}}{x}$ has a limit
but no value at~$0$.

We now come back to the problem from
which we started this discussion on the nature
of a limit. How are we going to define the
rate of increase of the function~$x^{2}$ at any
value~$x$ of its argument. Our answer is that
this rate of increase is the limit of the function
$\dfrac{(x + h)^{2} - x^{2}}{h}$ at the value zero for its
argument~$h$. (Note that $x$~is here a ``constant.'')
Let us see how this answer works
\PageSep{233}
in the light of our definition of a limit. We
have
\[
\frac{(x + h)^{2} - x^{2}}{h}
  = \frac{2hx + h^{2}}{h}
  = \frac{h(2x + h)}{h}\Add{.}
\]

Now in finding the limit of~$\dfrac{h(2x + h)}{h}$ at the
value~$0$ of the argument~$h$, the value (if any)
of the function at $h = 0$ is excluded. But for
all values of~$h$, except $h = 0$, we can divide
through by~$h$. Thus the limit of~$\dfrac{h(2x + h)}{h}$ at
$h = 0$ is the same as that of $2x + h$ at $h = 0$.
Now, whatever standard of approximation~$k$
we choose to take, by considering the interval
from $-\frac{1}{2}k$ to~$+\frac{1}{2}k$ we see that, for values of~$h$
which fall within it, $2x + h$~differs from~$2x$
by less than~$\frac{1}{2}k$, that is by less than~$k$. This
is true for \emph{any} standard~$k$. Hence in the neighbourhood
of the value~$0$ for~$h$, $2x + h$~approximates
to~$2x$ within \emph{every} standard of approximation,
and therefore $2x$~is the limit of~$2x + h$
at $h = 0$. Hence by what has been said above
$2x$~is the limit of $\dfrac{(x + h)^{2} - x^{2}}{h}$ at the value~$0$
for~$h$. It follows, therefore, that $2x$~is what
we have called the rate of increase of~$x^{2}$ at
the value~$x$ of the argument. Thus this
method conducts us to the same rate of increase
\PageSep{234}
for~$x^{2}$ as did the Leibnizian way of
making $h$~grow ``infinitely small.''

The more abstract terms ``differential coefficient,''
\index{Differential Coefficient}%
or ``derived function,'' are generally
\index{Derived Function}%
used for what we have hitherto called the
``rate of increase'' of a function. The
general definition is as follows: the differential
coefficient of the function~$f(x)$ is the
limit, if it exist, of the function $\dfrac{f(x + h) - f(x)}{h}$
of the argument~$h$ at the value~$0$ of its argument.

How have we, by this definition and the
subsidiary definition of a limit, really managed
to avoid the notion of ``infinitely small numbers''
which so worried our mathematical
forefathers? For them the difficulty arose
because on the one hand they had to use an
interval $x$~to $x + h$ over which to calculate
the average increase, and, on the other hand,
they finally wanted to put $h = 0$. The result
was they seemed to be landed into the notion
of an existent interval of zero size. Now
how do we avoid this difficulty? In this
way---we use the notion that corresponding
to \emph{any} standard of approximation, \emph{some} interval
with such and such properties can be
found. The difference is that we have
\index{Variable, The}%
grasped the importance of the notion of ``the
variable,'' and they had not done so. Thus,
\PageSep{235}
at the end of our exposition of the essential
notions of mathematical analysis, we are led
back to the ideas with which in \ChapRef{II}.\
we commenced our enquiry---that in mathematics
the fundamentally important ideas
are those of ``\emph{some} things'' and ``\emph{any}
things.''
\PageSep{236}


\Chapter{XVI}{Geometry}

\First{Geometry}, like the rest of mathematics, is
\index{Geometry|EtSeq}%
abstract. In it the properties of the shapes
and relative positions of things are studied.
But we do not need to consider who is observing
the things, or whether he becomes acquainted
with them by sight or touch or
hearing. In short, we ignore all particular
sensations. Furthermore, particular things
such as the Houses of Parliament, or the
terrestrial globe are ignored. Every proposition
refers to any things with such and
such geometrical properties. Of course it
helps our imagination to look at particular
examples of spheres and cones and triangles
and squares. But the propositions do not
merely apply to the actual figures printed in
the book, but to any such figures.

Thus geometry, like algebra, is dominated
by the ideas of ``any'' and ``some'' things.
Also, in the same way it studies the interrelations
of sets of things. For example, consider
any two triangles $ABC$ and~$DEF$.
\PageSep{237}

What relations must exist between some of
\index{Triangle}%
the parts of these triangles, in order that the
triangles may be in all respects equal? This
is one of the first investigations undertaken
in all elementary geometries. It is a study
\Figure{33}
of a certain set of possible correlations between
the two triangles. The answer is that
the triangles are in all respects equal, if:---
Either, (a)~Two sides of the one and the included
angle are respectively equal to two
sides of the other and the included angle:

Or, (b)~Two angles of the one and the side
joining them are respectively equal to two
angles of the other and the side joining them:

Or, (c)~Three sides of the one are respectively
equal to three sides of the other.

This answer at once suggests a further enquiry.
What is the nature of the correlation
between the triangles, when the three angles
of the one are respectively equal to the three
angles of the other? This further investigation
leads us on to the whole theory of similarity
\index{Similarity}%
\PageSep{238}
(\Chg{cf.}{\cf}\ \ChapRef{XIII}.), which is another
type of correlation.

Again, to take another example, consider
the internal structure of the triangle~$ABC$.
Its sides and angles are inter-related---the
greater angle is opposite to the greater side,
and the base angles of an isosceles triangle
are equal. If we proceed to trigonometry
this correlation receives a more exact determination
in the familiar shape
\[
\frac{\sin A}{a} = \frac{\sin B}{b} = \frac{\sin C}{c},
\]
$a^{2} = b^{2} + c^{2} - 2bc \cos A$, with two similar
formulæ.

Also there is the still simpler correlation
between the angles of the triangle, namely,
that their sum is equal to two right angles;
and between the three sides, namely, that the
sum of the lengths of any two is greater than
the length of the third\Add{.}

Thus the true method to study geometry is
to think of interesting simple figures, such as
the triangle, the parallelogram, and the circle,
and to investigate the correlations between
their various parts. The geometer has in his
mind not a detached proposition, but a figure
with its various parts mutually inter-dependent.
Just as in algebra, he generalizes the
triangle into the polygon, and the side into
\PageSep{239}
the conic section. Or, pursuing a converse
route, he classifies triangles according as they
are equilateral, isosceles, or scalene, and
polygons according to their number of sides,
and conic sections according as they are hyperbolas,
ellipses, or parabolas.

The preceding examples illustrate how the
fundamental ideas of geometry are exactly
the same as those of algebra; except that
algebra deals with numbers and geometry
with lines, angles, areas, and other geometrical
entities. This fundamental identity
is one of the reasons why so many geometrical
truths can be put into an algebraic dress.
Thus if $A$,~$B$, and~$C$ are the numbers of degrees
respectively in the angles of the triangle~$ABC$,
the correlation between the angles is represented
by the equation
\[
A + B + C = 180°;
\]
and if $a$,~$b$,~$c$ are the number of feet respectively
in the three sides, the correlation between the
sides is represented by $a < b + c$, $b < c + a$,
$c < a + b$. Also the trigonometrical formulæ
quoted above are other examples of the same
\index{Variable, The}%
fact. Thus the notion of the variable and
the correlation of variables is just as essential
in geometry as it is in algebra.

But the parallelism between geometry and
algebra can be pushed still further, owing to
the fact that lengths, areas, volumes, and
\PageSep{240}
angles are all measurable; so that, for example,
the size of any length can be determined
by the number (not necessarily integral) of
times which it contains some arbitrarily known
unit, and similarly for areas, volumes, and
angles. The trigonometrical formulæ, given
above, are examples of this fact. But it receives
its crowning application in analytical
geometry. This great subject is often misnamed
as Analytical Conic Sections, thereby
\index{Analytical Conic Sections}%
fixing attention on merely one of its subdivisions.
It is as though the great science
of Anthropology were named the Study of
Noses, owing to the fact that noses are a
prominent part of the human body.

Though the mathematical procedures in
geometry and algebra are in essence identical
and intertwined in their development, there
is necessarily a fundamental distinction between
the properties of space and the properties
of number---in fact all the essential difference
between space and number. The ``spaciness''
of space and the ``numerosity'' of
number are essentially different things, and
must be directly apprehended. None of the
applications of algebra to geometry or of
geometry to algebra go any step on the road
to obliterate this vital distinction.

One very marked difference between space
and number is that the former seems to be so
much less abstract and fundamental than the
\PageSep{241}
latter. The number of the archangels can be
counted just because they are things. When
we once know that their names are Raphael,
Gabriel, and Michael, and that these distinct
names represent distinct beings, we know without
further question that there are three of
them. All the subtleties in the world about
the nature of angelic existences cannot alter
this fact, granting the premisses.

But we are still quite in the dark as to their
relation to space. Do they exist in space at
all? Perhaps it is equally nonsense to say
that they are here, or there, or anywhere, or
everywhere. Their existence may simply have
no relation to localities in space. Accordingly,
while numbers must apply to all things,
space need not do so.

The perception of the locality of things
would appear to accompany, or be involved
in many, or all, of our sensations. It is independent
of any particular sensation in the
sense that it accompanies many sensations.
But it is a special peculiarity of the things
which we apprehend by our sensations. The
direct apprehension of what we mean by the
positions of things in respect to each other
is a thing \Foreign{sui generis}, just as are the apprehensions
of sounds, colours, tastes, and smells.
At first sight therefore it would appear that
mathematics, in so far as it includes geometry
in its scope, is not abstract in the sense in
\PageSep{242}
which abstractness is ascribed to it in
\ChapRef{I}.

This, however, is a mistake; the truth being
\index{Abstract Nature of Geometry|EtSeq}%
that the ``spaciness'' of space does not enter
into our geometrical \emph{reasoning} at all. It
enters into the geometrical intuitions of
mathematicians in ways personal and peculiar
to each individual. But what enter into the
reasoning are merely certain properties of
things in space, or of things forming space,
which properties are completely abstract in
the sense in which abstract was defined in
\ChapRef{I}.; these properties do not involve
any peculiar space-apprehension or space-intuition
or space-sensation. They are on
exactly the same basis as the mathematical
properties of number. Thus the space-intuition
which is so essential an aid to the study
of geometry is logically irrelevant: it does
not enter into the premisses when they are
properly stated, nor into any step of the reasoning.
It has the practical importance of an
example, which is essential for the stimulation
of our thoughts. Examples are equally necessary
to stimulate our thoughts on number.
When we think of ``two'' and ``three'' we
see strokes in a row, or balls in a heap, or
some other physical aggregation of particular
things. The peculiarity of geometry is the
fixity and overwhelming importance of the
one particular example which occurs to our
\PageSep{243}
minds. The abstract logical form of the
propositions when fully stated is, ``If any
collections of things have such and such
abstract properties, they also have such and
such other abstract properties.'' But what
appears before the mind's eye is a collection
of points, lines, surfaces, and volumes in the
space: this example inevitably appears, and
is the sole example which lends to the proposition
its interest. However, for all its overwhelming
importance, it is but an example.

Geometry, viewed as a mathematical science,
is a division of the more general science of
order. It may be called the science of dimensional
order; the qualification ``dimensional''
has been introduced because the limitations,
which reduce it to only a part of the general
science of order, are such as to produce the
regular relations of straight lines to planes,
and of planes to the whole of space.

It is easy to understand the practical importance
of space in the formation of the
scientific conception of an external physical
world. On the one hand our space-perceptions
are intertwined in our various sensations
and connect them together. We normally
judge that we touch an object in the same
place as we see it; and even in abnormal
cases we touch it in the same space as we see
it, and this is the real fundamental fact which
ties together our various sensations. Accordingly,
\PageSep{244}
the space perceptions are in a sense the
common part of our sensations. Again it
happens that the abstract properties of space
form a large part of whatever is of spatial
interest. It is not too much to say that to
every property of space there corresponds an
abstract mathematical statement. To take
the most unfavourable instance, a curve may
have a special beauty of shape: but to this
shape there will correspond some abstract
mathematical properties which go with this
shape and no others.

Thus to sum up: (1)~the properties of space
which are investigated in geometry, like those
of number, are properties belonging to things
as things, and without special reference to
any particular mode of apprehension: (2)~Space-perception
accompanies our sensations,
perhaps all of them, certainly many; but it
does not seem to be a necessary quality of
things that they should all exist in one space
or in any space.
\PageSep{245}


\Chapter{XVII}{Quantity}

\First{In} the previous chapter we pointed out
\index{Quantity|EtSeq}%
that lengths are measurable in terms of some
unit length, areas in term of a unit area, and
volumes in terms of a unit volume.

When we have a set of things such as
lengths which are measurable in terms of any
one of them, we say that they are quantities
of the same kind. Thus lengths are quantities
of the same kind, so are areas, and so are
volumes. But an area is not a quantity of
the same kind as a length, nor is it of the
same kind as a volume. Let us think a little
more on what is meant by being measurable,
taking lengths as an example.

Lengths are measured by the foot-rule. By
transporting the foot-rule from place to place
we judge of the equality of lengths. Again,
three adjacent lengths, each of one foot, form
one whole length of three feet. Thus to
measure lengths we have to determine the
equality of lengths and the addition of lengths.
When some test has been applied, such as the
transporting of a foot-rule, we say that the
lengths are equal; and when some process
\PageSep{246}
has been applied, so as to secure lengths being
contiguous and not overlapping, we say that
the lengths have been added to form one
whole length. But we cannot arbitrarily take
any test as the test of equality and any
process as the process of addition. The results
of operations of addition and of judgments
of equality must be in accordance with
certain preconceived conditions. For example,
the addition of two greater lengths must
yield a length greater than that yielded by
the addition of two smaller lengths. These
preconceived conditions when accurately formulated
may be called axioms of quantity.
The only question as to their truth or falsehood
\index{Axioms of Quantity|EtSeq}%
which can arise is whether, when the axioms
are satisfied, we necessarily get what ordinary
people call quantities. If we do not, then
the name ``axioms of quantity'' is ill-judged---that
is all.

These axioms of quantity are entirely abstract,
just as are the mathematical properties
of space. They are the same for all quantities,
and they presuppose no special mode of perception.
The ideas associated with the notion
of quantity are the means by which a continuum
like a line, an area, or a volume can
be split up into definite parts. Then these
parts are counted; so that numbers can be
used to determine the exact properties of a
continuous whole.
\PageSep{247}

Our perception of the flow of time and of
\index{Time|EtSeq}%
the succession of events is a chief example
of the application of these ideas of quantity.
We measure time (as has been said in considering
periodicity) by the repetition of
similar events---the burning of successive
inches of a uniform candle, the rotation of
the earth relatively to the fixed stars, the
rotation of the hands of a clock are all examples
of such repetitions. Events of these
types take the place of the foot-rule in relation
to lengths. It is not necessary to assume
that events of any one of these types are
exactly equal in duration at each recurrence.
What is necessary is that a rule should be
known which will enable us to express the
relative durations of, say, two examples of
some type. For example, we may if we like
suppose that the rate of the earth's rotation
is decreasing, so that each day is longer than
the preceding by some minute fraction of a
second. Such a rule enables us to compare
the length of any day with that of any other
day. But what is essential is that one series
of repetitions, such as successive days, should
be taken as the standard series; and, if the
various events of that series are not taken as
of equal duration, that a rule should be
stated which regulates the duration to be
assigned to each day in terms of the duration
of any other day.
\PageSep{248}

What then are the requisites which such
a rule ought to have? In the first place it
should lead to the assignment of nearly equal
durations to events which common sense
judges to possess equal durations. A rule
which made days of violently different lengths,
and which made the speeds of apparently
similar operations vary utterly out of proportion
to the apparent minuteness of their
differences, would never do. Hence the first
requisite is general agreement with common
sense. But this is not sufficient absolutely
to determine the rule, for common sense is a
rough observer and very easily satisfied. The
next requisite is that minute adjustments of
the rule should be so made as to allow of the
simplest possible statements of the laws of
nature. For example, astronomers tell us
that the earth's rotation is slowing down, so
that each day gains in length by some inconceivably
minute fraction of a second. Their
only reason for their assertion (as stated more
fully in the discussion of periodicity) is that
without it they would have to abandon the
Newtonian laws of motion. In order to keep
\index{Laws of Motion}%
the laws of motion simple, they alter the
measure of time. This is a perfectly legitimate
procedure so long as it is thoroughly
understood.

What has been said above about the abstract
nature of the mathematical properties
\PageSep{249}
of space applies with appropriate verbal
changes to the mathematical properties of
time. A sense of the flux of time accompanies
all our sensations and perceptions, and practically
all that interests us in regard to time
can be paralleled by the abstract mathematical
properties which we ascribe to it.
Conversely what has been said about the two
requisites for the rule by which we determine
the length of the day, also applies to the rule
for determining the length of a yard measure---namely,
the yard measure appears to retain
the same length as it moves about. Accordingly,
any rule must bring out that, apart
from minute changes, it does remain of invariable
length; Again, the second requisite
is this, a definite rule for minute changes
shall be stated which allows of the simplest
expression of the laws of nature. For example,
in accordance with the second requisite
the yard measures are supposed to
expand and contract with changes of temperature
according to the substances which
they are made of.

Apart from the facts that our sensations
are accompanied with perceptions of locality
and of duration, and that lines, areas, volumes,
and durations, are each in their way quantities,
the theory of numbers would be of very
subordinate use in the exploration of the laws
of the Universe, As it is, physical science
\PageSep{250}
reposes on the main ideas of number, quantity,
space, and time. The mathematical
sciences associated with them do not form
the whole of mathematics, but they are the
substratum of mathematical physics as at
present existing.


\BackMatter
\Appendix{Notes}

\Note{A} (\Pageref{60}).---In reading these equations it must be noted
that a bracket is used in mathematical symbolism to
mean that the operations within it are to be performed
first. Thus $(1 + 3) + 2$ directs us first to add $3$ to~$1$, and
then to add~$2$ to the result; and $1 + (3 + 2)$ directs us
first to add $2$ to~$3$, and then to add the result to~$1$. Again
a numerical example of equation~\Eq{(5)} is
\[
2 × (3 + 4) = (2 × 3) + (2 × 4).
\]
We perform first the operations in brackets and obtain
\[
2 × 7 = 6 + 8
\]
which is obviously true.


\Note{B} (\Pageref{136}).---This fundamental ratio~$\dfrac{SP}{PN}$ is called the
eccentricity of the curve. The shape of the curve, as
\index{Eccentricity}%
distinct from its scale or size, depends upon the value of
its eccentricity. Thus it is wrong to think of ellipses
in general or of hyperbolas in general as having in either
case one definite shape. Ellipses with different eccentricities
have different shapes, and their sizes depend
upon the lengths of their major axes. An ellipse with
small eccentricity is very nearly a circle, and an ellipse
of eccentricity only slightly less than unity is a long
flat oval. All parabolas have the same eccentricity and
are therefore of the same shape, though they can be
drawn to different scales.
\PageSep{251}

\Note{C} (\Pageref{204}).---If a series with all its terms positive is
\index{Absolute Convergence}%
\index{Convergence, Absolute}%
convergent, the modified series found by making some
terms positive and some negative according to any
definite rule is also convergent. Each one of the set of
series thus found, including the original series, is called
``absolutely convergent.'' But it is possible for a series
with terms partly positive and partly negative to be
convergent, although the corresponding series with all
its terms positive is divergent. For example, the series
\[
1 - \tfrac{1}{2} + \tfrac{1}{3} - \tfrac{1}{4} + \text{etc.}
\]
is convergent though we have just proved that
\[
1 + \tfrac{1}{2} + \tfrac{1}{3} + \tfrac{1}{4} + \text{etc.}
\]
is divergent. Such convergent series, which are not
absolutely convergent, are much more difficult to deal
with than absolutely convergent series.


\Appendix[Note on the Study of Mathematics]{Bibliography}


\First{The} difficulty that beginners find in the study of this
science is due to the large amount of technical detail which
has been allowed to accumulate in the elementary text-books,
obscuring the important ideas.

The first subjects of study, apart from a knowledge of
arithmetic which is presupposed, must be elementary
geometry and elementary algebra. The courses in both
subjects should be short, giving only the necessary ideas;
the algebra should be studied graphically, so that in
practice the ideas of elementary coordinate geometry are
also being assimilated. The next pair of subjects should
be elementary trigonometry and the coordinate geometry
of the straight line and circle. The latter subject is a
short one; for it really merges into the algebra. The
student is then prepared to enter upon conic sections, a
very short course of geometrical conic sections and a longer
one of analytical conics. But in all these courses great
care should be taken not to overload the mind with more
\PageSep{252}
detail than is necessary for the exemplification of the
fundamental ideas.

The differential calculus and afterwards the integral
calculus now remain to be attacked on the same system.
A good teacher will already have illustrated them by the
consideration of special cases in the course on algebra
and coordinate geometry. Some short book on three-dimensional
geometry must be also read.

This elementary course of mathematics is sufficient for
some types of professional career. It is also the necessary
preliminary for any one wishing to study the subject for
its intrinsic interest. He is now prepared to commence
on a more extended course. He must not, however, hope
to be able to master it as a whole. The science has grown
to such vast proportions that probably no living mathematician
can claim to have achieved this.

Passing to the serious treatises on the subject to be read
\emph{after} this preliminary course, the following may be mentioned:
Cremona's \Title{Pure Geometry} (English Translation,
Clarendon Press, Oxford), Hobson's \Title{Treatise on Trigonometry},
Chrystal's \Title{Treatise on Algebra} (2~volumes), Salmon's
\Title{Conic Sections}, Lamb's \Title{Differential Calculus}, and some
book on \Title{Differential Equations}. The student will probably
not desire to direct equal attention to all these subjects,
but will study one or more of them, according as his interest
dictates. He will then be prepared to select more advanced
works for himself, and to plunge into the higher
parts of the subject. If his interest lies in analysis, he
should now master an elementary treatise on the theory
of Functions of the Complex Variable; if he prefers to
specialize in Geometry, he must now proceed to the
standard treatises on the Analytical Geometry of three
dimensions. But at this stage of his career in learning
he will not require the advice of this note.

I have deliberately refrained from mentioning any
elementary works. They are very numerous, and of
various merits, but none of such outstanding superiority
as to require special mention by name to the exclusion
of all the others.


%[** TN: Index text]
% ** Page 253
\printindex
\iffalse

Abel 156

Abscissa 95

Absolute Convergence 251

Abstract Nature of Geometry|EtSeq 242

Abstractness (\emph{defined}) 9, 13

Adams 220

Addition-Theorem 212

Ahmes 71

Alexander the Great 128, 129

Algebra, Fundamental Laws of 60

Ampere@Ampère 34

Analytical Conic Sections 240

Apollonius of Perga 131, 134

Approximation|EtSeq 197

Arabic Notation|EtSeq 58

Archimedes|EtSeq 37

Argument of a Function 146

Aristotle 30, 42, 128

Astronomy 137, 173, 174

Axes 125

Axioms of Quantity|EtSeq 246

Bacon 156

Ball, W. W. R. 58

Beaconsfield, Lord 41

Berkeley, Bishop 226

Bhaskara 58

Cantor, Georg 79

Circle 120, 130

Circle@Circle|EtSeq 180

Circular Cylinder 143

Clerk Maxwell 34, 35

Columbus 122

Compact Series 76

Complex Quantities 109

Conic Sections|EtSeq 128

Constants 69, 117

Continuous Functions@Continuous Functions|EtSeq 150

Continuous Functions@Continuous Functions (\emph{defined}) 162

Convergence, Absolute 251

Convergent|EtSeq 203

Coordinate Geometry|EtSeq 112

Coordinates 95

Copernicus 45, 137

%[** TN: Entry italicized in the original, "Sine" not italicized]
Cosine|EtSeq 182

Coulomb 33

Cross Ratio 140

Darwin 138, 220

Derived Function 234

Descartes 95, 113, 116, 122, 218

Differential Calculus|EtSeq 217

Differential Coefficient 234

Directrix 135

Discontinuous Functions|EtSeq 150

Distance 30

Divergent|EtSeq 203
% ** Page 254

Dynamical Explanation 13, 14

Dynamical Explanation@Dynamical Explanation|EtSeq 47

Dynamics 30

Dynamics@Dynamics|EtSeq 43

Eccentricity 250

Electric Current 33

Electricity|EtSeq 32

Electromagnetism|EtSeq 31

Ellipse 45, 120

Ellipse@Ellipse|EtSeq 130

Euclid 114

Exponential Series|EtSeq 211

Faraday 34

Fermat 218

Fluxions 219

Focus 120, 135

Force 30

Form, Algebraic@Form, Algebraic|EtSeq 66

Form, Algebraic 82, 117

Fourier's Theorem 191

Fractions|EtSeq 71

Franklin 32, 122

Function|EtSeq 144

Galileo@Galileo|EtSeq 42

Galileo 30, 122

Galvani 33

Generality in Mathematics 82

Geometrical Series|EtSeq 206

Geometry 36

Geometry@Geometry|EtSeq 236

Gilbert, Dr. 32

Graphs|EtSeq 148

Gravitation 29, 139

Halley 139

Harmonic Analysis 192

Harriot, Thomas 66

Herz 35

Hiero 38

Hipparchus 173

Hyperbola|EtSeq 131

Imaginary Numbers|EtSeq 87

Imaginary Quantities 109

Incommensurable Ratios|EtSeq 72

Infinitely Small Quantities|EtSeq 226

Integral Calculus 222

Interval|EtSeq 158

Kepler 45, 46, 137, 138

Kepler's Laws 138

Laputa 10

Laws of Motion@Laws of Motion|EtSeq 167

Laws of Motion 248

Leibniz 16

Leibniz@Leibniz|EtSeq 218

Leonardo da Vinci 42

Leverrier 220

Light 35

Limit of a Function|EtSeq 227

Limit of a Series|EtSeq 199

Limits 77

Locus@Locus|EtSeq 121

Locus 141

Macaulay 156

Malthus 138

Marcellus 37

Mass 30

Mechanics 46

Menaechmus 128, 129

Motion, First Law of 43
% ** Page 255

Neighbourhood|EtSeq 159

Newton 10, 16, 30, 34, 37, 38, 43, 46, 139

Newton@Newton|EtSeq 218

Non-Uniform Convergence|EtSeq 208

Normal Error, Curve of 214

Oersted@Öersted 34

Order|EtSeq 194

Order, Type of@Order, Type of|EtSeq 75

Order, Type of 196

Ordered Couples|EtSeq 93

Ordinate 95

Origin 95, 126

Pappus 135, 136

Parabola|EtSeq 131

Parallelogram Law@Parallelogram Law|EtSeq 51

Parallelogram Law 99, 126

Parameters 69, 117

Pencils 140

Period 170

Period@Period|EtSeq 189

Periodicity@Periodicity|EtSeq 164

Periodicity 188, 216

Pitt, William 194

Pizarro 122

Plutarch 37

Positive and Negative Numbers|EtSeq 83

Projective Geometry 139

Ptolemy 137, 173

Pythagoras 18

Quantity|EtSeq 245

Rate of Increase of Functions|EtSeq 220

Ratio|EtSeq 72

Real Numbers|EtSeq 73

Rectangle 57

Relations between Variables|EtSeq 18

Resonance 170, 171

Rosebery, Lord 194

Scale of a Map 178

Seidel 210

Series|EtSeq 74, 194

Shelley (quotation from) 217

Similarity@Similarity|EtSeq 177

Similarity 237

Sine|EtSeq 182

Specific Gravity 41

Squaring the Circle 187

Standard of Approximation|EtSeq 159, 201, 229

Steps@Steps|EtSeq 79

Steps 96

Stifel 85

Stokes, Sir George 210

Sum to Infinity|EtSeq 201

Surveys|EtSeq 176

Swift 10

Tangents 221, 222

Taylor's Theorem 156, 157

Time|EtSeq 166, 247

Transportation, Vector of|EtSeq 54

Triangle@Triangle|EtSeq 176

Triangle 237

Triangulation 177

Trigonometry|EtSeq 173

Uniform Convergence|EtSeq 208

Unknown, The 17, 23
% ** Page 256

Value of a Function 146

Variable, The 18, 24, 49, 82, 234, 239

Variable Function 147

Vectors@Vectors|EtSeq 51

Vectors 85, 96

Vertex 134

Volta 33

Wallace 220

Weierstrass 156, 226, 228

Zero@Zero|EtSeq 63

Zero 103
%[** TN: End of index]
\fi

% Printed by Hazell, Walton \& Viney, Ld., London and Aylesbury.
%[** TN: Raw OCR output of book catalog follows]
\iffalse

The

Home University
Library



of Modern
Knowledge



jj Comprehensive Series of New
and Specially Written (Books



EDITORS:

PROF. GILBERT MURRAY, D.Litt., LL.D., F.B.A.
HERBERT FISHER, M.A., F.B.A.
PKOF. J. ARTHUR THOMSON, M.A.
PROF. WM. T. BREWSTER, M.A.

The Home University Library

" Is without the slightest doubt the pioneer in supplying serious literature
for a large section of the public who are interested in the liberal educa-
tion of the State." The Daily Mail.

" It is a thing very favourable to the real success of The Home
University Library that its volumes do not merely attempt to feed
ignorance with knowledge. The authors noticeably realise that the
simple willing appetite of sharp-set ignorance is not specially common
nowadays; what is far more common is a hunger which has been
partially but injudiciously filled, with more or less serious results of
indigestion. The food supplied is therefore frequently medicinal as
well as nutritious; and this is certainly what the time requires. "
Manchester Guardian.

"Each volume represents a three-hours' traffic with the talking-power
of a good brain, operating with the ease and interesting freedom of a
specialist dealing with his own subject. ... A series which promises to
perform a real social service." The Times.

"We can think of no series now being issued which better deserves
support." The Observer.



We think if they were given as prizes in place of the more costly

di:

id prol
series they might well take:



ispensed on prize days, the pupils would
rofit. If the publishers want a motto for the



rubbish that is wont to be

find more pleasure and

series they might well take: ' Infinite riches in a little room.'" Irish

Journal of Education,

" The scheme was successful at the start because it met a want
among earnest readers; but its wider and* sustained success, surely,
comes from the fact that it has to a large extent created and certainly
refined the taste by which it is appreciated." Daily Chronicle.

" Here is the world's learning in little, and none too poor t&lt;
house-room!" Daily Telegraph.



to give it



]/- net
in cloth


256 Pages


2/6 net
in leather



History and (geography



3. THE FRENCH REVOLUTION

By HILAIKE BELLOC, M.A. (With Maps.) "It is coloured with all the
militancy of the author's temperament." Daily News.

4. HISTORY OF WAR AND PEACE

By G. H. FERRIS. The Rt. Hon. JAMES BRYCE writes: " I have read it with
much interest and pleasure, admiring the skill with which you have managed
to compress so many facts and views into so small a volume."

8. POLAR EXPLORATION

By Dr W. S. BRUCE, F.R.S.E., Leader of the "Scotia" Expedition. (With
Maps.) "A very freshly written and interesting narrative." The Times.
"A fascinating book." Portsmouth Times.

12. THE OPENING-UP OF AFRICA

By Sir H. H. JOHNSTON. G.C.M.G., K.C.B., D.Sc., F.Z.S. (With Maps.)
" The Home University Library is much enriched by this excellent work."
Daily Mail.

13. MEDIAEVAL EUROPE

By H. W. C. DAVIS, M.A. (With Maps.) "One more illustration of the
fact that it takes a complete master of the subject to write briefly upon it."
Manchester Guardian.

14. THE PAPACY \&* MODERN TIMES (1303-1870)

By WILLIAM BARRY, D.D. "Dr Barry has a wide range of knowledge
and an artist's power of selection." Manchester Guardian.

23. HISTORY OF OUR TIME, 1885-1911

By G. P. GOOCH, M.A. " Mr Gooch contrives to breathe vitality into his story,
and to give us the flesh as well as the bones of recent happenings." Observer.

25. THE CIVILISATION OF CHINA

By H. A. GILES, LL.D., Professor of Chinese in the University of Cambridge.
"In all the mass of facts, Professor Giles never becomes dull. He is always
ready with a ghost story or a street adventure for the reader's recreation."
Spectator.

29. THE DA WN OF HISTORY

By J.L.MYRES, M. A., F.S. A., Wykeham Professor of Ancient History, Oxford.
"There is not a page in it that is not suggestive." Manchester Guardian.

33. THE HISTORY OF ENGLAND:
A Study in Political Evolution.

By Prof. A. F. POLLARD, M.A. With a Chronological Table. " It takes its
place at once among the authoritative works on English history." Observer.

34. CANADA

By A. G. BRADLEY. " Who knows Canada, better than Mr A. G. Bradley? "
Daily Chronicle. "The volume makes an immediate appeal to the man who
wants to know something vivid and true about Canada." Canadian Gazette.



37. PEOPLES 6* PROBLEMS OF INDIA

By Sir T. W. HOLDERNESS, K.C.S.I., Secretary of the Revenue, Statistics,
! and Commerce Department of the India Office. "Just the book which news-
paper readers require to-day, and a marvel of comprehensiveness." Pall
\ Mall Gazette.

42. ROME

By W. WARDE FOWLHR, M.A. " A masterly sketch of Roman character and
of what it did for the world." The Spectator. "It has all the lucidity and
charm of presentation we expect from this writer." Manchester Guardian.

48. THE AMERICAN CIVIL WAR

By F. L. PAXSON, Professor of American History, Wisconsin University.
(With Maps.) "A stirring study." The Guardian.

51. WARFARE IN BRITAIN

By HILAIRE BELLOC, M.A. An account of how and where great battles of the
past were fought on British soil, the roads and physical conditions determining
the island's strategy, the castles, walled towns, etc.

55. MASTER MARINERS

By J. R. SPEARS. The romance of the sea, the great voyages of discovery,
naval battles, the heroism of the sailor, and the development of the ship, from
ancient times to to-day.

IN PREPARATION

ANCIENT GREECE. By Prof. GILBERT MURRAY, D.Litt., LL.D., F.B.A
ANCIENT EGYPT. By F. LL. GRIFFITH, M.A.
THE ANCIENT EAST. By D. G. HOGARTH, M.A., F.B.A.
A SHORT h'ISTOR YOFEUROPE. By HERBERT FISHER, M. A., F.B.A.
PREHISTORIC BRITAIN. By ROBERT MUNRO, M.A., M.D., LL.D.
THE BYZANTINE EMPIRE. By NORMAN H. BAVNES.
THE REFORM A TION. By Principal LINDSAY, LL.D.
NAPOLEON. By HERBERT FISHER, M.A., F.B.A.
A SHORT HISTORY OF RUSSIA. By Prof. MILYOUKOV.
MODERN TURKEY. By D. G. HOGARTH, M.A.
FRANCE OF TO-DAY. By ALBERT THOMAS.
GERMANY OF TO-DA Y. By CHARLES TOWER.
THE NAVY AND SEA POWER. By DAVID HANNAY.
HISTORY OF SCOTLAND. By R. S. RAIT, M.A.
SOUTH AMERICA. By Prof. W. R. SHEPHERD.
LONDON. By Sir LAURENCE GOMME, F.S.A.

HISTORY AND LITERATURE OF SPAIN. By J. FITZMAURICE-
KELLY, F.B.A., Litt.D.



Literature and



2. SHAKESPEARE

By JOHN MASEFIELD. " The book is a joy. We have had half-a-dozen more
learned books on Shakespeare in the last few years, but not one so wise."
Manchester Guardian.

27. ENGLISH LITERATURE: MODERN

By G. H. MAIR, M.A. " Altogether a fresh and individual book." Olstrver.

35. LANDMARKS IN FRENCH LITERATURE

By G. L. STRACHEY. " Mr Strachey is to be congratulated on his courage and
success. It is difficult to imagine how a better account of French Literature
could be given in 250 small pages than he has given here." The Times.



39- ARCHITECTURE

By Prof. W. R. LETHABY. (Over forty Illustrations.) " Popular guide-books
to architecture are, as a rule, not worth ranch. This volume is a welcome excep-
tion." Building News. " Delightfully bright reading." Christian World.

43. ENGLISH LITERATURE: MEDIAEVAL

By Prof. W. P. KER, M.A. "Prof. Ker has long proved his worth as one of
the soundest scholars in English we have, and he is the very man to put an
outline of English Mediaeval Literature before the uninstructed public. His
knowledge and taste are unimpeachable, and his style is effective, simple, yet
never dry." The Athemeum.

45. THE ENGLISH LANGUAGE

By L. PEARSALL SMI-TH, M.A. "A wholly fascinating study of the different
streams that went to the making of the great river of the English speech."
Daily News.

52. GREAT WRITERS OF AMERICA

By Prof. J. EKSKINE and Prof. W. P. TRENT. A popular sketch by two
foremost authorities.

IN PREPARATION

ANCIENT ART AND RITUAL. By Miss JANE HARRISON, LL.D.,

D.Litt.

GREEK LITERA TURE. By Prof. GILBERT MURRAY, D.Litt.
LA TIN LITER A TURE. By Prof. J. S. PHILLIMORE.
CHA UCER AND HIS TIME. By Miss G. E. HADOW.
THE RENAISSANCE. By Mrs R. A. TAYLOR.

ITALIAN A RTOF THE RENAISSANCE. By ROGER E. FRY, M.A.
THE ART OF PAINTING. By Sir FREUERICK WEDMORE.
DR JOHNSON AND HIS CIRCLE. By JOHN BAILEY, M.A.
THE VIC IORIAN AGE. By G. K- CHESTERTON.
ENGLISH COMPOSITION. By Prof. WM. T. BREWSTER.
GREA T WRITERS OF RUSSIA. By C. T. HAGBERG WRIGHT, LL.D.
THE LITERATURE OF GERMANY. By Prof. J. G. ROBERTSON,

M.A., Ph.D.
SCANDINAVIAN HISTORY AND LITERATURE. By T. C.

SNOW, M.A.



Science



7. MODERN GEOGRAPHY

By Dr MARION NEWBIGIN. (Illustrated.) "Geography, again: what a dull,
tedious study that was wont to be I . . . But Miss Marion Newbigin invests its
dry bones with the flesh and blood of romantic interest, taking stock of
geography as a fairy-book of science." Daily Telegraph.

9. THE EVOLUTION OF PLANTS

By Dr D. H. SCOTT, M.A., F.R.S., late Hon. Keeper of the Jodrell Laboratory,
Kew. (Fully illustrated.) "The information which the book provides is as
trustworthy as first-band knowledge can make it. ... Dr Scott's candid and
familiar style makes the difficult subject both fascinating and easy."
Gardeners' Chronicle.

17. HEALTH AND DISEASE

By W. LESLIE MACKKNZIE, M.D., Local Government Board, Edinburgh.
"The science of public health administration has had no abler or more attractive
exponent than Dr Mackenzie. He adds to a thorough grasp of the problems
an illuminating style, and an arresting manner of treating a subject often
dull and sometimes unsavoury." Economist.



1 8. INTRODUCTION TO MATHEMATICS

' By A. N. WHITEHEAD, Sc.D., F.R.S. (With Diagrams.) "MrWhitehead
has discharged with conspicuous success the task he is so exceptionally qualified

I to undertake. For he is one of our great authorities upon the foundations of the
science, and has the breadth of view which is so requisite in presenting to the
reader its aims. His exposition is clear and striking." Westminster Gazette.

19. THE ANIMAL WORLD

By Professor F. W. GAMBLE, D.Sc., F.R.S. With Introduction hy Sir Oliver
Lodge. (Many Illustrations.) " A delightful and instructive epitome of animal
(and vegetable) life. ... A most fascinating and suggestive survey." Morning
Post.

20. EVOLUTION

By Professor J. ARTHUR THOMSON and Professor PATRICK GEDDES. "A
many-coloured and romantic panorama, opening up, like no other book we know,
a rational vision of world-development." Belfast News-Letter.

22. CRIME AND INSANITY

By Dr C. A. MERCIER, F.R.C.P., F.R.C.S., Author of "Text-Book of In-
sanity," etc- " Furnishes much valuable information from one occupying the
highest position among medico-legal psychologists." Asylum NCVJS.

28. PSYCHICAL RESEARCH



and thus what he has to say on thought-reading, hypnotism, telepathy, crystal-
vision, spiritualism, divinings, and so on, will be read with avidity." Dundee




31. ASTRONOMY

By A. R. HINKS, M.A., Chief Assistant, Cambridge Observatory. "Original
in thought, eclectic in substance, and critical in treatment. . . . No better
little book is available." School World.

32. INTRODUCTION TO SCIENCE

By J. ARTHUR THOMSON, M.A., Regius Professor of Natural History, Aberdeen
University. " Professor Thomson's delightful literary style is well known; and
here he discourses freshly and easily on the methods of science and its relations
with philosophy, art, religion, and practical life." Aberdeen Journal,

36.



By H. N. DICKSON, D.Sc. Oxon., M.A., F.R.S.E., President of the Royal
Meteorological Society; Professor of Geography in University College, Reading.
(With Diagrams.) "The author has succeeded in presenting in a very lucid
and agreeable manner the causes of the movement of the atmosphere and of
the more stable winds." Manchester Guardian.

41. ANTHROPOLOGY

By R R. MARETT, M.A., Reade
"An absolutely perfect handboo
fascinating and human that it bea

44. THE PRINCIPLES OF PHYSIOLOGY

By Prof. J. G. McKENDRiCK, M.D. " It is a delightful and wonderfully com-
prehensive handling of a subject which, while of importance to all, does not
readily lend itself to untechnical explanation. . . . The little book is more than
a mere repository of knowledge; upon every page of it is stamped the impress
of a creative imagination." Glasgow Herald.



By R. R. MARETT, M.A., Reader in Social Anthropology in Oxford University.
"An absolutely perfect handbook, so clear that a child could understand it, so
fascinating and human that it beats fiction ' to a frazzle.' " Morning Leader.



46. MATTER AND ENERGY

By F. SODDY, M.A., F.R.S. "A most fascinating and instructive account or
the great facts of physical science, concerning which our knowledge, of later
years, has made such wonderful progress." The Bookseller.

49. PSYCHOLOGY, THE STUDY OF BEHAVIOUR

By Prof. W. McDouGALL, F.R.S., M.B. "A happy example of the non-
technical handling of an unwieldy science, suggesting rather than dogmatising.
It should whet appetites for deeper study." Christian World.

53. THE MAKING OF THE EARTH

ByProf.J.W. GREGORY, F.R.S. (With 38 Maps and Figures.) The Professor
of Geology at Glasgow describes the origin of the earth, the formation and
changes of its surface and structure, its geological history, the first appearance
of life, and its influence upon the globe.

57. THE HUMAN BODY

By A. KEITH, M.D., LL,D., Conservator of Museum and Hunterian Pro-
fessor, Royal College of Surgeons. (Illustrated.) The work of the dissecting-
room is described, and among other subjects dealt with are: the development
of the body; malformations and monstrosities; changes of youth and age; sex
differences, are they increasing or decreasing? race characters; bodily features
as indexes of mental character; degeneration and regeneration; and the
genealogy and antiquity of man.

58. ELECTRICITY

By GisBERT KAPP, D.Eng., M.I.E.E., M.I.C.E., Professor of Electrical
Engineering in the University of Birmingham. (Illustrated.) Deals with
frictional and contact electricity; potential; electrification by mechanical
means; the electric current; the dynamics of electric currents; alternating
currents; the distribution of electricity, etc.

IN PREPARATION

CHEMISTRY. Py Prof. R. MELDOLA, F.R.S.

THE MINERAL WORLD. By Sir T. H. HOLLAND, K.C.I. E., D.Sc.

PLANT LII-'E. By Prof. J. B. FARMER, F.R.S.

NERVES. By Prof. D. FRASER HARRIS, M.D., D.Sc.

A STUDY OF SEX. By Prof. J. A. THOMSON and Prof. PATRICK GEDDES.

THE GROWTH OF EUROPE. By Prof. GRKNVILLE COLE.



Philosophy and "Religion



ig's
:tate



15. MOHAMMEDANISM

By Prof. D. S. MARGOLIOUTH, M.A., D.Litt. "This generous shilling':
worth of wisdom. ... A delicate, humorous, and most responsible tractati
by an illuminative professor." Daily Mail.

40. THE PROBLEMS OF PHILOSOPHY

By the Hon. BERTRAND RUSSELL, F.R.S.: 'A book that the ' man in the
street ' will recognise at once to be a boon. . . . Consistently lucid and non-
technical throughout." Christian World.

47. BUDDHISM



go. NONCONFORMITY: Its ORIGIN and PROGRESS

I'.'- Principal W. B. SELBIE, M.A. "The historical part is brilliant in its
:., clarity, and proportion, and in the later chapters on the present position
.urns of Nonconformity Dr Selbie proves himself to be an ideal exponent
of sound and moderate views." Christian World.

54. ETHICS

By G. E. MOORE, M.A., Lecturer in Moral Science in Cambridge University.
Discusses Utilitarianism, the Objectivity of Moral Judgments, the Test of
Right and Wrong, Free Will, and Intrinsic Value.

56. THE MAKING OF THE NEW TESTAMENT

By Prof. B. W. BACON, LL. LX, D.D. An authoritative summary of the results
of modern critical research with regard to the origins of the New Testament, in
" the formative period when conscious inspiration was still in its full glow rather
than the period of collection into an official canon," showing the mingling of the
two great currents of Christian thought " Pauline and 'Apostolic,' the Greek-
Christian gospel about Jesus, and the Jewish-Christian gospel of Jesus, the
gospel of the Spirit and the gospel of au thority."

jo. MISSIONS: THEIR RISE and DEVELOPMENT

By Mrs CREIGHTON. The beginning of modern missions after the Reforma-
tion and their growth are traced, and an account is given of their present
work, its extent and character.

IN PREPARATION

THE OLD TESTAMENT. By Prof. GEORGE MOORE, D.D., LL.D.
BETWEEN THE OLD AND NEW TESTAMENTS. By R. H.

CHARLES, D.D.

COMPARATIVE RELIGION. By Prof. J. ESTLIN CARPENTER, D.Litt.
A HISTOR Y of FREEDOM of THOUGHT. By Prof. J. B. BURY, LL.D.
A HISTORY OF PHILOSOPHY. By CLEMKNT WKBB, M.A.



Social Science



. PARLIAMENT

Its History, Constitution, and Practice. By Sir COURTENAY P. ILBERT.
K.C.B., K.C.S.I., Clerk of the House of Commons. "The best book on the
history and practice of the House of Commons since Bagehot's 'Constitution.'"
Yorkshire Post.

. THE STOCK EXCHANGE

By F. W. HIRST, Editor of " The Economist." " To an unfinancial mind must
be a revelation. . . . The book is as clear, vigorous, and sane as Bagehot's ' Lom-
bard Street,' than which there is no higher compliment." Morning Leader

. IRISH NATIONALITY

By Mrs J. R. GREEN. " As glowing as it is learned. No book could be more
timely." Daily News. "A powerful study. . . . A magnificent demonstration
of the deserved vitality of the Gaelic spirit." Freeman s Journal.

3. THE SOCIALIST MOVEMENT

RAMSAY MACDONALD, M.T. "Admirably adapted for the purpose of
exposition." The Times. "Mr MacDonald is a very lucid exponent. . . . The
volume will be of great use in dispelling illusions about the tendencies of
Socialism in this country." The Nation.

i. CONSERVATISM

Jy Lord HUGH CECIL, M.A., M.P. "One of those great little books which
seldom appear more than once in a generation." Morning Post.



1 6. THE SCIENCE OF WEALTH

By J. A. HOUSON, M.A. "Mr J. A. Hobson holds an unique position among
living economists. . . . The text-book produced is altogether admirable.
Original, reasonable, and illuminating." The Nation.

21. LIBERALISM

By L. T. HOBHOUSE, M. A., Professor of Sociology in the University of London.
"A book of rare quality. . . . We have nothing but praise for the rapid and
masterly summaries of the arguments from first principles which form a large
part of this book." Westminster Gazette.

24. THE EVOLUTION OF INDUSTRY

ByD. H. MACGREGCR, M.A., Professor of Political Economy in the University
of Leeds. "A volume so dispassionate in terms may be read with profit by all
interested in the present state of unrest." Aberdeen Journal.

26. AGRICULTURE

By Prof. W. SOMERVILLE, F.L.S. " It makes the results of laboratory work
at the University accessible to the practical farmer." Athena-urn.

30. ELEMENTS OF ENGLISH LA W

By W. M. GELDART, M.A., B.C.L., Vinerian Professor of English Law at
Oxford. "Contains a very clear account of the elementary principles under-
lying the rules of English law; and we can recommend it to all who wish to
become acquainted with these elementary principles with a minimum of
trouble." Scots Law Times.

38. THE SCHOOL

An Introduction to the Study of Education.

By J. J. FINDLAY, M.A., Ph.D., Professor of Education in Manchester
University. &lt;: An amazingly comprehensive volume. . . . It is a remarkable
performance, distinguished in its crisp, striking phraseology as well as its
inclusiveness of subject-matter." Morning Post.

-59. ELEMENTS OF POLITICAL ECONOMY

By S. J. CHAPMAN, M.A., Professor of Political Economy in Manchester
University. A simple explanation, in the light of the latest economic thought,
of the working of demand and supply; the nature of monopoly; money and
international trade; the relation of wages, profit, interest, and rent; and the
effects of labour combination prefaced by a short sketch of economic study
since Adam Smith.

IN PREPARATION

THE CRIMINAL AND THE COMMUNITY. By Viscount ST.

CYRES, M.A.

COMMONSENSE IN LA W. By Prof. P. VINOGRADOFF, D.C.L.
THE CIVIL SERVICE. By GRAHAM WALLAS, M.A.
PRACTICAL IDEALISM. By MAURICE HEWLETT.
NEWSPAPERS. By G. BINNEY DIBBLEE.
ENGLISH VILLAGE LIFE. By E. N. BENNETT, M.A.
CO -PARTNERSHIP At\D PROFIT-SHARING. By ANEURIN

WILLIAMS, J.P.

THE SOCIAL SETTLEMENT. By JANE ADDAMS and R. A. WOODS.
GREA T INVENTIONS. By Prof. J. L. MYRES, M.A., F.S.A.
TOWN PLANNING. By RAYMOND UNWIN.
POLITICAL THOUGHT IN ENGLAND: From Bentham to J. S.

Mill. By Prof. W. L. DAVIDSON.
POLITICAL THOUGHT IN ENGLAKD: From Herbert Spencer

to To-day. By ERNEST BARKER, M.A.

London: WimTMS^AND~NORGATE

And of all Bookshops and Bookstalls.
%[** TN: End of catalog OCR text]
\fi
%%%%%%%%%%%%%%%%%%%%%%%%% GUTENBERG LICENSE %%%%%%%%%%%%%%%%%%%%%%%%%%
\PGLicense
\begin{PGtext}
End of the Project Gutenberg EBook of An Introduction to Mathematics, by 
Alfred North Whitehead

*** END OF THIS PROJECT GUTENBERG EBOOK AN INTRODUCTION TO MATHEMATICS ***

***** This file should be named 41568-tex.tex or 41568-tex.zip *****
This and all associated files of various formats will be found in:
        http://www.gutenberg.org/4/1/5/6/41568/

Produced by Andrew D. Hwang. (This ebook was produced using
OCR text generously provided by the University of
California, Santa Barbara, through the Internet Archive.)


Updated editions will replace the previous one--the old editions
will be renamed.

Creating the works from public domain print editions means that no
one owns a United States copyright in these works, so the Foundation
(and you!) can copy and distribute it in the United States without
permission and without paying copyright royalties.  Special rules,
set forth in the General Terms of Use part of this license, apply to
copying and distributing Project Gutenberg-tm electronic works to
protect the PROJECT GUTENBERG-tm concept and trademark.  Project
Gutenberg is a registered trademark, and may not be used if you
charge for the eBooks, unless you receive specific permission.  If you
do not charge anything for copies of this eBook, complying with the
rules is very easy.  You may use this eBook for nearly any purpose
such as creation of derivative works, reports, performances and
research.  They may be modified and printed and given away--you may do
practically ANYTHING with public domain eBooks.  Redistribution is
subject to the trademark license, especially commercial
redistribution.



*** START: FULL LICENSE ***

THE FULL PROJECT GUTENBERG LICENSE
PLEASE READ THIS BEFORE YOU DISTRIBUTE OR USE THIS WORK

To protect the Project Gutenberg-tm mission of promoting the free
distribution of electronic works, by using or distributing this work
(or any other work associated in any way with the phrase "Project
Gutenberg"), you agree to comply with all the terms of the Full Project
Gutenberg-tm License available with this file or online at
  www.gutenberg.org/license.


Section 1.  General Terms of Use and Redistributing Project Gutenberg-tm
electronic works

1.A.  By reading or using any part of this Project Gutenberg-tm
electronic work, you indicate that you have read, understand, agree to
and accept all the terms of this license and intellectual property
(trademark/copyright) agreement.  If you do not agree to abide by all
the terms of this agreement, you must cease using and return or destroy
all copies of Project Gutenberg-tm electronic works in your possession.
If you paid a fee for obtaining a copy of or access to a Project
Gutenberg-tm electronic work and you do not agree to be bound by the
terms of this agreement, you may obtain a refund from the person or
entity to whom you paid the fee as set forth in paragraph 1.E.8.

1.B.  "Project Gutenberg" is a registered trademark.  It may only be
used on or associated in any way with an electronic work by people who
agree to be bound by the terms of this agreement.  There are a few
things that you can do with most Project Gutenberg-tm electronic works
even without complying with the full terms of this agreement.  See
paragraph 1.C below.  There are a lot of things you can do with Project
Gutenberg-tm electronic works if you follow the terms of this agreement
and help preserve free future access to Project Gutenberg-tm electronic
works.  See paragraph 1.E below.

1.C.  The Project Gutenberg Literary Archive Foundation ("the Foundation"
or PGLAF), owns a compilation copyright in the collection of Project
Gutenberg-tm electronic works.  Nearly all the individual works in the
collection are in the public domain in the United States.  If an
individual work is in the public domain in the United States and you are
located in the United States, we do not claim a right to prevent you from
copying, distributing, performing, displaying or creating derivative
works based on the work as long as all references to Project Gutenberg
are removed.  Of course, we hope that you will support the Project
Gutenberg-tm mission of promoting free access to electronic works by
freely sharing Project Gutenberg-tm works in compliance with the terms of
this agreement for keeping the Project Gutenberg-tm name associated with
the work.  You can easily comply with the terms of this agreement by
keeping this work in the same format with its attached full Project
Gutenberg-tm License when you share it without charge with others.

1.D.  The copyright laws of the place where you are located also govern
what you can do with this work.  Copyright laws in most countries are in
a constant state of change.  If you are outside the United States, check
the laws of your country in addition to the terms of this agreement
before downloading, copying, displaying, performing, distributing or
creating derivative works based on this work or any other Project
Gutenberg-tm work.  The Foundation makes no representations concerning
the copyright status of any work in any country outside the United
States.

1.E.  Unless you have removed all references to Project Gutenberg:

1.E.1.  The following sentence, with active links to, or other immediate
access to, the full Project Gutenberg-tm License must appear prominently
whenever any copy of a Project Gutenberg-tm work (any work on which the
phrase "Project Gutenberg" appears, or with which the phrase "Project
Gutenberg" is associated) is accessed, displayed, performed, viewed,
copied or distributed:

This eBook is for the use of anyone anywhere at no cost and with
almost no restrictions whatsoever.  You may copy it, give it away or
re-use it under the terms of the Project Gutenberg License included
with this eBook or online at www.gutenberg.org

1.E.2.  If an individual Project Gutenberg-tm electronic work is derived
from the public domain (does not contain a notice indicating that it is
posted with permission of the copyright holder), the work can be copied
and distributed to anyone in the United States without paying any fees
or charges.  If you are redistributing or providing access to a work
with the phrase "Project Gutenberg" associated with or appearing on the
work, you must comply either with the requirements of paragraphs 1.E.1
through 1.E.7 or obtain permission for the use of the work and the
Project Gutenberg-tm trademark as set forth in paragraphs 1.E.8 or
1.E.9.

1.E.3.  If an individual Project Gutenberg-tm electronic work is posted
with the permission of the copyright holder, your use and distribution
must comply with both paragraphs 1.E.1 through 1.E.7 and any additional
terms imposed by the copyright holder.  Additional terms will be linked
to the Project Gutenberg-tm License for all works posted with the
permission of the copyright holder found at the beginning of this work.

1.E.4.  Do not unlink or detach or remove the full Project Gutenberg-tm
License terms from this work, or any files containing a part of this
work or any other work associated with Project Gutenberg-tm.

1.E.5.  Do not copy, display, perform, distribute or redistribute this
electronic work, or any part of this electronic work, without
prominently displaying the sentence set forth in paragraph 1.E.1 with
active links or immediate access to the full terms of the Project
Gutenberg-tm License.

1.E.6.  You may convert to and distribute this work in any binary,
compressed, marked up, nonproprietary or proprietary form, including any
word processing or hypertext form.  However, if you provide access to or
distribute copies of a Project Gutenberg-tm work in a format other than
"Plain Vanilla ASCII" or other format used in the official version
posted on the official Project Gutenberg-tm web site (www.gutenberg.org),
you must, at no additional cost, fee or expense to the user, provide a
copy, a means of exporting a copy, or a means of obtaining a copy upon
request, of the work in its original "Plain Vanilla ASCII" or other
form.  Any alternate format must include the full Project Gutenberg-tm
License as specified in paragraph 1.E.1.

1.E.7.  Do not charge a fee for access to, viewing, displaying,
performing, copying or distributing any Project Gutenberg-tm works
unless you comply with paragraph 1.E.8 or 1.E.9.

1.E.8.  You may charge a reasonable fee for copies of or providing
access to or distributing Project Gutenberg-tm electronic works provided
that

- You pay a royalty fee of 20% of the gross profits you derive from
     the use of Project Gutenberg-tm works calculated using the method
     you already use to calculate your applicable taxes.  The fee is
     owed to the owner of the Project Gutenberg-tm trademark, but he
     has agreed to donate royalties under this paragraph to the
     Project Gutenberg Literary Archive Foundation.  Royalty payments
     must be paid within 60 days following each date on which you
     prepare (or are legally required to prepare) your periodic tax
     returns.  Royalty payments should be clearly marked as such and
     sent to the Project Gutenberg Literary Archive Foundation at the
     address specified in Section 4, "Information about donations to
     the Project Gutenberg Literary Archive Foundation."

- You provide a full refund of any money paid by a user who notifies
     you in writing (or by e-mail) within 30 days of receipt that s/he
     does not agree to the terms of the full Project Gutenberg-tm
     License.  You must require such a user to return or
     destroy all copies of the works possessed in a physical medium
     and discontinue all use of and all access to other copies of
     Project Gutenberg-tm works.

- You provide, in accordance with paragraph 1.F.3, a full refund of any
     money paid for a work or a replacement copy, if a defect in the
     electronic work is discovered and reported to you within 90 days
     of receipt of the work.

- You comply with all other terms of this agreement for free
     distribution of Project Gutenberg-tm works.

1.E.9.  If you wish to charge a fee or distribute a Project Gutenberg-tm
electronic work or group of works on different terms than are set
forth in this agreement, you must obtain permission in writing from
both the Project Gutenberg Literary Archive Foundation and Michael
Hart, the owner of the Project Gutenberg-tm trademark.  Contact the
Foundation as set forth in Section 3 below.

1.F.

1.F.1.  Project Gutenberg volunteers and employees expend considerable
effort to identify, do copyright research on, transcribe and proofread
public domain works in creating the Project Gutenberg-tm
collection.  Despite these efforts, Project Gutenberg-tm electronic
works, and the medium on which they may be stored, may contain
"Defects," such as, but not limited to, incomplete, inaccurate or
corrupt data, transcription errors, a copyright or other intellectual
property infringement, a defective or damaged disk or other medium, a
computer virus, or computer codes that damage or cannot be read by
your equipment.

1.F.2.  LIMITED WARRANTY, DISCLAIMER OF DAMAGES - Except for the "Right
of Replacement or Refund" described in paragraph 1.F.3, the Project
Gutenberg Literary Archive Foundation, the owner of the Project
Gutenberg-tm trademark, and any other party distributing a Project
Gutenberg-tm electronic work under this agreement, disclaim all
liability to you for damages, costs and expenses, including legal
fees.  YOU AGREE THAT YOU HAVE NO REMEDIES FOR NEGLIGENCE, STRICT
LIABILITY, BREACH OF WARRANTY OR BREACH OF CONTRACT EXCEPT THOSE
PROVIDED IN PARAGRAPH 1.F.3.  YOU AGREE THAT THE FOUNDATION, THE
TRADEMARK OWNER, AND ANY DISTRIBUTOR UNDER THIS AGREEMENT WILL NOT BE
LIABLE TO YOU FOR ACTUAL, DIRECT, INDIRECT, CONSEQUENTIAL, PUNITIVE OR
INCIDENTAL DAMAGES EVEN IF YOU GIVE NOTICE OF THE POSSIBILITY OF SUCH
DAMAGE.

1.F.3.  LIMITED RIGHT OF REPLACEMENT OR REFUND - If you discover a
defect in this electronic work within 90 days of receiving it, you can
receive a refund of the money (if any) you paid for it by sending a
written explanation to the person you received the work from.  If you
received the work on a physical medium, you must return the medium with
your written explanation.  The person or entity that provided you with
the defective work may elect to provide a replacement copy in lieu of a
refund.  If you received the work electronically, the person or entity
providing it to you may choose to give you a second opportunity to
receive the work electronically in lieu of a refund.  If the second copy
is also defective, you may demand a refund in writing without further
opportunities to fix the problem.

1.F.4.  Except for the limited right of replacement or refund set forth
in paragraph 1.F.3, this work is provided to you 'AS-IS', WITH NO OTHER
WARRANTIES OF ANY KIND, EXPRESS OR IMPLIED, INCLUDING BUT NOT LIMITED TO
WARRANTIES OF MERCHANTABILITY OR FITNESS FOR ANY PURPOSE.

1.F.5.  Some states do not allow disclaimers of certain implied
warranties or the exclusion or limitation of certain types of damages.
If any disclaimer or limitation set forth in this agreement violates the
law of the state applicable to this agreement, the agreement shall be
interpreted to make the maximum disclaimer or limitation permitted by
the applicable state law.  The invalidity or unenforceability of any
provision of this agreement shall not void the remaining provisions.

1.F.6.  INDEMNITY - You agree to indemnify and hold the Foundation, the
trademark owner, any agent or employee of the Foundation, anyone
providing copies of Project Gutenberg-tm electronic works in accordance
with this agreement, and any volunteers associated with the production,
promotion and distribution of Project Gutenberg-tm electronic works,
harmless from all liability, costs and expenses, including legal fees,
that arise directly or indirectly from any of the following which you do
or cause to occur: (a) distribution of this or any Project Gutenberg-tm
work, (b) alteration, modification, or additions or deletions to any
Project Gutenberg-tm work, and (c) any Defect you cause.


Section  2.  Information about the Mission of Project Gutenberg-tm

Project Gutenberg-tm is synonymous with the free distribution of
electronic works in formats readable by the widest variety of computers
including obsolete, old, middle-aged and new computers.  It exists
because of the efforts of hundreds of volunteers and donations from
people in all walks of life.

Volunteers and financial support to provide volunteers with the
assistance they need are critical to reaching Project Gutenberg-tm's
goals and ensuring that the Project Gutenberg-tm collection will
remain freely available for generations to come.  In 2001, the Project
Gutenberg Literary Archive Foundation was created to provide a secure
and permanent future for Project Gutenberg-tm and future generations.
To learn more about the Project Gutenberg Literary Archive Foundation
and how your efforts and donations can help, see Sections 3 and 4
and the Foundation information page at www.gutenberg.org


Section 3.  Information about the Project Gutenberg Literary Archive
Foundation

The Project Gutenberg Literary Archive Foundation is a non profit
501(c)(3) educational corporation organized under the laws of the
state of Mississippi and granted tax exempt status by the Internal
Revenue Service.  The Foundation's EIN or federal tax identification
number is 64-6221541.  Contributions to the Project Gutenberg
Literary Archive Foundation are tax deductible to the full extent
permitted by U.S. federal laws and your state's laws.

The Foundation's principal office is located at 4557 Melan Dr. S.
Fairbanks, AK, 99712., but its volunteers and employees are scattered
throughout numerous locations.  Its business office is located at 809
North 1500 West, Salt Lake City, UT 84116, (801) 596-1887.  Email
contact links and up to date contact information can be found at the
Foundation's web site and official page at www.gutenberg.org/contact

For additional contact information:
     Dr. Gregory B. Newby
     Chief Executive and Director
     gbnewby@pglaf.org

Section 4.  Information about Donations to the Project Gutenberg
Literary Archive Foundation

Project Gutenberg-tm depends upon and cannot survive without wide
spread public support and donations to carry out its mission of
increasing the number of public domain and licensed works that can be
freely distributed in machine readable form accessible by the widest
array of equipment including outdated equipment.  Many small donations
($1 to $5,000) are particularly important to maintaining tax exempt
status with the IRS.

The Foundation is committed to complying with the laws regulating
charities and charitable donations in all 50 states of the United
States.  Compliance requirements are not uniform and it takes a
considerable effort, much paperwork and many fees to meet and keep up
with these requirements.  We do not solicit donations in locations
where we have not received written confirmation of compliance.  To
SEND DONATIONS or determine the status of compliance for any
particular state visit www.gutenberg.org/donate

While we cannot and do not solicit contributions from states where we
have not met the solicitation requirements, we know of no prohibition
against accepting unsolicited donations from donors in such states who
approach us with offers to donate.

International donations are gratefully accepted, but we cannot make
any statements concerning tax treatment of donations received from
outside the United States.  U.S. laws alone swamp our small staff.

Please check the Project Gutenberg Web pages for current donation
methods and addresses.  Donations are accepted in a number of other
ways including checks, online payments and credit card donations.
To donate, please visit:  www.gutenberg.org/donate


Section 5.  General Information About Project Gutenberg-tm electronic
works.

Professor Michael S. Hart was the originator of the Project Gutenberg-tm
concept of a library of electronic works that could be freely shared
with anyone.  For forty years, he produced and distributed Project
Gutenberg-tm eBooks with only a loose network of volunteer support.

Project Gutenberg-tm eBooks are often created from several printed
editions, all of which are confirmed as Public Domain in the U.S.
unless a copyright notice is included.  Thus, we do not necessarily
keep eBooks in compliance with any particular paper edition.

Most people start at our Web site which has the main PG search facility:

     www.gutenberg.org

This Web site includes information about Project Gutenberg-tm,
including how to make donations to the Project Gutenberg Literary
Archive Foundation, how to help produce our new eBooks, and how to
subscribe to our email newsletter to hear about new eBooks.
\end{PGtext}

% %%%%%%%%%%%%%%%%%%%%%%%%%%%%%%%%%%%%%%%%%%%%%%%%%%%%%%%%%%%%%%%%%%%%%%% %
%                                                                         %
% End of the Project Gutenberg EBook of An Introduction to Mathematics, by 
% Alfred North Whitehead                                                  %
%                                                                         %
% *** END OF THIS PROJECT GUTENBERG EBOOK AN INTRODUCTION TO MATHEMATICS ***
%                                                                         %
% ***** This file should be named 41568-tex.tex or 41568-tex.zip *****    %
% This and all associated files of various formats will be found in:      %
%         http://www.gutenberg.org/4/1/5/6/41568/                         %
%                                                                         %
% %%%%%%%%%%%%%%%%%%%%%%%%%%%%%%%%%%%%%%%%%%%%%%%%%%%%%%%%%%%%%%%%%%%%%%% %

\end{document}
###
@ControlwordReplace = (
  ['\\AD', 'A.D.'],
  ['\\BC', 'B.C.'],
  ['\\cf', 'cf.'],
  ['\\Cf', 'Cf.'],
  ['\\eg', 'e.g.'],
  ['\\ie', 'i.e.'],
  ['\\viz', 'viz.']
  );

@ControlwordArguments = (
  ['\\Diagram', 1, 0, '<GRAPHIC>', ''],
  ['\\Figure', 0, 0, '', '', 1, 0, '<GRAPHIC>', ''],
  ['\\ToCLine', 1, 1, '', ' ', 1, 1, '', '', 1, 0, '', ''],
  ['\\BookMark', 1, 0, '', '', 1, 0, '', ''],
  ['\\First', 1, 1, '', ''],
  ['\\Chapter', 0, 0, '', '', 1, 1, 'Chapter ', '. ', 1, 1, '', ''],
  ['\\ChapRef', 0, 0, '', '', 1, 1, 'Chapter ', ''],
  ['\\Appendix', 0, 0, '', '', 1, 1, '', ''],
  ['\\Note', 1, 1, '', ''],
  ['\\Fig', 0, 0, '', '', 1, 1, 'Fig. ', ''],
  ['\\FigNum', 1, 1, '', ''],
  ['\\Pagelabel', 1, 0, '', ''],
  ['\\Pageref', 0, 0, '', '', 1, 0, 'p. ', '00'],
  ['\\Eq', 0, 0, '', '', 1, 1, '', ''],
  ['\\Typo', 1, 0, '', '', 1, 1, '', ''],
  ['\\Add', 1, 1, '', ''],
  ['\\Chg', 1, 0, '', '', 1, 1, '', '']
  );
$PageSeparator = qr/^\\PageSep/;
$CustomClean = 'print "\\nCustom cleaning in progress...";
my $cline = 0;
 while ($cline <= $#file) {
   $file[$cline] =~ s/--------[^\n]*\n//; # strip page separators
   $cline++
 }
 print "done\\n";';
###
This is pdfTeX, Version 3.1415926-1.40.10 (TeX Live 2009/Debian) (format=pdflatex 2012.9.24)  6 DEC 2012 09:22
entering extended mode
 %&-line parsing enabled.
**41568-t.tex
(./41568-t.tex
LaTeX2e <2009/09/24>
Babel <v3.8l> and hyphenation patterns for english, usenglishmax, dumylang, noh
yphenation, farsi, arabic, croatian, bulgarian, ukrainian, russian, czech, slov
ak, danish, dutch, finnish, french, basque, ngerman, german, german-x-2009-06-1
9, ngerman-x-2009-06-19, ibycus, monogreek, greek, ancientgreek, hungarian, san
skrit, italian, latin, latvian, lithuanian, mongolian2a, mongolian, bokmal, nyn
orsk, romanian, irish, coptic, serbian, turkish, welsh, esperanto, uppersorbian
, estonian, indonesian, interlingua, icelandic, kurmanji, slovenian, polish, po
rtuguese, spanish, galician, catalan, swedish, ukenglish, pinyin, loaded.
(/usr/share/texmf-texlive/tex/latex/base/book.cls
Document Class: book 2007/10/19 v1.4h Standard LaTeX document class
(/usr/share/texmf-texlive/tex/latex/base/leqno.clo
File: leqno.clo 1998/08/17 v1.1c Standard LaTeX option (left equation numbers)
) (/usr/share/texmf-texlive/tex/latex/base/bk12.clo
File: bk12.clo 2007/10/19 v1.4h Standard LaTeX file (size option)
)
\c@part=\count79
\c@chapter=\count80
\c@section=\count81
\c@subsection=\count82
\c@subsubsection=\count83
\c@paragraph=\count84
\c@subparagraph=\count85
\c@figure=\count86
\c@table=\count87
\abovecaptionskip=\skip41
\belowcaptionskip=\skip42
\bibindent=\dimen102
) (/usr/share/texmf-texlive/tex/latex/base/inputenc.sty
Package: inputenc 2008/03/30 v1.1d Input encoding file
\inpenc@prehook=\toks14
\inpenc@posthook=\toks15
(/usr/share/texmf-texlive/tex/latex/base/latin1.def
File: latin1.def 2008/03/30 v1.1d Input encoding file
)) (/usr/share/texmf-texlive/tex/latex/base/ifthen.sty
Package: ifthen 2001/05/26 v1.1c Standard LaTeX ifthen package (DPC)
) (/usr/share/texmf-texlive/tex/latex/amsmath/amsmath.sty
Package: amsmath 2000/07/18 v2.13 AMS math features
\@mathmargin=\skip43
For additional information on amsmath, use the `?' option.
(/usr/share/texmf-texlive/tex/latex/amsmath/amstext.sty
Package: amstext 2000/06/29 v2.01
(/usr/share/texmf-texlive/tex/latex/amsmath/amsgen.sty
File: amsgen.sty 1999/11/30 v2.0
\@emptytoks=\toks16
\ex@=\dimen103
)) (/usr/share/texmf-texlive/tex/latex/amsmath/amsbsy.sty
Package: amsbsy 1999/11/29 v1.2d
\pmbraise@=\dimen104
) (/usr/share/texmf-texlive/tex/latex/amsmath/amsopn.sty
Package: amsopn 1999/12/14 v2.01 operator names
)
\inf@bad=\count88
LaTeX Info: Redefining \frac on input line 211.
\uproot@=\count89
\leftroot@=\count90
LaTeX Info: Redefining \overline on input line 307.
\classnum@=\count91
\DOTSCASE@=\count92
LaTeX Info: Redefining \ldots on input line 379.
LaTeX Info: Redefining \dots on input line 382.
LaTeX Info: Redefining \cdots on input line 467.
\Mathstrutbox@=\box26
\strutbox@=\box27
\big@size=\dimen105
LaTeX Font Info:    Redeclaring font encoding OML on input line 567.
LaTeX Font Info:    Redeclaring font encoding OMS on input line 568.
\macc@depth=\count93
\c@MaxMatrixCols=\count94
\dotsspace@=\muskip10
\c@parentequation=\count95
\dspbrk@lvl=\count96
\tag@help=\toks17
\row@=\count97
\column@=\count98
\maxfields@=\count99
\andhelp@=\toks18
\eqnshift@=\dimen106
\alignsep@=\dimen107
\tagshift@=\dimen108
\tagwidth@=\dimen109
\totwidth@=\dimen110
\lineht@=\dimen111
\@envbody=\toks19
\multlinegap=\skip44
\multlinetaggap=\skip45
\mathdisplay@stack=\toks20
LaTeX Info: Redefining \[ on input line 2666.
LaTeX Info: Redefining \] on input line 2667.
) (/usr/share/texmf-texlive/tex/latex/amsfonts/amssymb.sty
Package: amssymb 2009/06/22 v3.00
(/usr/share/texmf-texlive/tex/latex/amsfonts/amsfonts.sty
Package: amsfonts 2009/06/22 v3.00 Basic AMSFonts support
\symAMSa=\mathgroup4
\symAMSb=\mathgroup5
LaTeX Font Info:    Overwriting math alphabet `\mathfrak' in version `bold'
(Font)                  U/euf/m/n --> U/euf/b/n on input line 96.
)) (/usr/share/texmf-texlive/tex/latex/base/alltt.sty
Package: alltt 1997/06/16 v2.0g defines alltt environment
) (/usr/share/texmf-texlive/tex/latex/tools/indentfirst.sty
Package: indentfirst 1995/11/23 v1.03 Indent first paragraph (DPC)
) (/usr/share/texmf-texlive/tex/latex/footmisc/footmisc.sty
Package: footmisc 2009/09/15 v5.5a a miscellany of footnote facilities
\FN@temptoken=\toks21
\footnotemargin=\dimen112
\c@pp@next@reset=\count100
\c@@fnserial=\count101
Package footmisc Info: Declaring symbol style bringhurst on input line 855.
Package footmisc Info: Declaring symbol style chicago on input line 863.
Package footmisc Info: Declaring symbol style wiley on input line 872.
Package footmisc Info: Declaring symbol style lamport-robust on input line 883.

Package footmisc Info: Declaring symbol style lamport* on input line 903.
Package footmisc Info: Declaring symbol style lamport*-robust on input line 924
.
) (/usr/share/texmf-texlive/tex/latex/tools/multicol.sty
Package: multicol 2008/12/05 v1.6h multicolumn formatting (FMi)
\c@tracingmulticols=\count102
\mult@box=\box28
\multicol@leftmargin=\dimen113
\c@unbalance=\count103
\c@collectmore=\count104
\doublecol@number=\count105
\multicoltolerance=\count106
\multicolpretolerance=\count107
\full@width=\dimen114
\page@free=\dimen115
\premulticols=\dimen116
\postmulticols=\dimen117
\multicolsep=\skip46
\multicolbaselineskip=\skip47
\partial@page=\box29
\last@line=\box30
\mult@rightbox=\box31
\mult@grightbox=\box32
\mult@gfirstbox=\box33
\mult@firstbox=\box34
\@tempa=\box35
\@tempa=\box36
\@tempa=\box37
\@tempa=\box38
\@tempa=\box39
\@tempa=\box40
\@tempa=\box41
\@tempa=\box42
\@tempa=\box43
\@tempa=\box44
\@tempa=\box45
\@tempa=\box46
\@tempa=\box47
\@tempa=\box48
\@tempa=\box49
\@tempa=\box50
\@tempa=\box51
\c@columnbadness=\count108
\c@finalcolumnbadness=\count109
\last@try=\dimen118
\multicolovershoot=\dimen119
\multicolundershoot=\dimen120
\mult@nat@firstbox=\box52
\colbreak@box=\box53
) (/usr/share/texmf-texlive/tex/latex/base/makeidx.sty
Package: makeidx 2000/03/29 v1.0m Standard LaTeX package
) (/usr/share/texmf-texlive/tex/latex/graphics/graphicx.sty
Package: graphicx 1999/02/16 v1.0f Enhanced LaTeX Graphics (DPC,SPQR)
(/usr/share/texmf-texlive/tex/latex/graphics/keyval.sty
Package: keyval 1999/03/16 v1.13 key=value parser (DPC)
\KV@toks@=\toks22
) (/usr/share/texmf-texlive/tex/latex/graphics/graphics.sty
Package: graphics 2009/02/05 v1.0o Standard LaTeX Graphics (DPC,SPQR)
(/usr/share/texmf-texlive/tex/latex/graphics/trig.sty
Package: trig 1999/03/16 v1.09 sin cos tan (DPC)
) (/etc/texmf/tex/latex/config/graphics.cfg
File: graphics.cfg 2009/08/28 v1.8 graphics configuration of TeX Live
)
Package graphics Info: Driver file: pdftex.def on input line 91.
(/usr/share/texmf-texlive/tex/latex/pdftex-def/pdftex.def
File: pdftex.def 2009/08/25 v0.04m Graphics/color for pdfTeX
\Gread@gobject=\count110
))
\Gin@req@height=\dimen121
\Gin@req@width=\dimen122
) (/usr/share/texmf-texlive/tex/latex/caption/caption.sty
Package: caption 2009/10/09 v3.1k Customizing captions (AR)
(/usr/share/texmf-texlive/tex/latex/caption/caption3.sty
Package: caption3 2009/10/09 v3.1k caption3 kernel (AR)
\captionmargin=\dimen123
\captionmargin@=\dimen124
\captionwidth=\dimen125
\caption@indent=\dimen126
\caption@parindent=\dimen127
\caption@hangindent=\dimen128
)
\c@ContinuedFloat=\count111
) (/usr/share/texmf-texlive/tex/latex/tools/calc.sty
Package: calc 2007/08/22 v4.3 Infix arithmetic (KKT,FJ)
\calc@Acount=\count112
\calc@Bcount=\count113
\calc@Adimen=\dimen129
\calc@Bdimen=\dimen130
\calc@Askip=\skip48
\calc@Bskip=\skip49
LaTeX Info: Redefining \setlength on input line 76.
LaTeX Info: Redefining \addtolength on input line 77.
\calc@Ccount=\count114
\calc@Cskip=\skip50
) (/usr/share/texmf-texlive/tex/latex/fancyhdr/fancyhdr.sty
\fancy@headwidth=\skip51
\f@ncyO@elh=\skip52
\f@ncyO@erh=\skip53
\f@ncyO@olh=\skip54
\f@ncyO@orh=\skip55
\f@ncyO@elf=\skip56
\f@ncyO@erf=\skip57
\f@ncyO@olf=\skip58
\f@ncyO@orf=\skip59
) (/usr/share/texmf-texlive/tex/latex/geometry/geometry.sty
Package: geometry 2008/12/21 v4.2 Page Geometry
(/usr/share/texmf-texlive/tex/generic/oberdiek/ifpdf.sty
Package: ifpdf 2009/04/10 v2.0 Provides the ifpdf switch (HO)
Package ifpdf Info: pdfTeX in pdf mode detected.
) (/usr/share/texmf-texlive/tex/generic/oberdiek/ifvtex.sty
Package: ifvtex 2008/11/04 v1.4 Switches for detecting VTeX and its modes (HO)
Package ifvtex Info: VTeX not detected.
)
\Gm@cnth=\count115
\Gm@cntv=\count116
\c@Gm@tempcnt=\count117
\Gm@bindingoffset=\dimen131
\Gm@wd@mp=\dimen132
\Gm@odd@mp=\dimen133
\Gm@even@mp=\dimen134
\Gm@dimlist=\toks23
(/usr/share/texmf-texlive/tex/xelatex/xetexconfig/geometry.cfg)) (/usr/share/te
xmf-texlive/tex/latex/hyperref/hyperref.sty
Package: hyperref 2009/10/09 v6.79a Hypertext links for LaTeX
(/usr/share/texmf-texlive/tex/generic/ifxetex/ifxetex.sty
Package: ifxetex 2009/01/23 v0.5 Provides ifxetex conditional
) (/usr/share/texmf-texlive/tex/latex/oberdiek/hycolor.sty
Package: hycolor 2009/10/02 v1.5 Code for color options of hyperref/bookmark (H
O)
(/usr/share/texmf-texlive/tex/latex/oberdiek/xcolor-patch.sty
Package: xcolor-patch 2009/10/02 xcolor patch
))
\@linkdim=\dimen135
\Hy@linkcounter=\count118
\Hy@pagecounter=\count119
(/usr/share/texmf-texlive/tex/latex/hyperref/pd1enc.def
File: pd1enc.def 2009/10/09 v6.79a Hyperref: PDFDocEncoding definition (HO)
) (/usr/share/texmf-texlive/tex/generic/oberdiek/etexcmds.sty
Package: etexcmds 2007/12/12 v1.2 Prefix for e-TeX command names (HO)
(/usr/share/texmf-texlive/tex/generic/oberdiek/infwarerr.sty
Package: infwarerr 2007/09/09 v1.2 Providing info/warning/message (HO)
)
Package etexcmds Info: Could not find \expanded.
(etexcmds)             That can mean that you are not using pdfTeX 1.50 or
(etexcmds)             that some package has redefined \expanded.
(etexcmds)             In the latter case, load this package earlier.
) (/etc/texmf/tex/latex/config/hyperref.cfg
File: hyperref.cfg 2002/06/06 v1.2 hyperref configuration of TeXLive
) (/usr/share/texmf-texlive/tex/latex/oberdiek/kvoptions.sty
Package: kvoptions 2009/08/13 v3.4 Keyval support for LaTeX options (HO)
(/usr/share/texmf-texlive/tex/generic/oberdiek/kvsetkeys.sty
Package: kvsetkeys 2009/07/30 v1.5 Key value parser with default handler suppor
t (HO)
))
Package hyperref Info: Option `hyperfootnotes' set `false' on input line 2864.
Package hyperref Info: Option `bookmarks' set `true' on input line 2864.
Package hyperref Info: Option `linktocpage' set `false' on input line 2864.
Package hyperref Info: Option `pdfdisplaydoctitle' set `true' on input line 286
4.
Package hyperref Info: Option `pdfpagelabels' set `true' on input line 2864.
Package hyperref Info: Option `bookmarksopen' set `true' on input line 2864.
Package hyperref Info: Option `colorlinks' set `true' on input line 2864.
Package hyperref Info: Hyper figures OFF on input line 2975.
Package hyperref Info: Link nesting OFF on input line 2980.
Package hyperref Info: Hyper index ON on input line 2983.
Package hyperref Info: Plain pages OFF on input line 2990.
Package hyperref Info: Backreferencing OFF on input line 2995.
Implicit mode ON; LaTeX internals redefined
Package hyperref Info: Bookmarks ON on input line 3191.
(/usr/share/texmf-texlive/tex/latex/ltxmisc/url.sty
\Urlmuskip=\muskip11
Package: url 2006/04/12  ver 3.3  Verb mode for urls, etc.
)
LaTeX Info: Redefining \url on input line 3428.
(/usr/share/texmf-texlive/tex/generic/oberdiek/bitset.sty
Package: bitset 2007/09/28 v1.0 Data type bit set (HO)
(/usr/share/texmf-texlive/tex/generic/oberdiek/intcalc.sty
Package: intcalc 2007/09/27 v1.1 Expandable integer calculations (HO)
) (/usr/share/texmf-texlive/tex/generic/oberdiek/bigintcalc.sty
Package: bigintcalc 2007/11/11 v1.1 Expandable big integer calculations (HO)
(/usr/share/texmf-texlive/tex/generic/oberdiek/pdftexcmds.sty
Package: pdftexcmds 2009/09/23 v0.6 LuaTeX support for pdfTeX utility functions
 (HO)
(/usr/share/texmf-texlive/tex/generic/oberdiek/ifluatex.sty
Package: ifluatex 2009/04/17 v1.2 Provides the ifluatex switch (HO)
Package ifluatex Info: LuaTeX not detected.
) (/usr/share/texmf-texlive/tex/generic/oberdiek/ltxcmds.sty
Package: ltxcmds 2009/08/05 v1.0 Some LaTeX kernel commands for general use (HO
)
)
Package pdftexcmds Info: LuaTeX not detected.
Package pdftexcmds Info: \pdf@primitive is available.
Package pdftexcmds Info: \pdf@ifprimitive is available.
)))
\Fld@menulength=\count120
\Field@Width=\dimen136
\Fld@charsize=\dimen137
\Field@toks=\toks24
Package hyperref Info: Hyper figures OFF on input line 4377.
Package hyperref Info: Link nesting OFF on input line 4382.
Package hyperref Info: Hyper index ON on input line 4385.
Package hyperref Info: backreferencing OFF on input line 4392.
Package hyperref Info: Link coloring ON on input line 4395.
Package hyperref Info: Link coloring with OCG OFF on input line 4402.
Package hyperref Info: PDF/A mode OFF on input line 4407.
(/usr/share/texmf-texlive/tex/generic/oberdiek/atbegshi.sty
Package: atbegshi 2008/07/31 v1.9 At begin shipout hook (HO)
)
\Hy@abspage=\count121
\c@Item=\count122
)
*hyperref using driver hpdftex*
(/usr/share/texmf-texlive/tex/latex/hyperref/hpdftex.def
File: hpdftex.def 2009/10/09 v6.79a Hyperref driver for pdfTeX
\Fld@listcount=\count123
)
\TmpLen=\skip60
\@indexfile=\write3
\openout3 = `41568-t.idx'.

Writing index file 41568-t.idx
(./41568-t.aux)
\openout1 = `41568-t.aux'.

LaTeX Font Info:    Checking defaults for OML/cmm/m/it on input line 494.
LaTeX Font Info:    ... okay on input line 494.
LaTeX Font Info:    Checking defaults for T1/cmr/m/n on input line 494.
LaTeX Font Info:    ... okay on input line 494.
LaTeX Font Info:    Checking defaults for OT1/cmr/m/n on input line 494.
LaTeX Font Info:    ... okay on input line 494.
LaTeX Font Info:    Checking defaults for OMS/cmsy/m/n on input line 494.
LaTeX Font Info:    ... okay on input line 494.
LaTeX Font Info:    Checking defaults for OMX/cmex/m/n on input line 494.
LaTeX Font Info:    ... okay on input line 494.
LaTeX Font Info:    Checking defaults for U/cmr/m/n on input line 494.
LaTeX Font Info:    ... okay on input line 494.
LaTeX Font Info:    Checking defaults for PD1/pdf/m/n on input line 494.
LaTeX Font Info:    ... okay on input line 494.
(/usr/share/texmf/tex/context/base/supp-pdf.mkii
[Loading MPS to PDF converter (version 2006.09.02).]
\scratchcounter=\count124
\scratchdimen=\dimen138
\scratchbox=\box54
\nofMPsegments=\count125
\nofMParguments=\count126
\everyMPshowfont=\toks25
\MPscratchCnt=\count127
\MPscratchDim=\dimen139
\MPnumerator=\count128
\everyMPtoPDFconversion=\toks26
)
Package caption Info: Begin \AtBeginDocument code.
Package caption Info: hyperref package is loaded.
Package caption Info: End \AtBeginDocument code.
*geometry auto-detecting driver*
*geometry detected driver: pdftex*
-------------------- Geometry parameters
paper: class default
landscape: --
twocolumn: --
twoside: true
asymmetric: --
h-parts: 9.03375pt, 307.14749pt, 9.03375pt
v-parts: 1.26749pt, 466.58623pt, 1.90128pt
hmarginratio: 1:1
vmarginratio: 2:3
lines: --
heightrounded: --
bindingoffset: 0.0pt
truedimen: --
includehead: true
includefoot: true
includemp: --
driver: pdftex
-------------------- Page layout dimensions and switches
\paperwidth  325.215pt
\paperheight 469.75499pt
\textwidth  307.14749pt
\textheight 404.71243pt
\oddsidemargin  -63.23624pt
\evensidemargin -63.23624pt
\topmargin  -71.0025pt
\headheight 12.0pt
\headsep    19.8738pt
\footskip   30.0pt
\marginparwidth 98.0pt
\marginparsep   7.0pt
\columnsep  10.0pt
\skip\footins  10.8pt plus 4.0pt minus 2.0pt
\hoffset 0.0pt
\voffset 0.0pt
\mag 1000
\@twosidetrue \@mparswitchtrue 
(1in=72.27pt, 1cm=28.45pt)
-----------------------
(/usr/share/texmf-texlive/tex/latex/graphics/color.sty
Package: color 2005/11/14 v1.0j Standard LaTeX Color (DPC)
(/etc/texmf/tex/latex/config/color.cfg
File: color.cfg 2007/01/18 v1.5 color configuration of teTeX/TeXLive
)
Package color Info: Driver file: pdftex.def on input line 130.
)
Package hyperref Info: Link coloring ON on input line 494.
(/usr/share/texmf-texlive/tex/latex/hyperref/nameref.sty
Package: nameref 2007/05/29 v2.31 Cross-referencing by name of section
(/usr/share/texmf-texlive/tex/latex/oberdiek/refcount.sty
Package: refcount 2008/08/11 v3.1 Data extraction from references (HO)
)
\c@section@level=\count129
)
LaTeX Info: Redefining \ref on input line 494.
LaTeX Info: Redefining \pageref on input line 494.
(./41568-t.out) (./41568-t.out)
\@outlinefile=\write4
\openout4 = `41568-t.out'.

\AtBeginShipoutBox=\box55

Overfull \hbox (20.10721pt too wide) in paragraph at lines 501--501
[]\OT1/cmtt/m/n/8 Project Gutenberg's An Introduction to Mathematics, by Alfred
 North Whitehead[] 
 []


Overfull \hbox (15.85715pt too wide) in paragraph at lines 519--519
[]\OT1/cmtt/m/n/8 *** START OF THIS PROJECT GUTENBERG EBOOK AN INTRODUCTION TO 
MATHEMATICS ***[] 
 []

LaTeX Font Info:    Try loading font information for U+msa on input line 521.
(/usr/share/texmf-texlive/tex/latex/amsfonts/umsa.fd
File: umsa.fd 2009/06/22 v3.00 AMS symbols A
)
LaTeX Font Info:    Try loading font information for U+msb on input line 521.
(/usr/share/texmf-texlive/tex/latex/amsfonts/umsb.fd
File: umsb.fd 2009/06/22 v3.00 AMS symbols B
) [1

{/var/lib/texmf/fonts/map/pdftex/updmap/pdftex.map}] [2] [1


] [2

] [1



] [2] [3] [4] [5] [6] [7

] [8] [9] [10] <./images/fig1.pdf, id=236, 299.1175pt x 166.6225pt>
File: ./images/fig1.pdf Graphic file (type pdf)
<use ./images/fig1.pdf> [11 <./images/fig1.pdf>] <./images/fig2.pdf, id=255, 31
1.1625pt x 166.6225pt>
File: ./images/fig2.pdf Graphic file (type pdf)
<use ./images/fig2.pdf> [12] [13 <./images/fig2.pdf>] [14] [15

] [16] [17] [18] [19] [20] [21]
LaTeX Font Info:    Try loading font information for OMS+cmr on input line 1529
.
(/usr/share/texmf-texlive/tex/latex/base/omscmr.fd
File: omscmr.fd 1999/05/25 v2.5h Standard LaTeX font definitions
)
LaTeX Font Info:    Font shape `OMS/cmr/m/n' in size <8> not available
(Font)              Font shape `OMS/cmsy/m/n' tried instead on input line 1529.

LaTeX Font Info:    Font shape `OMS/cmr/m/n' in size <7> not available
(Font)              Font shape `OMS/cmsy/m/n' tried instead on input line 1531.

[22] [23] [24] [25] <./images/fig3.pdf, id=327, 324.21124pt x 150.5625pt>
File: ./images/fig3.pdf Graphic file (type pdf)
<use ./images/fig3.pdf> [26] [27 <./images/fig3.pdf>] [28] [29

] [30] <./images/fig4.pdf, id=362, 150.5625pt x 128.48pt>
File: ./images/fig4.pdf Graphic file (type pdf)
<use ./images/fig4.pdf> <./images/fig5.pdf, id=364, 150.5625pt x 126.4725pt>
File: ./images/fig5.pdf Graphic file (type pdf)
<use ./images/fig5.pdf> [31] [32 <./images/fig4.pdf> <./images/fig5.pdf>] [33] 
[34] [35] [36] <./images/fig6.pdf, id=405, 162.6075pt x 136.51pt>
File: ./images/fig6.pdf Graphic file (type pdf)
<use ./images/fig6.pdf> [37 <./images/fig6.pdf>] [38] [39] <./images/fig7.pdf, 
id=428, 230.8625pt x 161.60374pt>
File: ./images/fig7.pdf Graphic file (type pdf)
<use ./images/fig7.pdf> [40] [41 <./images/fig7.pdf>] [42] [43

] [44] [45] [46] [47] [48] [49] [50] [51] [52] [53] [54

] <./images/pg76.pdf, id=515, 302.12875pt x 36.135pt>
File: ./images/pg76.pdf Graphic file (type pdf)
<use ./images/pg76.pdf> [55] [56 <./images/pg76.pdf>] [57] [58] [59] [60] [61] 
[62] [63] <./images/pg86.pdf, id=567, 324.21124pt x 32.12pt>
File: ./images/pg86.pdf Graphic file (type pdf)
<use ./images/pg86.pdf> [64 <./images/pg86.pdf>] [65] [66] [67

] [68] [69] [70] <./images/fig8.pdf, id=617, 341.275pt x 230.8625pt>
File: ./images/fig8.pdf Graphic file (type pdf)
<use ./images/fig8.pdf> [71 <./images/fig8.pdf>] <./images/fig9.pdf, id=633, 33
8.26375pt x 239.89626pt>
File: ./images/fig9.pdf Graphic file (type pdf)
<use ./images/fig9.pdf> [72] [73 <./images/fig9.pdf>] [74] [75] [76] <./images/
fig10.pdf, id=672, 341.275pt x 236.885pt>
File: ./images/fig10.pdf Graphic file (type pdf)
<use ./images/fig10.pdf> [77 <./images/fig10.pdf>] [78] [79] [80

] [81] [82] [83] <./images/fig11.pdf, id=720, 319.1925pt x 166.6225pt>
File: ./images/fig11.pdf Graphic file (type pdf)
<use ./images/fig11.pdf> [84] [85 <./images/fig11.pdf>] <./images/fig12.pdf, id
=742, 250.9375pt x 199.74625pt>
File: ./images/fig12.pdf Graphic file (type pdf)
<use ./images/fig12.pdf> [86 <./images/fig12.pdf>] [87] [88] [89] <./images/fig
13.pdf, id=768, 319.1925pt x 274.02374pt>
File: ./images/fig13.pdf Graphic file (type pdf)
<use ./images/fig13.pdf> [90

] [91 <./images/fig13.pdf>] [92] [93] [94] [95] [96] [97] [98] <./images/fig14.
pdf, id=819, 319.1925pt x 279.0425pt>
File: ./images/fig14.pdf Graphic file (type pdf)
<use ./images/fig14.pdf> [99] [100 <./images/fig14.pdf>] [101] [102] [103

] [104] <./images/fig15.pdf, id=864, 337.26pt x 357.335pt>
File: ./images/fig15.pdf Graphic file (type pdf)
<use ./images/fig15.pdf> [105] <./images/fig16.pdf, id=870, 290.08376pt x 156.5
85pt>
File: ./images/fig16.pdf Graphic file (type pdf)
<use ./images/fig16.pdf> <./images/fig17.pdf, id=871, 199.74625pt x 186.6975pt>
File: ./images/fig17.pdf Graphic file (type pdf)
<use ./images/fig17.pdf> <./images/fig18.pdf, id=875, 351.3125pt x 257.96375pt>
File: ./images/fig18.pdf Graphic file (type pdf)
<use ./images/fig18.pdf> [106 <./images/fig15.pdf>] [107 <./images/fig16.pdf>] 
[108 <./images/fig17.pdf>] [109 <./images/fig18.pdf>] [110] [111] [112] <./imag
es/fig19.pdf, id=959, 178.6675pt x 131.49126pt>
File: ./images/fig19.pdf Graphic file (type pdf)
<use ./images/fig19.pdf> [113] [114 <./images/fig19.pdf>] [115] [116] [117] [11
8

] [119] [120] <./images/fig20.pdf, id=1007, 275.0275pt x 120.45pt>
File: ./images/fig20.pdf Graphic file (type pdf)
<use ./images/fig20.pdf> [121 <./images/fig20.pdf>] [122] [123] <./images/fig21
.pdf, id=1039, 305.14pt x 230.8625pt>
File: ./images/fig21.pdf Graphic file (type pdf)
<use ./images/fig21.pdf> [124] [125 <./images/fig21.pdf>] [126] [127] [128] [12
9] [130] [131] [132] [133] [134

] [135] [136] [137] [138] [139] [140] [141

] <./images/fig22.pdf, id=1139, 163.61125pt x 135.50626pt>
File: ./images/fig22.pdf Graphic file (type pdf)
<use ./images/fig22.pdf> <./images/fig23.pdf, id=1140, 145.54375pt x 92.345pt>
File: ./images/fig23.pdf Graphic file (type pdf)
<use ./images/fig23.pdf> [142] [143 <./images/fig22.pdf>] <./images/fig24.pdf, 
id=1156, 348.30125pt x 143.53625pt>
File: ./images/fig24.pdf Graphic file (type pdf)
<use ./images/fig24.pdf> <./images/fig25.pdf, id=1157, 202.7575pt x 126.4725pt>
File: ./images/fig25.pdf Graphic file (type pdf)
<use ./images/fig25.pdf> <./images/fig26.pdf, id=1160, 309.155pt x 265.99374pt>
File: ./images/fig26.pdf Graphic file (type pdf)
<use ./images/fig26.pdf> [144 <./images/fig23.pdf>] [145 <./images/fig24.pdf>] 
[146 <./images/fig25.pdf>] [147] [148 <./images/fig26.pdf>] [149] <./images/fig
27.pdf, id=1226, 351.3125pt x 291.0875pt>
File: ./images/fig27.pdf Graphic file (type pdf)
<use ./images/fig27.pdf> [150] [151] [152 <./images/fig27.pdf>] [153] [154] <./
images/fig28.pdf, id=1267, 298.11375pt x 69.25874pt>
File: ./images/fig28.pdf Graphic file (type pdf)
<use ./images/fig28.pdf> [155 <./images/fig28.pdf>] [156] [157] [158] [159

] [160] [161] [162] [163] [164] [165] [166]
Overfull \hbox (0.85368pt too wide) in paragraph at lines 6806--6814
[]\OT1/cmr/m/n/12 It is tempt-ing to sup-pose that the con-di-tion for $\OML/cm
m/m/it/12 u[]$\OT1/cmr/m/n/12 , $\OML/cmm/m/it/12 u[]$\OT1/cmr/m/n/12 , ...,
 []

[167] [168] [169] [170] [171] [172]
Underfull \hbox (badness 1565) in paragraph at lines 7074--7079
[]\OT1/cmr/m/n/12 The crit-i-cal points, where non-uniform con-ver-gence
 []

[173] [174] <./images/fig29.pdf, id=1377, 341.275pt x 183.68625pt>
File: ./images/fig29.pdf Graphic file (type pdf)
<use ./images/fig29.pdf> [175] <./images/fig30.pdf, id=1385, 305.14pt x 129.483
75pt>
File: ./images/fig30.pdf Graphic file (type pdf)
<use ./images/fig30.pdf> [176 <./images/fig29.pdf>] <./images/fig31.pdf, id=140
0, 293.095pt x 149.55875pt>
File: ./images/fig31.pdf Graphic file (type pdf)
<use ./images/fig31.pdf> [177 <./images/fig30.pdf>] [178 <./images/fig31.pdf>] 
[179

] [180] [181] <./images/fig32.pdf, id=1442, 309.155pt x 166.6225pt>
File: ./images/fig32.pdf Graphic file (type pdf)
<use ./images/fig32.pdf> [182 <./images/fig32.pdf>] [183] [184] [185] [186] [18
7] [188] [189] [190] [191] [192] [193] <./images/fig33.pdf, id=1514, 339.2675pt
 x 114.4275pt>
File: ./images/fig33.pdf Graphic file (type pdf)
<use ./images/fig33.pdf> [194

] [195 <./images/fig33.pdf>] [196] [197] [198] [199] [200] [201] [202

] [203] [204] [205] [206] [207



] [208] [209

] [210] (./41568-t.ind [211] [212

] [213] [214] [215])
Overfull \hbox (7.35703pt too wide) in paragraph at lines 9490--9490
[]\OT1/cmtt/m/n/8 *** END OF THIS PROJECT GUTENBERG EBOOK AN INTRODUCTION TO MA
THEMATICS ***[] 
 []

[1


]
Overfull \hbox (3.10696pt too wide) in paragraph at lines 9560--9560
[]\OT1/cmtt/m/n/8 1.C.  The Project Gutenberg Literary Archive Foundation ("the
 Foundation"[] 
 []


Overfull \hbox (3.10696pt too wide) in paragraph at lines 9565--9565
[]\OT1/cmtt/m/n/8 located in the United States, we do not claim a right to prev
ent you from[] 
 []

[2]
Overfull \hbox (3.10696pt too wide) in paragraph at lines 9570--9570
[]\OT1/cmtt/m/n/8 freely sharing Project Gutenberg-tm works in compliance with 
the terms of[] 
 []

[3]
Overfull \hbox (3.10696pt too wide) in paragraph at lines 9633--9633
[]\OT1/cmtt/m/n/8 posted on the official Project Gutenberg-tm web site (www.gut
enberg.org),[] 
 []

[4] [5] [6] [7] [8] [9] (./41568-t.aux)

 *File List*
    book.cls    2007/10/19 v1.4h Standard LaTeX document class
   leqno.clo    1998/08/17 v1.1c Standard LaTeX option (left equation numbers)
    bk12.clo    2007/10/19 v1.4h Standard LaTeX file (size option)
inputenc.sty    2008/03/30 v1.1d Input encoding file
  latin1.def    2008/03/30 v1.1d Input encoding file
  ifthen.sty    2001/05/26 v1.1c Standard LaTeX ifthen package (DPC)
 amsmath.sty    2000/07/18 v2.13 AMS math features
 amstext.sty    2000/06/29 v2.01
  amsgen.sty    1999/11/30 v2.0
  amsbsy.sty    1999/11/29 v1.2d
  amsopn.sty    1999/12/14 v2.01 operator names
 amssymb.sty    2009/06/22 v3.00
amsfonts.sty    2009/06/22 v3.00 Basic AMSFonts support
   alltt.sty    1997/06/16 v2.0g defines alltt environment
indentfirst.sty    1995/11/23 v1.03 Indent first paragraph (DPC)
footmisc.sty    2009/09/15 v5.5a a miscellany of footnote facilities
multicol.sty    2008/12/05 v1.6h multicolumn formatting (FMi)
 makeidx.sty    2000/03/29 v1.0m Standard LaTeX package
graphicx.sty    1999/02/16 v1.0f Enhanced LaTeX Graphics (DPC,SPQR)
  keyval.sty    1999/03/16 v1.13 key=value parser (DPC)
graphics.sty    2009/02/05 v1.0o Standard LaTeX Graphics (DPC,SPQR)
    trig.sty    1999/03/16 v1.09 sin cos tan (DPC)
graphics.cfg    2009/08/28 v1.8 graphics configuration of TeX Live
  pdftex.def    2009/08/25 v0.04m Graphics/color for pdfTeX
 caption.sty    2009/10/09 v3.1k Customizing captions (AR)
caption3.sty    2009/10/09 v3.1k caption3 kernel (AR)
    calc.sty    2007/08/22 v4.3 Infix arithmetic (KKT,FJ)
fancyhdr.sty    
geometry.sty    2008/12/21 v4.2 Page Geometry
   ifpdf.sty    2009/04/10 v2.0 Provides the ifpdf switch (HO)
  ifvtex.sty    2008/11/04 v1.4 Switches for detecting VTeX and its modes (HO)
geometry.cfg
hyperref.sty    2009/10/09 v6.79a Hypertext links for LaTeX
 ifxetex.sty    2009/01/23 v0.5 Provides ifxetex conditional
 hycolor.sty    2009/10/02 v1.5 Code for color options of hyperref/bookmark (HO
)
xcolor-patch.sty    2009/10/02 xcolor patch
  pd1enc.def    2009/10/09 v6.79a Hyperref: PDFDocEncoding definition (HO)
etexcmds.sty    2007/12/12 v1.2 Prefix for e-TeX command names (HO)
infwarerr.sty    2007/09/09 v1.2 Providing info/warning/message (HO)
hyperref.cfg    2002/06/06 v1.2 hyperref configuration of TeXLive
kvoptions.sty    2009/08/13 v3.4 Keyval support for LaTeX options (HO)
kvsetkeys.sty    2009/07/30 v1.5 Key value parser with default handler support 
(HO)
     url.sty    2006/04/12  ver 3.3  Verb mode for urls, etc.
  bitset.sty    2007/09/28 v1.0 Data type bit set (HO)
 intcalc.sty    2007/09/27 v1.1 Expandable integer calculations (HO)
bigintcalc.sty    2007/11/11 v1.1 Expandable big integer calculations (HO)
pdftexcmds.sty    2009/09/23 v0.6 LuaTeX support for pdfTeX utility functions (
HO)
ifluatex.sty    2009/04/17 v1.2 Provides the ifluatex switch (HO)
 ltxcmds.sty    2009/08/05 v1.0 Some LaTeX kernel commands for general use (HO)

atbegshi.sty    2008/07/31 v1.9 At begin shipout hook (HO)
 hpdftex.def    2009/10/09 v6.79a Hyperref driver for pdfTeX
supp-pdf.mkii
   color.sty    2005/11/14 v1.0j Standard LaTeX Color (DPC)
   color.cfg    2007/01/18 v1.5 color configuration of teTeX/TeXLive
 nameref.sty    2007/05/29 v2.31 Cross-referencing by name of section
refcount.sty    2008/08/11 v3.1 Data extraction from references (HO)
 41568-t.out
 41568-t.out
    umsa.fd    2009/06/22 v3.00 AMS symbols A
    umsb.fd    2009/06/22 v3.00 AMS symbols B
./images/fig1.pdf
./images/fig2.pdf
  omscmr.fd    1999/05/25 v2.5h Standard LaTeX font definitions
./images/fig3.pdf
./images/fig4.pdf
./images/fig5.pdf
./images/fig6.pdf
./images/fig7.pdf
./images/pg76.pdf
./images/pg86.pdf
./images/fig8.pdf
./images/fig9.pdf
./images/fig10.pdf
./images/fig11.pdf
./images/fig12.pdf
./images/fig13.pdf
./images/fig14.pdf
./images/fig15.pdf
./images/fig16.pdf
./images/fig17.pdf
./images/fig18.pdf
./images/fig19.pdf
./images/fig20.pdf
./images/fig21.pdf
./images/fig22.pdf
./images/fig23.pdf
./images/fig24.pdf
./images/fig25.pdf
./images/fig26.pdf
./images/fig27.pdf
./images/fig28.pdf
./images/fig29.pdf
./images/fig30.pdf
./images/fig31.pdf
./images/fig32.pdf
./images/fig33.pdf
 41568-t.ind
 ***********

 ) 
Here is how much of TeX's memory you used:
 7739 strings out of 493848
 110834 string characters out of 1152823
 197518 words of memory out of 3000000
 10386 multiletter control sequences out of 15000+50000
 16794 words of font info for 64 fonts, out of 3000000 for 9000
 714 hyphenation exceptions out of 8191
 37i,13n,44p,298b,544s stack positions out of 5000i,500n,10000p,200000b,50000s
</usr/share/texmf-texlive/fonts/type1/public/amsfonts/cm/cmcsc10.pfb></usr/sh
are/texmf-texlive/fonts/type1/public/amsfonts/cm/cmex10.pfb></usr/share/texmf-t
exlive/fonts/type1/public/amsfonts/cm/cmmi10.pfb></usr/share/texmf-texlive/font
s/type1/public/amsfonts/cm/cmmi12.pfb></usr/share/texmf-texlive/fonts/type1/pub
lic/amsfonts/cm/cmmi8.pfb></usr/share/texmf-texlive/fonts/type1/public/amsfonts
/cm/cmr10.pfb></usr/share/texmf-texlive/fonts/type1/public/amsfonts/cm/cmr12.pf
b></usr/share/texmf-texlive/fonts/type1/public/amsfonts/cm/cmr17.pfb></usr/shar
e/texmf-texlive/fonts/type1/public/amsfonts/cm/cmr8.pfb></usr/share/texmf-texli
ve/fonts/type1/public/amsfonts/cm/cmssi12.pfb></usr/share/texmf-texlive/fonts/t
ype1/public/amsfonts/cm/cmsy10.pfb></usr/share/texmf-texlive/fonts/type1/public
/amsfonts/cm/cmsy7.pfb></usr/share/texmf-texlive/fonts/type1/public/amsfonts/cm
/cmsy8.pfb></usr/share/texmf-texlive/fonts/type1/public/amsfonts/cm/cmti10.pfb>
</usr/share/texmf-texlive/fonts/type1/public/amsfonts/cm/cmti12.pfb></usr/share
/texmf-texlive/fonts/type1/public/amsfonts/cm/cmtt10.pfb></usr/share/texmf-texl
ive/fonts/type1/public/amsfonts/cm/cmtt8.pfb></usr/share/texmf-texlive/fonts/ty
pe1/public/amsfonts/cm/cmu10.pfb>
Output written on 41568-t.pdf (228 pages, 1021150 bytes).
PDF statistics:
 2077 PDF objects out of 2487 (max. 8388607)
 462 named destinations out of 1000 (max. 500000)
 392 words of extra memory for PDF output out of 10000 (max. 10000000)

